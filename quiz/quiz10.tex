\documentclass[11pt]{article}
\usepackage{calc}
\usepackage[margin={1in,0.5in},footskip=0in]{geometry}
\usepackage[miniquiz]{hwk}
\usepackage{tikz, textcomp}

\include{mathcmds}

\renewcommand{\theclass}{math 1300}
\renewcommand{\dateinfo}{April 23, 2013}
\renewcommand{\theassignment}{Quiz 10}

\begin{document}
\pagestyle{empty}
\newsavebox{\quizfront}
\begin{lrbox}{\quizfront}
\begin{minipage}[top][4.5in][t]{\textwidth} \setlength{\parindent}{1.5em}
\drawtitle
\vspace{-0.5in}
\begin{enumerate}

\item Given the graph of $f(x)$, below, sketch the graph of
  \emph{two} different antiderivatives for $f$.
  \vfill
  \begin{center}
    \begin{tikzpicture}[scale=1.15]
      \def\xmin{-4}
      \def\xmax{4}
      \def\ymin{-3}
      \def\ymax{3}
      \draw[<->] (\xmin,0) -- (\xmax,0);
      \draw[<->] (0,\ymin) -- (0,\ymax);
      \draw[thick, domain=\xmin:\xmax] plot[samples=100]
      function{-1.5*sin(x)};
    \end{tikzpicture}
  \end{center}

  \vfill

\end{enumerate}

\hfill \textsc{over} $\longrightarrow$

\end{minipage}
\end{lrbox}

%%%%%%%%%%%%%%%%%%%%%%%%%%%%%%%%%%%%%%%%%%%%%%%%%%%%%%
%%%% This is for the back of the quiz
%%%%%%%%%%%%%%%%%%%%%%%%%%%%%%%%%%%%%%%%%%%%%%%%%%%%%%
\newsavebox{\quizback}
\begin{lrbox}{\quizback}
\begin{minipage}[top][4.5in][t]{\textwidth} \setlength{\parindent}{1.5em}
\begin{enumerate}
\item[2.] Suppose that $f(x)$ is a function with the following properties:
  \begin{align*}
    \int_1^5 f(x)\;dx = 11 &&& \int_3^5 f(x)\;dx = 7
  \end{align*}
  Calculate the values of
  \begin{enumerate}
  \item $\displaystyle \int_1^5 2\cdot f(x)\;dx$
    \vfill
  \item $\displaystyle \int_1^3 f(x)\;dx$
    \vfill
  \item $\displaystyle \int_3^5 (f(x) + 3)\;dx$
    \vfill
  \end{enumerate}

\end{enumerate}
\end{minipage}
\end{lrbox}

%%%%%%%%%%%%%%%%%%%%%%%%%%%%%%%%%%%%%%%%%%%%%%%%%%%%%%
%%%%
%%%% Now we make two copies of the ``quizfront'' box
%%%%
%%%%%%%%%%%%%%%%%%%%%%%%%%%%%%%%%%%%%%%%%%%%%%%%%%%%%%%
\noindent \usebox{\quizfront}
\vfill
\noindent \usebox{\quizfront}

%%%% Uncomment the rest to have a two-sided quiz.
\pagebreak
\noindent \usebox{\quizback}
\vfill
\noindent \usebox{\quizback}
\end{document}