\documentclass[11pt]{article}
\usepackage{calc}
\usepackage[margin={1in,0.5in},footskip=0in]{geometry}
\usepackage[miniquiz]{hwk}
\usepackage{tikz, textcomp}

\include{mathcmds}

\renewcommand{\theclass}{math 1300}
\renewcommand{\dateinfo}{April 2, 2013}
\renewcommand{\theassignment}{Quiz 8}

\begin{document}
\pagestyle{empty}
\newsavebox{\quizfront}
\begin{lrbox}{\quizfront}
\begin{minipage}[top][4.5in][t]{\textwidth} \setlength{\parindent}{1.5em}
\drawtitle
\vspace{-0.5in}
\begin{enumerate}

\item The revenue from selling $q$ items is $R(q) = 500q - q^2$, and
  the total cost is $C(q) = 10000 + 3q^2$. At what quantity is profit
  maximized?

  \vfill
  {\color{blue}

    We know that $\pi(q) = R(q) - C(q)$, so
    \[
    \pi(q) = 500q - q^2 - (10000 + 3q^2) = -4q^2 + 500q - 10000.
    \]
    Then $\pi'(q) = -8q + 500$, so $\pi(q)$ has a critical point at $q
    = 62.5$. We know this critical point is a global maximum since
    $\pi(q)$ is a parabola with a negative leading coefficient. Then
    producing $62.5$ items will maximize out profit.

  }
  \vfill

\end{enumerate}

%\hfill \textsc{over} $\longrightarrow$

\end{minipage}
\end{lrbox}

%%%%%%%%%%%%%%%%%%%%%%%%%%%%%%%%%%%%%%%%%%%%%%%%%%%%%%
%%%% This is for the back of the quiz
%%%%%%%%%%%%%%%%%%%%%%%%%%%%%%%%%%%%%%%%%%%%%%%%%%%%%%
\newsavebox{\quizback}
\begin{lrbox}{\quizback}
\begin{minipage}[top][4.5in][t]{\textwidth} \setlength{\parindent}{1.5em}
\begin{enumerate}
\item[2.] If you have 100 meters of fencing and want to enclose a
  rectangular area up against a long, straight wall, what dimensions
  maximize the area you can enclose?

  \vfill
  {\color{blue}

    We know that the area of a rectangular fence of length $\ell$ and
    width $w$ is given by
    \[
    A = \ell w.
    \]
    If the lengh of the fence is against the wall, we know that the
    total amount of fencing is given by $2w + \ell$. Since we have 100
    feet of fencing, we obtain
    \[
    100 = 2w + \ell,
    \]
    so $\ell = 100 - 2w$. Then we have
    \[
    A = (100-2w)w = 100w - 2w^2.
    \]
    Taking the derivative we see that $\frac{dA}{dw} = 100 - 4w$, so
    we have a critical point at $w = 25$. We know this critical point
    is a global maximum since $A$ is a parabola with a negative
    leading coefficient. Then a fence with a width of 25 feet and a
    length of 50 feet will maximize the area we can enclose.

  }
  \vfill

\end{enumerate}
\end{minipage}
\end{lrbox}

%%%%%%%%%%%%%%%%%%%%%%%%%%%%%%%%%%%%%%%%%%%%%%%%%%%%%%
%%%%
%%%% Now we make two copies of the ``quizfront'' box
%%%%
%%%%%%%%%%%%%%%%%%%%%%%%%%%%%%%%%%%%%%%%%%%%%%%%%%%%%%%
\noindent \usebox{\quizfront}
\vfill
\noindent \usebox{\quizback}

%%%% Uncomment the rest to have a two-sided quiz.
%\pagebreak
%\noindent \usebox{\quizback}
%\vfill
%\noindent \usebox{\quizback}
\end{document}