\documentclass[11pt]{article}
\usepackage{calc}
\usepackage[margin={1in,0.5in},footskip=0in]{geometry}
\usepackage[miniquiz]{hwk}
\usepackage{tikz, textcomp}

\include{mathcmds}

\renewcommand{\theclass}{math 1300}
\renewcommand{\dateinfo}{March 19, 2013}
\renewcommand{\theassignment}{Quiz 7}

\begin{document}
\pagestyle{empty}
\newsavebox{\quizfront}
\begin{lrbox}{\quizfront}
\begin{minipage}[top][3.5in][t]{\textwidth} \setlength{\parindent}{1.5em}
\drawtitle
\vspace{-0.5in}
\begin{enumerate}

\item Find and classify all local extrema of $f(x) = x^3-3x+1$.

  \vfill
  {\color{blue}

    We see that $f'(x) = 3x^2-3$. Since $f'(x)$ is a polynomial, it is
    defined everywhere, so critical points occur where $f'(x) =
    0$. Then we have critical points at
    \begin{align*}
      3x^2 - 3 &= 0\\
      3(x^2-1) &= 0\\
      (x+1)(x-1) &= 0\\
      x = \pm 1.
    \end{align*}
    We see that $f''(x) = 6x$, so $f''(1) > 0$ and $f''(-1) <
    0$. Then, by the second derivative test, $f(x)$ has a local max of
    $f(-1) = 3$ at $x= -1$ and a local min of $f(1) = -1$ at $x = 1$.

  }
  \vfill

\end{enumerate}

%\hfill \textsc{over} $\longrightarrow$

\end{minipage}
\end{lrbox}

%%%%%%%%%%%%%%%%%%%%%%%%%%%%%%%%%%%%%%%%%%%%%%%%%%%%%%
%%%% This is for the back of the quiz
%%%%%%%%%%%%%%%%%%%%%%%%%%%%%%%%%%%%%%%%%%%%%%%%%%%%%%
\newsavebox{\quizback}
\begin{lrbox}{\quizback}
\begin{minipage}[top][5.5in][t]{\textwidth} \setlength{\parindent}{1.5em}
\begin{enumerate}
\item[2.] Find all global extrema, if they exist, for $f(x) = x +
  \dfrac{1}{x}$ on $(0, \infty)$.
 
  \vfill
  {\color{blue}

    We see that $f'(x) = 1-x^{-2}$. Then $f'(x)$ does not exist at $x
    = 0$, but $x=0$ is not in the domain of $f(x)$, so it cannot be a
    critical point. Thus, critical points of $f(x)$ only occur where
    $f'(x) = 0$. Then
    \begin{align*}
      f'(x) &= 0\\
      1-x^{-2} & = 0\\
      1 &= x^{-2}\\
      x^2 &= 1\\
      x &= \pm 1.
    \end{align*}
    Again, since $x=-1$ is not in our domain, it cannot be a critical
    point, so the only critical point for $f(x)$ occurs at $x =
    1$. Now we can compare the $y$-values at the critical point and
    endpoints:
    \begin{align*}
      \lim_{x\to 0} f(x) &= \infty\\
      f(1) &= 2\\
      \lim_{x\to\infty} f(x) &= \infty.
    \end{align*}
    Then $f(x)$ has a global min of $2$ at $x=1$ and has no global
    max.

  }
  \vfill

\end{enumerate}
\end{minipage}
\end{lrbox}

%%%%%%%%%%%%%%%%%%%%%%%%%%%%%%%%%%%%%%%%%%%%%%%%%%%%%%
%%%%
%%%% Now we make two copies of the ``quizfront'' box
%%%%
%%%%%%%%%%%%%%%%%%%%%%%%%%%%%%%%%%%%%%%%%%%%%%%%%%%%%%%
\noindent \usebox{\quizfront}
\vfill
\noindent \usebox{\quizback}

%%%% Uncomment the rest to have a two-sided quiz.
%\pagebreak
%\noindent \usebox{\quizback}
%\vfill
%\noindent \usebox{\quizback}
\end{document}