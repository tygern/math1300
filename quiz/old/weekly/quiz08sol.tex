\documentclass[11pt]{article}
\usepackage{calc}
\usepackage[margin={1in,0.5in},footskip=0in]{geometry}
\usepackage[miniquiz]{hwk}
\usepackage{tikz, pgfplots}

\renewcommand{\theclass}{math 1300}
\renewcommand{\dateinfo}{November 6, 2012}
\renewcommand{\theassignment}{Quiz 8}

\begin{document}
\pagestyle{empty}
\newsavebox{\quizfront}
\begin{lrbox}{\quizfront}
\begin{minipage}[top][4.5in][t]{\textwidth} \setlength{\parindent}{1.5em}
\drawtitle
\vspace{-0.5in}
\begin{enumerate}

\item Find constants $a$ and $b$ such that $f(x)=x^2+ax+b$ has a
  critical point at $(7,10)$, and classify that critical point.

  \vfill
  {\color{blue}

    Since $f(x)$ has a critical point at $(7,10)$ we know that
    $f(7)=10$ and $f'(7)=0$.  Then
    \begin{align*}
      f'(x) &= 2x+a\\
      0 &= 2\cdot 7+a\\
      a &= -14,
    \end{align*}
    and
    \begin{align*}
      f(x) &= x^2-14x+b\\
      10 &= 49-98+b\\
      b &= 59.
    \end{align*}
    Furthermore, $f''(x) = 2$, so $f''(7)=2$, thus the critical point
    is a local minimum.

  }

\end{enumerate}

\vfill

\end{minipage}
\end{lrbox}

%%%%%%%%%%%%%%%%%%%%%%%%%%%%%%%%%%%%%%%%%%%%%%%%%%%%%%
%%%% This is for the back of the quiz
%%%%%%%%%%%%%%%%%%%%%%%%%%%%%%%%%%%%%%%%%%%%%%%%%%%%%%
\newsavebox{\quizback}
\begin{lrbox}{\quizback}
\begin{minipage}[top][4.5in][t]{\textwidth} \setlength{\parindent}{1.5em}
\begin{enumerate}
\item[2.] Find the global maximum and minimum for the function
  $f(x)=\frac{x+1}{x^2+3}$ on the interval $[-1,2]$.


  \vfill
  {\color{blue}

    We see that $f(x)$ is continuous on $[-1,2]$, so by the extreme
    value theorem we know that both a global maximum and the global
    minimum must exist on this interval.  We see that
    \[
    f'(x)=\frac{x^2+3-2x(x+1)}{(x^2+3)^2} = \frac{-x^2-2x+3}{(x^2+3)^2}.
    \]
    Now $(x^2+3)^2$ is never zero, so the critical points of $f(x)$
    occur when $-x^2-2x+3=0$. This happens when $x=-3$ and when $x=1$,
    but $x=3$ is not in our interval, so our only critical point is at
    $x=1$.  Then
    \begin{align*}
      f(-1) &= 0\\
      f(1) &= \frac{1}{2}\\
      f(2) &= \frac{3}{7},
    \end{align*}
    so $f(x)$ has a global max or $\frac{1}{2}$ at $x=1$ and a global
    min of $0$ at $x=-1$.

  }

\vfill

\end{enumerate}
\end{minipage}
\end{lrbox}

%%%%%%%%%%%%%%%%%%%%%%%%%%%%%%%%%%%%%%%%%%%%%%%%%%%%%%
%%%%
%%%% Now we make two copies of the ``quizfront'' box
%%%%
%%%%%%%%%%%%%%%%%%%%%%%%%%%%%%%%%%%%%%%%%%%%%%%%%%%%%%%
\noindent \usebox{\quizfront}
\vfill
\noindent \usebox{\quizback}

%%%% Uncomment the rest to have a two-sided quiz.
%\pagebreak
%\noindent \usebox{\quizback}
%\vfill
%\noindent \usebox{\quizback}
\end{document}