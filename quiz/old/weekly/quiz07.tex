\documentclass[11pt]{article}
\usepackage{calc}
\usepackage[margin={1in,0.5in},footskip=0in]{geometry}
\usepackage[miniquiz]{hwk}
\usepackage{tikz, pgfplots}

\renewcommand{\theclass}{math 1300}
\renewcommand{\dateinfo}{October 30, 2012}
\renewcommand{\theassignment}{Quiz 7}

\begin{document}
\pagestyle{empty}
\newsavebox{\quizfront}
\begin{lrbox}{\quizfront}
\begin{minipage}[top][4.5in][t]{\textwidth} \setlength{\parindent}{1.5em}
\drawtitle
\vspace{-0.5in}
\begin{enumerate}

\item Suppose $f(x) = -x^2 - 8x + 5$ is linearized at $x=1$. What is
  the approximate value of $\dfrac{E(x)}{(x-1)^2}$? (You do not
  actually need to \emph{find} the linearization to do this.)

\end{enumerate}

\vfill

\hfill\textsc{over} $\longrightarrow$


\end{minipage}
\end{lrbox}

%%%%%%%%%%%%%%%%%%%%%%%%%%%%%%%%%%%%%%%%%%%%%%%%%%%%%%
%%%% This is for the back of the quiz
%%%%%%%%%%%%%%%%%%%%%%%%%%%%%%%%%%%%%%%%%%%%%%%%%%%%%%
\newsavebox{\quizback}
\begin{lrbox}{\quizback}
\begin{minipage}[top][4.5in][t]{\textwidth} \setlength{\parindent}{1.5em}
\begin{enumerate}
\item[2.] Consider the function $f(x) = x^{1/3}$ with $a=-2$ and
  $b=3$. Explain why $f$ does not satisfy the hypotheses of the Mean
  Value Theorem.

\end{enumerate}
\end{minipage}
\end{lrbox}

%%%%%%%%%%%%%%%%%%%%%%%%%%%%%%%%%%%%%%%%%%%%%%%%%%%%%%
%%%%
%%%% Now we make two copies of the ``quizfront'' box
%%%%
%%%%%%%%%%%%%%%%%%%%%%%%%%%%%%%%%%%%%%%%%%%%%%%%%%%%%%%
\noindent \usebox{\quizfront}
\vfill
\noindent \usebox{\quizfront}

%%%% Uncomment the rest to have a two-sided quiz.
\pagebreak
\noindent \usebox{\quizback}
\vfill
\noindent \usebox{\quizback}
\end{document}