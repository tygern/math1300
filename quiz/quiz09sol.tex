\documentclass[11pt]{article}
\usepackage{calc}
\usepackage[margin={1in,0.5in},footskip=0in]{geometry}
\usepackage[miniquiz]{hwk}
\usepackage{tikz, textcomp}

\include{mathcmds}

\renewcommand{\theclass}{math 1300}
\renewcommand{\dateinfo}{April 16, 2013}
\renewcommand{\theassignment}{Quiz 9}

\begin{document}
\pagestyle{empty}
\newsavebox{\quizfront}
\begin{lrbox}{\quizfront}
\begin{minipage}[top][4.5in][t]{\textwidth} \setlength{\parindent}{1.5em}
\drawtitle
\vspace{-0.5in}
\begin{enumerate}

\item Suppose a car moves with a velocity of $v(t) = 16 - 4t$ meters
  per second.  What is the total displacement of the car between $t =
  1$ and $t = 6$?

    \vfill
    {\color{blue}
      
      We know that the displacement of the car is given by
      \[
      \int_1^6 (16-4t)\; dt.
      \]
      Using geometry, we see that
      \[
      \int_1^6 (16-4t)\; dt = \frac{1}{2}\cdot 3\cdot 12 -
      \frac{1}{2}\cdot 2\cdot 8 = 10,
      \]
      so the total displacement of the car between $t=1$ and $t=6$ is
      $10$ meters.
      
    }
  \vfill

\end{enumerate}

%\hfill \textsc{over} $\longrightarrow$

\end{minipage}
\end{lrbox}

%%%%%%%%%%%%%%%%%%%%%%%%%%%%%%%%%%%%%%%%%%%%%%%%%%%%%%
%%%% This is for the back of the quiz
%%%%%%%%%%%%%%%%%%%%%%%%%%%%%%%%%%%%%%%%%%%%%%%%%%%%%%
\newsavebox{\quizback}
\begin{lrbox}{\quizback}
\begin{minipage}[top][4.5in][t]{\textwidth} \setlength{\parindent}{1.5em}
\begin{enumerate}
\item[2.] If you have 100 meters of fencing and want to enclose a
  rectangular area up against a long, straight wall, what dimensions
  maximize the area you can enclose?

\end{enumerate}
\end{minipage}
\end{lrbox}

%%%%%%%%%%%%%%%%%%%%%%%%%%%%%%%%%%%%%%%%%%%%%%%%%%%%%%
%%%%
%%%% Now we make two copies of the ``quizfront'' box
%%%%
%%%%%%%%%%%%%%%%%%%%%%%%%%%%%%%%%%%%%%%%%%%%%%%%%%%%%%%
\noindent \usebox{\quizfront}
\vfill
%\noindent \usebox{\quizfront}

%%%% Uncomment the rest to have a two-sided quiz.
%\pagebreak
%\noindent \usebox{\quizback}
%\vfill
%\noindent \usebox{\quizback}
\end{document}