\documentclass[11pt]{article}
\usepackage{calc}
\usepackage[margin={1in,0.5in},footskip=0in]{geometry}
\usepackage[miniquiz]{hwk}
\usepackage{tikz, textcomp, pgfplots}

\include{mathcmds}

\renewcommand{\theclass}{math 1300}
\renewcommand{\dateinfo}{October 26, 2012}
\renewcommand{\theassignment}{Quiz 4.1}

\begin{document}
\pagestyle{empty}
\newsavebox{\quizfront}
\begin{lrbox}{\quizfront}
\begin{minipage}[top][4.5in][t]{\textwidth} \setlength{\parindent}{1.5em}
\drawtitle
\vspace{-0.5in}
\begin{enumerate}

\item List the $x$-values of all critical points of the function $f(x)$ given below

  \vfill
  \begin{center}
    \begin{tikzpicture}[xscale=1.5]
      \draw[<->] (-2.5,0) -- (6.5,0);
      \draw[<->] (0,-4) -- (0,3.5);

      \foreach \i in {-2,...,6} { \ifnum\i=0 { } \else { \draw[-]
          (\i,.1) -- (\i,-.1) node[below] {$\i$}; } \fi }
      \node at (-.25,-.34) {$0$};

      \draw[domain=-2.5:-.15, thick, <->, samples = 100] plot(\x,{1/(\x*(\x-2))});
      \draw[domain=.15:1.85, thick, <->, samples = 100] plot(\x,{1/(\x*(\x-2))});

      \draw[domain=2.15:5.85, thick, <->, samples = 100]
      plot(\x,{(\x-4)/((\x-6)*(\x-2))}) node[right] {$f(x)$};

    \end{tikzpicture}
    \vfill
    \textbigcircle\; $x=0$\hspace{.7in} \textcircled{{\color{blue}$\checkmark$}}\; $x=1$\hspace{.7in}
    \textbigcircle\; $x=2$\hspace{.7in} \textbigcircle\; $x=4$
  \end{center}
  \vfill


\end{enumerate}
%\begin{flushright}
%Problem on the back $\longrightarrow$
%\end{flushright}



\end{minipage}
\end{lrbox}

%%%%%%%%%%%%%%%%%%%%%%%%%%%%%%%%%%%%%%%%%%%%%%%%%%%%%%
%%%% This is for the back of the quiz
%%%%%%%%%%%%%%%%%%%%%%%%%%%%%%%%%%%%%%%%%%%%%%%%%%%%%%
\newsavebox{\quizback}
\begin{lrbox}{\quizback}
\begin{minipage}[top][4.5in][t]{\textwidth} \setlength{\parindent}{1.5em}
\begin{itemize}
 \item[3.] Find $\ln(e)$. 


\end{itemize}
\end{minipage}
\end{lrbox}

%%%%%%%%%%%%%%%%%%%%%%%%%%%%%%%%%%%%%%%%%%%%%%%%%%%%%%
%%%%
%%%% Now we make two copies of the ``quizfront'' box
%%%%
%%%%%%%%%%%%%%%%%%%%%%%%%%%%%%%%%%%%%%%%%%%%%%%%%%%%%%%
\noindent \usebox{\quizfront}
\vfill
%\noindent \usebox{\quizfront}

%%%% Uncomment the rest to have a two-sided quiz.
%\pagebreak
%\noindent \usebox{\quizback}
%\vfill
%\noindent \usebox{\quizback}
\end{document}