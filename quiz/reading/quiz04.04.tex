\documentclass[11pt]{article}
\usepackage{calc}
\usepackage[margin={1in,0.5in},footskip=0in]{geometry}
\usepackage[miniquiz]{hwk}
\usepackage{tikz, textcomp, pgfplots}

\include{mathcmds}

\renewcommand{\theclass}{math 1300}
\renewcommand{\dateinfo}{November 2, 2012}
\renewcommand{\theassignment}{Quiz 4.4}

\begin{document}
\pagestyle{empty}
\newsavebox{\quizfront}
\begin{lrbox}{\quizfront}
\begin{minipage}[top][4.5in][t]{\textwidth} \setlength{\parindent}{1.5em}
\drawtitle
\vspace{-0.5in}
\begin{enumerate}

\item For which of the functions below does the extreme value theorem
guarantee both a global minimum and a global maximum on the given interval?
(check all that apply)

\vfill

\begin{enumerate}
  \item[\textbigcircle] $f(x)=\frac{1}{x}$ on $[-2,4]$
  \item[\textbigcircle] $g(x)=\sin(x)$ on $(0,\pi)$
  \item[\textbigcircle] $h(x)=4$ on $[5,8]$
  \item[\textbigcircle] $P(x)=\ln(x)$ on $[1,2]$
\end{enumerate}
  \vfill


\end{enumerate}
%\begin{flushright}
%Problem on the back $\longrightarrow$
%\end{flushright}



\end{minipage}
\end{lrbox}

%%%%%%%%%%%%%%%%%%%%%%%%%%%%%%%%%%%%%%%%%%%%%%%%%%%%%%
%%%% This is for the back of the quiz
%%%%%%%%%%%%%%%%%%%%%%%%%%%%%%%%%%%%%%%%%%%%%%%%%%%%%%
\newsavebox{\quizback}
\begin{lrbox}{\quizback}
\begin{minipage}[top][4.5in][t]{\textwidth} \setlength{\parindent}{1.5em}
\begin{itemize}
 \item[3.] Find $\ln(e)$. 


\end{itemize}
\end{minipage}
\end{lrbox}

%%%%%%%%%%%%%%%%%%%%%%%%%%%%%%%%%%%%%%%%%%%%%%%%%%%%%%
%%%%
%%%% Now we make two copies of the ``quizfront'' box
%%%%
%%%%%%%%%%%%%%%%%%%%%%%%%%%%%%%%%%%%%%%%%%%%%%%%%%%%%%%
\noindent \usebox{\quizfront}
\vfill
\noindent \usebox{\quizfront}

%%%% Uncomment the rest to have a two-sided quiz.
%\pagebreak
%\noindent \usebox{\quizback}
%\vfill
%\noindent \usebox{\quizback}
\end{document}