\documentclass[11pt]{article}
\usepackage{calc}
\usepackage[margin={1in,0.5in},footskip=0in]{geometry}
\usepackage[miniquiz]{hwk}
\usepackage{tikz, textcomp}

\include{mathcmds}

\renewcommand{\theclass}{math 1300}
\renewcommand{\dateinfo}{April 30, 2013}
\renewcommand{\theassignment}{Quiz 11}

\begin{document}
\pagestyle{empty}
\newsavebox{\quizfront}
\begin{lrbox}{\quizfront}
\begin{minipage}[top][4.5in][t]{\textwidth} \setlength{\parindent}{1.5em}
\drawtitle
\vspace{-0.5in}
\begin{enumerate}

\item Compute the following indefinite integrals.
  \begin{enumerate}
  \item $\displaystyle\int \left( x^3 - \frac{1}{x^2} +
    \sqrt[4]{x}\right)\;dx$
    
    \vfill  
    {\color{blue}

      \[
      = \int \left( x^3 - x^{-2} + x^{1/4}\right)\;dx = \frac{x^4}{4}
      - \frac{x^{-1}}{-1} + \frac{x^{5/4}}{5/4} + C
      \]

    }
    \vfill

  \item $\displaystyle\int x^3\left( 1 - x\right)\;dx$

    \vfill
    {\color{blue}

      \[
      =\int\left(x^3-x^4\right)\; dx = \frac{x^4}{4} - \frac{x^5}{5} + C
      \]

    }
    \vfill

  \end{enumerate}

\end{enumerate}

%\hfill \textsc{over} $\longrightarrow$

\end{minipage}
\end{lrbox}

%%%%%%%%%%%%%%%%%%%%%%%%%%%%%%%%%%%%%%%%%%%%%%%%%%%%%%
%%%% This is for the back of the quiz
%%%%%%%%%%%%%%%%%%%%%%%%%%%%%%%%%%%%%%%%%%%%%%%%%%%%%%
\newsavebox{\quizback}
\begin{lrbox}{\quizback}
\begin{minipage}[top][4.5in][t]{\textwidth} \setlength{\parindent}{1.5em}
\begin{enumerate}
\item[2.] Use the fundamental theorem of calculus to compute the following 
definite integrals.
  \begin{enumerate}
  \item $\displaystyle\int_0^{2\pi} \sin(t)\;dt$
    
    \vfill
    {\color{blue}

      We see that $\int \sin(t)\; dt = -\cos(t)+C$, so  
      \[
      \int_{0}^{2\pi} \sin(t)\;dt = -\cos(2\pi) + \cos(0) = -1 + 1 = 0.
      \]

    }
    \vfill

  \item $\displaystyle\int_1^{e^2} \frac{1}{x}\;dx$

    \vfill
    {\color{blue}

      We see that $\int \frac{1}{x}\; dx = \ln|x|+C$, so  
      \[
      \int_{1}^{e^2} \frac{1}{x}\;dx = \ln|e^2| - \ln|1| = 2 - 0 = 2.
      \]

    }
    \vfill

  \end{enumerate}
  
  


\end{enumerate}
\end{minipage}
\end{lrbox}

%%%%%%%%%%%%%%%%%%%%%%%%%%%%%%%%%%%%%%%%%%%%%%%%%%%%%%
%%%%
%%%% Now we make two copies of the ``quizfront'' box
%%%%
%%%%%%%%%%%%%%%%%%%%%%%%%%%%%%%%%%%%%%%%%%%%%%%%%%%%%%%
\noindent \usebox{\quizfront}
\vfill
\noindent \usebox{\quizback}

%%%% Uncomment the rest to have a two-sided quiz.
%\pagebreak
%\noindent \usebox{\quizback}
%\vfill
%\noindent \usebox{\quizback}
\end{document}