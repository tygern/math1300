\documentclass[11pt]{article}
\usepackage{calc}
\usepackage[margin={1in,0.5in},footskip=0in]{geometry}
\usepackage[miniquiz]{hwk}
\usepackage{tikz, textcomp}

\include{mathcmds}

\renewcommand{\theclass}{math 1300}
\renewcommand{\dateinfo}{March 12, 2013}
\renewcommand{\theassignment}{Quiz 6}

\begin{document}
\pagestyle{empty}
\newsavebox{\quizfront}
\begin{lrbox}{\quizfront}
\begin{minipage}[top][4.5in][t]{\textwidth} \setlength{\parindent}{1.5em}
\drawtitle
\vspace{-0.5in}
\begin{enumerate}

\item Find $\frac{dy}{dx}$ in terms of $x$ and $y$ if $x^5+4xy+y^2 = 13$.

  \vfill
  {\color{blue}

    We can differentiate both sides to get
    \[
    5x^4 + 4y + 4x\frac{dy}{dx} + 2y\frac{dy}{dx} = 0.
    \]
    Then
    \begin{align*}
      4x\frac{dy}{dx} + 2y\frac{dy}{dx} &= -5x^4 - 4y\\ 
      \frac{dy}{dx}(4x + 2y) &= -5x^4 - 4y\\ 
      \frac{dy}{dx} &= \frac{-5x^4 - 4y}{4x+2y}.\\ 
    \end{align*}

  }
  \vfill

\end{enumerate}

\end{minipage}
\end{lrbox}

%%%%%%%%%%%%%%%%%%%%%%%%%%%%%%%%%%%%%%%%%%%%%%%%%%%%%%
%%%% This is for the back of the quiz
%%%%%%%%%%%%%%%%%%%%%%%%%%%%%%%%%%%%%%%%%%%%%%%%%%%%%%
\newsavebox{\quizback}
\begin{lrbox}{\quizback}
\begin{minipage}[top][4.5in][t]{\textwidth} \setlength{\parindent}{1.5em}
\begin{enumerate}
\item[2.] Find the local linearization of $\displaystyle f(x) =
  \frac{1}{1+3x}$ near $x = 0$.
  \vfill
  {\color{blue}
    
    We see that $f'(x) = -3(1+3x)^{-2}$, so $f'(0) = -3(1+3\cdot0)^{-2}
    = -3$. Then the local linearization is given by
    \begin{align*}
      \ell(x) &= f(0) + f'(0)(x-0)\\
      &= 1 + -3(x-0)\\
      &= -3x + 1.
    \end{align*}

  }
  \vfill

\end{enumerate}
\end{minipage}
\end{lrbox}

%%%%%%%%%%%%%%%%%%%%%%%%%%%%%%%%%%%%%%%%%%%%%%%%%%%%%%
%%%%
%%%% Now we make two copies of the ``quizfront'' box
%%%%
%%%%%%%%%%%%%%%%%%%%%%%%%%%%%%%%%%%%%%%%%%%%%%%%%%%%%%%
\noindent \usebox{\quizfront}
\vfill
\noindent \usebox{\quizback}

%%%% Uncomment the rest to have a two-sided quiz.
%\pagebreak
%\noindent \usebox{\quizback}
%\vfill
%\noindent \usebox{\quizback}
\end{document}