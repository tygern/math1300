\documentclass[11pt]{article} 
\usepackage{calc}
\usepackage[margin={1in,1in}]{geometry} 
\usepackage[hwkhandout]{hwk}
\usepackage[pdftitle={Calc 1
  Notes},colorlinks=true,urlcolor=blue]{hyperref}

\renewcommand{\theclass}{\textsc{math}1300: calculus I}
\renewcommand{\theauthor}{Tyson Gern}
\renewcommand{\theassignment}{The Definite Integral}
\renewcommand{\dateinfo}{sections 5.1 \& 5.2}

\newcommand{\ds}{\displaystyle}

\begin{document}
\drawtitle

\section*{Introduction}
\begin{description}
\item[Example] 60 mph for 2 hours.  How far did you travel?
\item[Abstract] Draw velocity curve with labels (total 4
  sec). Estimate every 2, then 1 second.
\item[Area] More rectangles $\Rightarrow$ better estimate.  Infinite
  gives area under the curve.
\item[Rule] If velocity is positive, then the area under the curve is
  total distance traveled.
\item[Negative?] Talk about displacement.  Show picture.
\item[Sum]
  \[
  \text{Distance Traveled} = \lim_{n\to\infty}\sum_{i=0}^{n-1}f(t_i)\Delta t
  = \lim_{n\to\infty}\sum_{i=1}^{n}f(t_i)\Delta t
  \]
\end{description}

\section*{Generalize}
\begin{description}
\item[General function] $f(x)$ on $[a,b]$.
  \[
  \text{Area under curve} = \int_a^b f(x)\; dx.
  \]
\item[Negative] Count area under the $x$-axis as negative.
\item[Sum]
  \[
  \int_{a}^{b}f(t)\;dt = \lim_{n\to\infty}\sum_{i=0}^{n-1}f(t_i)\Delta t
  = \lim_{n\to\infty}\sum_{i=1}^{n}f(t_i)\Delta t
  \]
\item[Accuracy] Difference between upper and lower small $\Rightarrow$ more
  accurate.  Draw picture, stack rectangles, then
  \[
  |f(b)-f(a)|\cdot\Delta t.
  \]
\end{description}

\section*{Computing}
\begin{description}
\item[Approximation] Example: $\ds\int_{1}^{5}\dfrac{1}{x} \; dx$ with
  $n=4$.
\item[Computer]
\item[Area] Talk about when positive or negative.  Compute
  \begin{align*}
    \int_1^8 (2x-6)\; dx &&& \int_{-3}^3 \sqrt{9-x^2}\; dx
  \end{align*}
\end{description}
\end{document}
