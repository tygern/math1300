\documentclass[11pt]{article} 
\usepackage{calc}
\usepackage[margin={1in,1in}]{geometry} 
\usepackage[hwkhandout]{hwk}
\usepackage[pdftitle={Calc 1
  Notes},colorlinks=true,urlcolor=blue]{hyperref}

\renewcommand{\theclass}{\textsc{math}1300: calculus I}
\renewcommand{\theauthor}{Tyson Gern}
\renewcommand{\theassignment}{Measuring Distance}
\renewcommand{\dateinfo}{section 5.1}

\newcommand{\ds}{\displaystyle}

\begin{document}
\drawtitle

\section*{Introduction}
\begin{description}
\item[Example] 60 mph for 2 hours.  How far did you travel?
\item[Abstract] Draw velocity curve with labels (total 4
  sec). Estimate every 2 seconds, then every second.
\item[Area] More rectangles $\Rightarrow$ better estimate.  Imagine an
  infinite amount of rectangles.  This gives the area under the curve.
\item[Rule] If velocity is positive, then the area under the curve is
  total distance traveled.
\item[Negative?] Talk about displacement.  Show picture.
\end{description}

\section*{Sums}
\begin{description}
\item[Define] $n=$ number of intervals, $a=$ start, $b=$ end, $\Delta
  t=\dfrac{b-a}{n}$.  Enumerate $t_0, t_1, \dots, t_n$.  Then
  \begin{align*}
    \mbox{left} &= f(t_0)\Delta t + f(t_1)\Delta t + \cdots +
    f(t_{n-1})\Delta t\\
    \mbox{right} &= f(t_1)\Delta t + f(t_2)\Delta t + \cdots +
    f(t_{n})\Delta t
  \end{align*}
\item[Accuracy] Difference between upper and lower small $\Rightarrow$ more
  accurate.  Draw picture, stack rectangles, then
  \[
  |f(b)-f(a)|\cdot\Delta t.
  \]
\end{description}

\section*{Group Work}
\begin{description}
\item[Section 5.1] 1, 4, 14
\end{description}
\end{document}
