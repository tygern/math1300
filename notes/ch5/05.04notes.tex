\documentclass[11pt]{article} 
\usepackage{calc}
\usepackage[margin={1in,1in}]{geometry} 
\usepackage[hwkhandout]{hwk}
\usepackage[pdftitle={Calc 1
  Notes},colorlinks=true,urlcolor=blue]{hyperref}
%\usepackage{mathpazo}

\renewcommand{\theclass}{\textsc{math}1300: calculus I}
\renewcommand{\theauthor}{Tyson Gern}
\renewcommand{\theassignment}{Properties of Definite Integrals}
\renewcommand{\dateinfo}{section 5.4}

\newcommand{\ds}{\displaystyle}

\begin{document}
\drawtitle

For the following notes, let $f(x)$ and $g(x)$ be continuous
functions, and let $a$, $b$, and $c$ be constants.

\section*{Rules}
\begin{description}
\item[Warm up] What is $\int_a^a f(x)\;dx$?

  This is the area under the curve from $a$ to $a$, or the area under
  the curve at one point.  Since the area we are looking for has a
  width of zero, it must be zero (see figure 1).  Then
  \begin{center}
    \framebox{$\displaystyle \int_a^a f(x)\;dx = 0$.}
  \end{center}

\item[Limits of Integration] What happens if we switch the limits of
  integration?

  We are used to looking at $\int_a^b f(x)\; dx$ when $a\leq b$, but
  what about $\int_b^a f(x)\; dx$?  To compute this consider the
  definition of the integral:
  \[
  \int_a^b f(x)\; dx = \lim_{n\to\infty} \sum_{i=1}^n f(x_i)\Delta x =
  \lim_{n\to\infty} \sum_{i=1}^n f(x_i)\cdot\frac{b-a}{n}.
  \]
  When we switch $b$ and $a$, we see that the only part of the formula
  that changes is the ``$\Delta x$'' part:
  \[
  \int_b^a f(x)\; dx = \lim_{n\to\infty} \sum_{i=1}^n f(x_i)
  \cdot\frac{a-b}{n} = -\left(\lim_{n\to\infty} \sum_{i=1}^n f(x_i)
    \cdot\frac{a-b}{n}\right)=-\int_a^b f(x)\; dx,
  \]
  so
  \begin{center}
    \framebox{$\displaystyle\int_b^a f(x)\; dx = -\int_b^a f(x)\; dx$}
  \end{center}

\item[Adding] How can we break apart and combine integrals?

  We have the following fact:
  \begin{center}
    \framebox{$\displaystyle \int_a^c f(x)\;dx + \int_c^b f(x)\;dx = \int_a^b
    f(x)\;dx.$}
  \end{center}
  This fact is fairly easy to see if $a\leq c\leq b$ (refer to figure
  2).

  What if $a\leq b \leq c$?  The fact is still true, but it is a bit
  harder to see.  Consider figure 3.  We see that $\int_a^c f(x)\; dx$
  measures the area under $f$ from $a$ to $c$.  Then $\int_c^b f(x)\;
  dx$ measures \emph{negative} the area under $f$ between $c$ and $b$
  since $c<b$.  When we add them together, we get the area between $a$
  and $b$.

  Note that this fact is true regardless of the relationship between
  $a$, $b$, and $c$.
\item[Sum and Multiple] How do we deal with addition and scalar
  multiplication of integrals?

  We have the following rule:
  \begin{center}
    \framebox{$\displaystyle \int_a^b f(x)\; dx \pm \int_a^b g(x)\; dx =
      \int_a^b \left(f(x)\pm g(x)\right)\; dx$.}
  \end{center}
  See figure 4 for a graphical depiction of this.  We also have this rule:
  \begin{center}
    \framebox{$\displaystyle \int_a^b (c\cdot f(x))\; dx
      = c\cdot\int_a^b f(x)\; dx$.}
  \end{center}
  To see that this is true, we know that the graph of $c\cdot f(x)$ is
  obtained by stretching the graph of $f$ by a factor of $c$, so the
  area should be scaled by a factor of $c$ (see figure 5).
  
\item[Example] Compute $\int_2^7 (2x^2+3x-7)\;dx$.

  To do this, we will use the rule from Wednesday that if $f(x) =
  x^k$, then an antiderivative for $f$ is $F(x)=\frac{1}{k+1}x^{k+1}$.
  Then we can use the rules above to split up the integral:
  \[
  \int_2^7 (2x^2+3x-7)\;dx =
  \int_2^7 2x^2\; dx +\int_2^7 3x\; dx - \int_2^7 7\; dx = 
  2\int_2^7 x^2\; dx +3\int_2^7 x\; dx - 7\int_2^7 x^0\; dx.
  \]
  Using what we know about antiderivatives we get
  \[
  \int_2^7 (2x^2+3x-7)\;dx
  = 2\left(\frac{7^3}{3}-\frac{2^3}{3}\right) +
  3\left(\frac{7^2}{2}-\frac{2^2}{2}\right) -
  7\left(\frac{7^1}{1}-\frac{2^1}{1}\right)
  = \frac{1535}{6} \approx 255.8333.
  \]
  
\end{description}

\section*{Applications}
\begin{description}
\item[Area Between Functions] If $f(x)\geq g(x)$ on the interval
  $[a,b]$ then we can calculate the area in between $f(x)$ and $g(x)$
  by
  \begin{center}
    \framebox{$\displaystyle\text{area} = \int_a^b (f(x)-g(x))\; dx$.}
  \end{center}

  Let's look at an example for how to calculate the area between two
  functions. Let $f(x)=x^2$ and let $g(x)=x^3$, and suppose we are
  asked to calculate the area between $f$ and $g$.  We should first
  graph these function to get an idea for what area we are looking for
  (see figure 6). We can see from the graph that we are looking for
  the area between their two points of intersection.  To find these
  points of intersection we can set the functions equal to each other:
  \begin{align*}
    g(x) &= f(x)\\
    x^3 &= x^2\\
    x^3 - x^2 &= 0\\
    x^2(x-1) &= 0\\
    x &= 0\text{, or }1.
  \end{align*}

  Then we see that $x^2\geq x^3$ on $[0,1]$, so
  \[
  \text{area} = \int_0^1 (x^2 - x^3)\; dx = \frac{1}{12} \approx .08333.
  \]
  
\item[Symmetry] Recall the definitions of even and odd functions.

  If $f(x)$ is an even function then we have
  \begin{center}
    \framebox{$\displaystyle\int_{-a}^a f(x)\; dx = 2\int_0^a f(x)\;
      dx$.}
  \end{center}
  This works because $f(x)$ is symmetric about the $y$-axis, so the
  area under the curve from $-a$ to 0 is the same as from 0 to $a$
  (see figure 7).

  If $g(x)$ is an odd function then we have
  \begin{center}
    \framebox{$\displaystyle\int_{-a}^a g(x)\; dx = 0$.}
  \end{center}
  This works because $g(x)$ is symmetric about the origin, so the area
  under the curve from $-a$ to 0 is the same as from 0 to $a$, but is
  negative (see figure 8).

\item[Examples] Suppose that $f$ is even, and that $\int_{-2}^2
  f(x)\;dx=6$ and $\int_{-5}^5 f(x)\;dx=14$.  Find $\int_2^5
  f(x)\;dx$.

  Since $\int_{-2}^2 f(x)\;dx=6$ then we can use the above fact to
  show that $\int_{0}^2 f(x)\;dx=3$.  Similarly, we can use the fact
  that $\int_{-5}^5 f(x)\;dx=14$ to show that $\int_{0}^5 f(x)\;dx=7$.
  Then we can use what we know about splitting up integrals from above
  to find
  \begin{align*}
    \int_{2}^5 f(x)\;dx &= \int_{2}^0 f(x)\;dx + \int_{0}^5 f(x)\;dx\\
    &= -\int_{0}^2 f(x)\;dx + \int_{0}^5 f(x)\;dx\\
    &= -3 + 7 = 4.
  \end{align*}

  
\end{description}

\section*{Theorems}
\begin{description}
\item[Comparison] If $f(x)\leq g(x)$ on $[a,b]$ then
  \begin{center}
    \framebox{$\displaystyle \int_a^b f(x)\;dx \leq \int_a^b
      g(x)\;dx$.}
  \end{center}
  This makes sense due to the fact that if all the $y$-values of $g$
  are larger than the $y$-values of $f$, then the area under $g$
  should be larger than the area under $f$ (see figure 9).
\end{description}

\end{document}
