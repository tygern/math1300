\documentclass[11pt]{article} 
\usepackage{calc}
\usepackage[margin={1in,1in}]{geometry} 
\usepackage[hwkhandout]{hwk}
\usepackage[pdftitle={Calc 1
  Notes},colorlinks=true,urlcolor=blue]{hyperref}
\usepackage{graphicx, tikz}

\renewcommand{\theclass}{\textsc{math}1300: calculus I}
\renewcommand{\theauthor}{Tyson Gern}
\renewcommand{\theassignment}{Practice Midterm 3}
\renewcommand{\dateinfo}{spring 2012}

\newcommand{\ds}{\displaystyle}

\begin{document}
\drawtitle

\noindent Please answer the following questions completely, showing all work.
If you use your calculator please indicate how it was used.  Clearly
state each answer.

\begin{enumerate}
\item Find the tangent line approximation for $\sqrt{2 + x}$ near
  $x=0$.

  \vfill

  If $f(x)=\sqrt{2+x}$ then $f'(x)=\frac{1}{2}\cdot(2+x)^{-1/2}$, so
  the tangent line $\ell(x)$ is given by
  \begin{align*}
    \ell(x) &= f(0) + f'(0)(x - 0) \\
    &= \sqrt(2) + \frac{1}{2\sqrt{2}}x.
  \end{align*}

  \vfill

\item A voltage $V$ across a resistance $R$ generates a current
  $I=\frac{V}{R}$.  If a constant voltage of 6 volts is put across a
  resistance that is increasing at a rate of $0.7$ ohms per second
  when the resistance is 3 ohms, at what rate is the current
  changing?  Include units in your answer.

  \vfill

  Since the voltage is always constant we have
  \[
  I=\frac{6}{R}=6\cdot R^{-1}.
  \]
  We want to find $\frac{dI}{dt}$, so if we take the derivative with
  respect to $t$ we have
  \[
  \frac{dI}{dt}=6\cdot(-1)R^{-2}\cdot\frac{dR}{dt},
  \]
  so when $R=3$
  \[
  \frac{dI}{dt} = 6\cdot (-1)3^{-2}\cdot (.7) = -\frac{7}{15}.
  \]

  \vfill
  \newpage
  
\item A rectangle has one side on the $x$-axis and two vertices on the
  curve $y=\frac{1}{1 + x^2}$. Find the vertices of the rectangle with
  maximum area.

  \vfill

  Using the following picture
  \begin{center}
    \begin{tikzpicture}[scale=2.5]
      \def\xmin{-2}
      \def\xmax{2}
      \def\ymin{0}
      \def\ymax{1.2}
      
      \draw[color=red, very thick] (-1,0) -- (-1,.5) -- (1,.5) -- (1,0) -- (-1,0);
      
      \node at ( .5, .1) {$x$};
      \node at ( 1.1, .2) {$y$};
      
      \draw[->] (\xmin,0) -- (\xmax,0);
      \draw[->] (0,\ymin) -- (0,\ymax);
      
      \draw[color=blue, thick, domain=\xmin:\xmax] plot[id=x,samples=100]
      function{1/(1+x*x)};
    \end{tikzpicture}
  \end{center}
  we can see that the area of the rectangle is given by $A=2xy$.  Then
  we know that $y=\frac{1}{1+x^2}$, so
  \[
  A=\frac{2x}{1+x^2}.
  \]
  We know that $x$ is a length, so it must be positive.  Then we only
  need to consider $x$-values in the interval $[0,\infty)$.  To
  maximize the area we will find the critical points, and compare the
  area of the rectangle at the critical points to the area at the
  endpoints.

  Taking derivatives we see that
  \[
  \frac{dA}{dx} = \frac{2+2x^2-4x^2}{(1+x^2)^2} =
  \frac{2-2x^2}{(1+x^2)^2}.
  \]
  The denominator of the derivative is never zero, so to find the
  critical points we need to find where the numerator is zero.  This
  happens when $x=\pm 1$.  Since we are only considering positive
  values of $x$, the only critical point is at $x=1$.  Then
  \begin{align*}
    A(0) &= 0\\
    A(1) &= 1\\
    \lim_{x\to\infty} A(x) &= 0,
  \end{align*}
  so we have a global maximum area of 1 when $x=1$.  Then the vertices
  of the rectangle with maximum area are $(1,0)$, $(1,.5)$, $(-1.0)$,
  and $(-1,-5)$.

  \vfill
  
  \newpage
  
\item Two particles move in the $xy$-plane.  At time $t$, the position
  of particle $A$ is given by $x(t) = 5t - 5$ and $y(t) = 2t - k$, and
  the position of particle $B$ is given by $x(t) = 4t$ and $y(t) = t^2
  - 2t - 1$.
  \begin{enumerate}
  \item If $k=-7$ do the particles ever collide?

    \vfill

    We see that the $x$-coordinates are equal when
    \begin{align*}
      5t-5 &= 4t\\
      t-5 &= 0\\
      t &= 5.
    \end{align*}
    At $t=5$ the $y$-coordinate of particle $A$ is $y(5)=2\cdot 5 -
    (-7)= 17$ and the $y$-coordinate of particle $B$ is
    $y(5)=5^2-2\cdot 5 -1 = 14$, so the particles do not collide,
    since they are never in the same place at the same time.

    \vfill
    
  \item Find $k$ so that the two particles are certain to collide.

    \vfill

    From the discussion above we see that the $x$-coordinates of the
    two particles are the same when $t=5$.  If the particles are to
    collide, we need the $y$-coordinates to be equal at $t=5$, so
    \begin{align*}
      2\cdot 5 -k &= 5^2-2\cdot 5 -1\\
      10-k &= 14\\
      k &= -4
    \end{align*}

    \vfill
    
  \item At the time the particles collide in part (b), which is moving
    faster?

    \vfill

    The speed of particle $A$ is given by
    \[
    \text{speed} = \sqrt{(5)^2+(2)^2}=\sqrt{29}\approx 5.38516.
    \]
    The speed of particle $B$ is given by
    \[
    \text{speed} = \sqrt{(4^2)+(2t-2)^2} = \sqrt{16+64} \approx 8.94427,
    \]
    So particle $B$ is moving faster when they collide.

    \vfill
  \end{enumerate}
  \newpage

\item In parts (a) and (b) determine whether the limit exists, and
  where possible evaluate it.
  \begin{enumerate}
  \item $\displaystyle\lim_{x\to \pi}\frac{\sin^2(x)}{x-\pi}$

    \vfill

    If we try to plug in $x=\pi$ to the above equation we get
    $\frac{0}{0}$, so we can use l'Hopital's rule to solve the limit.
    Taking derivatives we obtain
    \begin{align*}
      \displaystyle\lim_{x\to \pi}\frac{\sin^2(x)}{x-\pi} &= \displaystyle\lim_{x\to \pi}\frac{2\sin(x)\cos(x)}{1}\\
      &= \frac{2\cos(\pi)\sin(\pi)}{1}\\
      &= 0.
    \end{align*}

    \vfill
    
  \item $\displaystyle\lim_{n\to \infty}\sqrt[n]{n}$

    \vfill

    We rewrite the above limit as $\displaystyle\lim_{n\to
      \infty}n^{\frac{1}{n}}$. If we try to plug in $n=\infty$ we get
    $\infty^0$, an indeterminate form.  We now need to get the limit
    in the correct form to use l'Hopital's rule.  Let
    \[
    L = \displaystyle\lim_{n\to \infty}n^{\frac{1}{n}}
    \]
    Then
    \begin{align*}
      \ln(L) &= \lim_{n\to \infty}\ln\left(n^{\frac{1}{n}}\right)\\
      &= \lim_{n\to \infty}\frac{1}{n}\ln(n)\\
      &= \lim_{n\to \infty}\frac{\ln(n)}{n}.
    \end{align*}
    If we plug in $n=\infty$ to the above limit we get
    $\frac{\infty}{\infty}$, so we can use l'Hopital's rule.  Then
    \begin{align*}
      \ln(L) &= \lim_{n\to \infty} \frac{\frac{1}{n}}{1}\\
      &= \lim_{n\to\infty}\frac{1}{n}\\
      &= 0,
    \end{align*}
    so $L=e^0=1$.
    
    \vfill
  \end{enumerate}
  \newpage


\item Let $f(x)=x-\ln(x)$ for $0.1\leq x\leq 2$. Find the value(s) of
  $x$ for which:
  \begin{enumerate}
  \item $f(x)$ has a local maximum or local minimum. Indicate which
    ones are maxima and which are minima.

    \vfill

    We see that $f'(x) = 1-\frac{1}{x}$.  Then $f'(x)$ does not exist
    when $x=0$ and is zero when $x=1$.  Since $x=0$ is not in the
    above interval, the only critical point of $f$ occurs at $x=1$.
    We see that $f'(.5)=-1$ and $f'(2)=.5$, so $f$ has a local minimum
    at $x=1$ by the first derivative test.

    \vfill
    
  \item $f(x)$ has a global maximum or global minimum.

    \vfill

    Since $f$ continuous on $[.1,2]$ we know that $f$ has both a
    global max and a global min on the interval.  We see that
    \begin{align*}
      f(.1) &\approx 2.40259\\
      f(1) &= 0\\
      f(2) &= 1.30685,
    \end{align*}
    so $f$ has a global max of 2.40259 at $x=.1$ and a global min of 0
    at $x=1$.

    \vfill
  \end{enumerate}
  \newpage
  
\item Let $f(x)$ be a differentiable function.  Use the racetrack
  principle to prove that if $f'(x)\leq 1$ for all $x$ and $f(0)=0$,
  then $f(x)\leq x$ for all $x\geq 0$.

  \vfill

  Consider the functions $f(x)$ and $g(x)=x$ on the interval $[0,x]$.
  From the information above we know that $f(0)=g(0)=0$, so the
  functions ``start'' in the same place.  Also $f'(x)\leq 1=g'(x)$, so
  we know that $g(x)$ is ``faster'' than $f(x)$.  Then by the
  racetrack principle, $g$ must have ``traveled farther'' than $f$, or
  $f(x)\leq g(x) = x$.

  \vfill

\end{enumerate}

\end{document}
