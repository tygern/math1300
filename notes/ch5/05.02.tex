\documentclass[11pt]{article} 
\usepackage{calc}
\usepackage[margin={1in,1in}]{geometry} 
\usepackage[hwkhandout]{hwk}
\usepackage[pdftitle={Calc 1
  Notes},colorlinks=true,urlcolor=blue]{hyperref}

\renewcommand{\theclass}{\textsc{math}1300: calculus I}
\renewcommand{\theauthor}{Tyson Gern}
\renewcommand{\theassignment}{Definite Integral}
\renewcommand{\dateinfo}{section 5.2}

\newcommand{\ds}{\displaystyle}

\begin{document}
\drawtitle

\section*{Introduction}
\begin{description}
\item[Goal] Generalize idea of area between the curve and the $x$-axis
  from last class.
\item[First] sigma notation
  \begin{align*}
    \sum_{i=1}^6 (i+2) &&& \sum_{i=2}^5 i^2
  \end{align*}
\item[Rectangles] $\Delta t=\dfrac{b-a}{n}$, $a=t_0,t_1,\dots, t_n=b$.
  Then
  \begin{align*}
    \sum_{i=0}^{n-1}f(t_i)\Delta t &&& \sum_{i=1}^{n}f(t_i)\Delta t
  \end{align*}
\item[Limit] as $n\to\infty$ of sums will give us area
\item[Definition]
  \[
  \int_{a}^{b}f(t)\;dt = \lim_{n\to\infty}\sum_{i=0}^{n-1}f(t_i)\Delta t
  = \lim_{n\to\infty}\sum_{i=1}^{n}f(t_i)\Delta t
  \]
\end{description}

\section*{Computing}
\begin{description}
\item[Approximation] Example: $\ds\int_{1}^{5}\dfrac{1}{x} \; dx$ with
  $n=4$.
\item[Calculator] \fbox{MATH} + \fbox{9}
  \begin{center}
    \verb#fnInt(f(X),X,a,b)#
  \end{center}
\item[Area] Talk about when positive or negative.  Compute
  \begin{align*}
    \int_1^8 (2x-6)\; dx &&& \int_{-3}^3 \sqrt{9-x^2}\; dx
  \end{align*}
\end{description}

\section*{Group Work}
\begin{description}
\item[Section 5.2] 23, 31, 32
\end{description}
\end{document}
