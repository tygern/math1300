\documentclass[11pt]{article} 
\usepackage{calc}
\usepackage[margin={1in,1in}]{geometry} 
\usepackage[hwkhandout]{hwk}
\usepackage[pdftitle={Calc 1
  Notes},colorlinks=true,urlcolor=blue]{hyperref}
%\usepackage{mathpazo}

\renewcommand{\theclass}{\textsc{math}1300: calculus I}
\renewcommand{\theauthor}{Tyson Gern}
\renewcommand{\theassignment}{Properties of Definite Integrals}
\renewcommand{\dateinfo}{section 5.4}

\newcommand{\ds}{\displaystyle}

\begin{document}
\drawtitle

\begin{enumerate}
\item Suppose that $f(x)$ is a function with the following properties:
  \begin{align*}
    \int_3^5 f(x)\;dx = 4 &&& \int_5^7 f(x)\;dx = -2
  \end{align*}
  Calculate the values of
  \begin{enumerate}
  \item $\displaystyle \int_3^5 2\cdot f(x)\;dx$
    \vfill
  \item $\displaystyle \int_7^5 f(x)\;dx$
    \vfill
  \item $\displaystyle \int_3^7 f(x)\;dx$
    \vfill
  \item $\displaystyle \int_5^7 (f(x) + 4)\;dx$
    \vfill
  \end{enumerate}

\newpage

\item Suppose that $g(x)$ is an odd function with the following properties:
  \begin{align*}
    \int_4^{11} g(x)\;dx = 6 &&& \int_0^{11} g(x)\;dx = 13
  \end{align*}
  Calculate the values of
  \begin{enumerate}
  \item $\displaystyle \int_0^4 g(x)\;dx$
    \vfill
  \item $\displaystyle \int_0^{-4} g(x)\;dx$
    \vfill
  \item $\displaystyle \int_{-4}^4 g(x)\;dx$
    \vfill

    \newpage

  \item $\displaystyle \int_{-4}^{-11} g(x)\;dx$
    \vfill
  \item $\displaystyle \int_{-4}^{11} g(x)\;dx$
    \vfill
  \end{enumerate}

  \newpage

\item Find the area between $f(x) = x+1$ and $g(x) = 2x+3$ on the
  interval $[1,6]$.

  \vfill

\item Set up an integral that finds the area between $f(x) = x^2 - 6$
  and $g(x) = x$.  You do not need to evaluate the integral.

  \vfill

  \newpage

\item Without direct computation, find the values of to following
  integrals.
  \begin{enumerate}
  \item $\displaystyle\int_{-2}^2\sin(x)\;dx$
    \vfill
  \item $\displaystyle\int_{-\pi}^\pi x^{437}\;dx$
    \vfill
  \end{enumerate}

\item Suppose that $\displaystyle\int_a^b f(x)\; dx = 7$. Find
  $\displaystyle\int_{a+3}^{b+3} f(x-3)\; dx$.
  \vfill

  \newpage

\item Without direct computation, show that
  \[
  2\leq \int_0^2 \sqrt{1+x^3}\; dx \leq 6.
  \]

\end{enumerate}

\end{document}
