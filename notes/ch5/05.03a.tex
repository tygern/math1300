\documentclass[11pt]{article} 
\usepackage{calc}
\usepackage[margin={1in,1in}]{geometry} 
\usepackage[hwkhandout]{hwk}
\usepackage[pdftitle={Calc 1
  Notes},colorlinks=true,urlcolor=blue]{hyperref}
%\usepackage{mathpazo}

\renewcommand{\theclass}{\textsc{math}1300: calculus I}
\renewcommand{\theauthor}{Tyson Gern}
\renewcommand{\theassignment}{Average Value of a Function}
\renewcommand{\dateinfo}{section 5.3}

\newcommand{\ds}{\displaystyle}

\begin{document}
\drawtitle

\section*{Introduction}
\begin{description}

\item[Recall] FTOC.  If $f(x)$ is continuous on $[a,b]$ and $F(x)$ is
  an anti-derivative for $f(x)$, then
  \[
  F(b)-F(a)=\int_a^b f(x)\;dx.
  \]
\item[Goal] Using this we can tell the average value of a function.
\item[Simple] Storage costs.  Warehouse, cost of storage on day $x$ is
  $C(x)=300-10x$.  What is the average daily storage cost?
\item[Complicated] Suppose population is given by
  $f(x)=10000(1.05)^x$, where $x$ is years since 2000.  What is the
  average population between 2000 and 2010?
\end{description}

\section*{Average Value}

\begin{description}
\item[Finite Average] If we have a finite number of points then
  \[
  \text{average} = \frac{x_1 + x_2 + \cdots + x_n}{n}.
  \]
\item[Approximate] Average value of a function $f$
  \begin{align*}
    \text{average} &\approx \frac{f(x_1) + f(x_2) + \cdots + f(x_n)}{n}\\
    &\approx \left(f(x_1) + f(x_2) + \cdots + f(x_n)\right)\cdot\frac{1}{n}
  \end{align*}

\item[Exact] Average value of a function $f$
  \[
  \text{average} = \lim_{n\to\infty} \sum_{i=1}^n f(x_i)\cdot\frac{1}{n}
  = \frac{1}{b-a}\lim_{n\to\infty}\sum_{i=1}^n f(x_i)\cdot\frac{b-a}{n}
  = \frac{1}{b-a}\int_a^b f(x)\; dx
  \]

\end{description}

%\section*{Average Rate of Change}
%\begin{description}
%\item[Goal] Let $f(x)$ be a continuous function on $[a,b]$.  We want
%  to find the average value of $f$ on $[a,b]$.
%\item[Rate] Suppose $F(x)$ is an anti-derivative of $f(x)$.  Then
%  \[
%  \text{average rate of change of }F(x) = \text{average value of }f(x).
%  \]
%\item[Know] Average rate of change of $F(x)=\dfrac{F(b)-F(a)}{b-a}$.
%  Then
%  \begin{align*}
%    \text{average value of }f(x) &= \text{average rate of change of }F(x)\\
%    &= \dfrac{F(b)-F(a)}{b-a}\\
%    \intertext{so by the FTOC}
%    &= \dfrac{\int_a^b f(x)\;dx}{b-a}\\
%    &= \dfrac{1}{b-a}\int_a^b f(x)\; dx
%  \end{align*}
%\item[Apply] to previous example.  Average value is $\approx 12,890$.
%\end{description}

\section*{Group Work}
\begin{description}
\item[Section 5.3] worksheet
\end{description}
\end{document}
