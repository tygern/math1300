\documentclass[11pt]{article} 
\usepackage{calc}
\usepackage[margin={1in,1in}]{geometry} 
\usepackage[hwkhandout]{hwk}
\usepackage[pdftitle={Calc 1
  Notes},colorlinks=true,urlcolor=blue]{hyperref}
%\usepackage{mathpazo}

\renewcommand{\theclass}{\textsc{math}1300: calculus I}
\renewcommand{\theauthor}{Tyson Gern}
\renewcommand{\theassignment}{Fundamental Theorem of Calculus}
\renewcommand{\dateinfo}{section 5.3}

\newcommand{\ds}{\displaystyle}

\begin{document}
\drawtitle

\section*{Introduction}
\begin{description}

\item[Units] of $\ds\int_a^b f(x)\;dx$.  Think
  $(f\text{-units})\cdot(x\text{-units})$, use the ``$dx$'' as a clue.
\item[Example] How do we interpret $\ds\int_a^b f(x)\;dx$, say, in a
  word problem?
  \begin{itemize}
  \item Let $f(t)=v(t)=\text{velocity}$.
  \item Then $\int_a^b f(t)\;dt$ is the distance traveled between
    $t=a$ and $t=b$.
  \item How else could we write this?  Let
    $F(t)=s(t)=\text{position}$.  Then distance traveled is
    $F(b)-F(a)$.
  \item Note that $F'(t)=f(t)$, and in this case
    \[
    \int_a^b f(t)\;dt = F(b)-F(a).
    \]
  \end{itemize}

\end{description}

\section*{In General}
\begin{description}
\item[Arbitrary Function] If $f(x)$ is any function, we can use the
  above example to say something about $\ds\int_a^b f(x)\;dx$.
\item[Fundamental Theorem] If $f(x)$ is continuous on the interval
  $[a,b]$ and $F(x)$ is a function such that $F'(x)=f(x)$ then
  \[
  \int_a^b f(x)\;dx = F(b)-F(a).
  \]
\end{description}

\section*{Example}
\begin{description}
\item[Water Tank] water leaking. Volume changing at rate of
  $f(t)=-\dfrac{t}{10}$ gal/hr.  How much water is lost after one day?
\end{description}


\section*{Group Work}
\begin{description}
\item[Section 5.3] 4-7, 27, 38
\end{description}
\end{document}
