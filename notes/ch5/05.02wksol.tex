\documentclass[11pt]{article} 
\usepackage{calc}
\usepackage[margin={1in,1in}]{geometry} 
\usepackage[hwkhandout]{hwk}
\usepackage[pdftitle={Calc 1
  Notes},colorlinks=true,urlcolor=blue]{hyperref}

\renewcommand{\theclass}{\textsc{math}1300: calculus I}
\renewcommand{\theauthor}{Tyson Gern}
\renewcommand{\theassignment}{The Definite Integral}
\renewcommand{\dateinfo}{section 5.2}

\newcommand{\ds}{\displaystyle}

\begin{document}
\drawtitle

\begin{enumerate}
\item Compute the following definite integrals using geometry.
  \begin{enumerate}
  \item $\ds\int_{-7}^{12} (-2)\;dx$
    \vfill
    {\color{blue}
      
      This integral corresponds to the area of a rectangle with width
      19 and height 2. Since the rectangle is below the $x$-axis, the
      integral is negative, so
      \[
      \int_{-7}^{12} (-2)\;dx = -38.
      \]

    }
     \vfill
  \item $\ds\int_1^5 6x \;dx$
    \vfill
    {\color{blue}
      
      This integral corresponds to the area of a rectangle with width
      4 and height 6, and triangle with width 4 and height 24. Since
      all area is above the $x$-axis, the integral is positive, so
      \[
      \int_{1}^{5} 6x\;dx = 4\cdot 6 + \frac{1}{2}\cdot 4\cdot 24 = 72.
      \]
     

    }
     \vfill
    \newpage
  \item $\ds\int_{-4}^4 \sqrt{16-x^2}\;dx$ \textit{(Hint: draw a picture)}
    \vfill
    {\color{blue}
      
      This integral corresponds to the area of a half-circle with
      radius 4. Since all area is above the $x$-axis, the integral is
      positive, so
      \[
      \int_{-4}^4 \sqrt{16-x^2}\;dx = \frac{1}{2}\pi\cdot 4^2 = 8\pi.
      \]
     

    }
     \vfill
  \end{enumerate}

\item Suppose a car moves with a velocity of $v(t) = 16 - 4t$ meters
  per second.  What is the total displacement of the car between $t =
  1$ and $t = 6$?
    \vfill
    {\color{blue}
      
      We know that the displacement of the car is given by
      \[
      \int_1^6 (16-4t)\; dt.
      \]
      Using geometry, we see that
      \[
      \int_1^6 (16-4t)\; dt = \frac{1}{2}\cdot 3\cdot 12 -
      \frac{1}{2}\cdot 2\cdot 8 = 10.
      \]
      
    }
   \vfill

  \newpage

\item Use the table below to estimate $\ds\int_0^{8} f(x)\;dx$
  \[
  \begin{array}{c|c|c|c|c|c}
    \hline
    x&0&2&4&6&8\\
    \hline
    f(x)&-4&-2&1&6&7\\
    \hline
  \end{array}
  \]
  \vfill
  {\color{blue}
    
    We can use rectangles to find that
    \begin{align*}
      \text{LH} &= -4\cdot 2 + -2\cdot 2 + 1\cdot 2 + 6\cdot 2 = 2,\\
      \text{RH} &= -2\cdot 2 + 1\cdot 2 + 6\cdot 2 + 7\cdot 2 = 24.\\
    \end{align*}
    The average of these two estimates should give us a decent
    estimate for the integral so
    \[
    \int_0^8 f(x)\; dx \approx \frac{2+24}{2} = 13.
    \]

  }
  \vfill
  \newpage

\item Use a left hand sum with $n=3$ to approximate
  $\ds\int_1^4\dfrac{1}{x}\;dx$.  Is your estimate an overestimate or
  an underestimate?

  \vfill
  {\color{blue}

    We have
    \[
    LH = \frac{1}{1}\cdot 1 + \frac{1}{2}\cdot 1 + \frac{1}{3}\cdot 1
    = \frac{11}{6}.
    \]

  }
  \vfill

\item Find, but do not compute, an integral that describes the area of
  the region under the curve $f(x)=9-x^2$ and above the $x$-axis.

  \vfill
  {\color{blue}

    If we look at a graph of $f$, we see that the function is positive
    when $-3\leq x\leq 3$. Then
    \[
    \text{area} = \int_{-3}^3 (9-x^2)\; dx.
    \]

  }
  \vfill

  \newpage

\item Without directly computing the sums, find the differences between
  the left- and right-hand Riemann sums if we use $n=200$ to
  approximate $\ds\int_1^{27}\sqrt[3]{x}\;dx$.

  \vfill
  {\color{blue}

    We know that
    \[
    \left|\text{LH} - \text{RH}\right| = \left|f(b) - f(a)\right|
    \cdot \left(\frac{b-a}{n}\right).
    \]
    when estimating an integral using $n$ rectangles on the interval
    $[a,b]$. Then 
    \[
    \left|\text{LH} - \text{RH}\right| = \left|f(27) - f(1)\right|
    \cdot \left(\frac{27-1}{200}\right) = |3 - 1|\cdot\frac{26}{200} =
    \frac{13}{50}.
    \]
    Since this number is small, this probably gives us a good
    estimate.

  }
  \vfill

  \newpage

\item Let $f(x)$ be an even function.  Using the fact that
  $\ds\int_0^3 f(x)\;dx = 4$ find the following definite integrals.
  \begin{enumerate}
  \item $\ds\int_{-3}^3 f(x)\;dx$
    \vfill
    {\color{blue}

      We know that $f$ is symmetric about the $y$-axis because it is
      even. Then
      \[
      \int_{-3}^3 f(x)\;dx = 2\int_{0}^3 f(x)\;dx = 8.
      \]
      
    }
    \vfill
  \item $\ds\int_{-3}^0 f(x)\;dx$
    \vfill
    {\color{blue}

      We know that $f$ is symmetric about the $y$-axis because it is
      even. Then
      \[
      \int_{-3}^0 f(x)\;dx = \int_{0}^3 f(x)\;dx = 4.
      \]
      
    }
     \vfill
  \end{enumerate}

\end{enumerate}

\end{document}
