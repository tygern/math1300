\documentclass[12pt]{article} 
\usepackage{calc}
\usepackage[margin={1in,1in}]{geometry} 
\usepackage[hwkhandout]{hwk}
\usepackage[pdftitle={Calc 1
  Notes},colorlinks=true,urlcolor=blue]{hyperref}
%\usepackage{mathpazo}

\renewcommand{\theclass}{\textsc{math}1300: calculus I}
\renewcommand{\theauthor}{Tyson Gern}
\renewcommand{\theassignment}{Average Value of a Function}
\renewcommand{\dateinfo}{section 5.3}

\newcommand{\ds}{\displaystyle}

\begin{document}
\drawtitle

\begin{enumerate}
\item Explain in words what the following integrals mean.  Give units.
  \begin{enumerate}
  \item $\ds\int_{2004}^{2010} f(t)\;dt$ where $f(t)$ is the rate at
    which the world's population is growing in year $t$, in people per
    year.

    \vfill

  \item $\ds\dfrac{1}{24-0}\int_0^{24} T(t)\;dt$ where $T(t)$ is the
    temperature at hour $t$ of a given day.

    \vfill
  \end{enumerate}

\item Use the fundamental theorem to find the average value of
  $f(x)=e^x$ on the interval $[2,8]$.

  \vfill

\item Use geometry to find the average value of $g(x)=-\sqrt{16-x^2}$
  on the interval $[-4,4]$.

\vfill\vfill\vfill\vfill

\end{enumerate}

\end{document}
