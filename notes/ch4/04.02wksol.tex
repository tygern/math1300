\documentclass[11pt]{article} 
\usepackage{calc}
\usepackage[margin={1in,1in}]{geometry} 
\usepackage[hwkhandout]{hwk}
\usepackage[pdftitle={Calc 1
  Notes},colorlinks=true,urlcolor=blue]{hyperref}

\renewcommand{\theclass}{\textsc{math}1300: calculus I}
\renewcommand{\theauthor}{Tyson Gern}
\renewcommand{\theassignment}{Global Extrema}
\renewcommand{\dateinfo}{section 4.2}

\newcommand{\ds}{\displaystyle}

\begin{document}
\drawtitle
\begin{enumerate}
\item Find all global extrema, if they exist, for the following
  functions on the given intervals.
  \begin{enumerate}
  \item $f(x)=3x^{1/3}-x$ on $[-1,8]$

    \vfill
    {\color{blue}
      
      We have $f'(x)=x^{-\frac{2}{3}}-1$, so $f'(x) = 0$ when $x=\pm
      1$ and $f'(x)$ is undefined when $x=0$.  Then $f(x)$ has
      critical points at $x=-1$, 0, and 1.  Also, $x=-1$ and 8 are
      included endpoints.  We have
      \begin{align*}
        f(-1) &= -2\\
        f(0) &= 0\\
        f(1) &= 2\\
        f(8) &= -2
      \end{align*}
      so $f(x)$ has a global max of 2 at $x=1$ and a global min of
      $-2$ at $x=-1$ and $x=8$.

    }
    \vfill
  
  \item $g(x)=xe^{-x}$ on $(0,\infty)$

    \vfill
    {\color{blue}
      
      We have $g'(x)=e^{-x}-xe^{-x}= e^{-x}(1-x)$, so $g'(x) = 0$ when
      $x=1$.  Then $f(x)$ has a critical point at $x=1$. We have
      \begin{align*}
        g(0) &= 0\\
        g(1) &= \frac{1}{e}\\
        \lim_{x\to\infty}g(x) &= 0
      \end{align*}
      so $g(x)$ has a global max of $\frac{1}{e}$ at $x=1$ and no
      global min.

    }
    \vfill
    
    \newpage

  \item $h(x)=\dfrac{1}{x^2}$ on $(0,\infty)$
    \vfill
    {\color{blue}
      
      We have $h'(x)=\frac{-2}{x^3}$, $h'(x)$ is undefined when $x=0$.
      However, 0 is not in our interval and we have no included
      endpoints, so $h(x)$ has no global extrema.

    }
    \vfill

  \item $P(x)=\dfrac{x^2}{x^2+1}+2$ on the whole real line
    \vfill
    {\color{blue}

      We have $P'(x) = \frac{2x}{(x^2+1)^2}$ so $P'(x) = 0$ at $x=0$.
      Then $P(x)$ has one critical point at $x=0$, and
      \begin{align*}
        \lim_{x\to -\infty}P(x) &= 3\\
        P(0) &= 2\\
        \lim_{x\to \infty}P(x) &= 3
      \end{align*}
      so $P(x)$ has a global min of 2 at $x=0$ and no global max.

    }
    \vfill

    \newpage

  \item $\displaystyle Q(x)=e^{\left(e^x\right)}$ on $[-4,7]$
    \vfill
    {\color{blue}
      
      We have $Q'(x)=e^x\cdot e^{\left(e^x\right)} > 0$, so $Q(x)$ has
      no critical points and is always increasing.  Then $Q(x)$ has a
      global min of $e^{e^{-4}}$ at $x=-4$ and a global max of
      $e^{e^7}$ at $x=7$.

    }
    \vfill

  \item $R(x)=3$ on $(1,4]$.
    \vfill
    {\color{blue}

      Since $R(x)$ is constant, it has a global max of 3 and a global
      min of 3 occurring at every value of $x$ in the interval
      $(1,4]$.
      
    }
    \vfill 
  \end{enumerate}

\newpage


\item A grapefruit is straight up with an initial velocity of 50
  ft/sec.  The grapefruit is 5 feet above the ground when it is
  released.  It's height at time $t$ is given by
  \[
  y=-16t^2+50t+5.
  \]
  \begin{enumerate}
  \item What is the grapefruits maximum height? minimum height?
    \vfill
    {\color{blue}

      We see that $v(t) = -32t+50$, so $v(t) = 0$ when
      $t=\frac{50}{32}=\frac{25}{16}$. Then the maximum height occurs
      at $h\left(\frac{25}{16}\right) = \frac{705}{16}$.  The minimum
      height of 0 occurs when the grapefruit hits the ground at
      $t\approx 3.22$
      
    }
    \vfill
    
  \item What is the grapefruits maximum velocity? minimum velocity?
    \vfill
    {\color{blue}

      The velocity $v(t) = -32t+50$ is a decreasing function, so the
      max velocity must occur when the grapefruit is thrown and the
      min must occur when the grapefruit hits the ground. Then there
      is a max velocity of 50 at $t=0$ and a min velocity of $\approx
      53.1$ at $t\approx 3.22$.

    }
    \vfill
    
  \item What is the grapefruits maximum speed? minimum speed?
    \vfill
    {\color{blue}

      We know that speed is the absolute value of velocity, so the
      maximum speed of $\approx 53.1$ occurs at $t\approx 3.22$.  The
      minimum speed of 0 occurs at the top of the grapefruit's flight
      at $t=\frac{25}{16}$.

    }    
    \vfill
    
  \end{enumerate}

  \newpage
  
\item When an electric current passes through two resistors with
  resistance $r_1$ and $r_2$, connected in parallel, the combined
  resistance, $R$, can be calculated from the equation
  \[
  \frac{1}{R}=\frac{1}{r_1}+\frac{1}{r_2},
  \]
  Where $R$, $r_1$, and $r_2$ are positive.  Assume $r_2$ is constant.
  \begin{enumerate}
  \item Show that $R$ is an increasing function of $r_1$.
    \vfill
    {\color{blue}

      We can solve for $R$ to get $R=\frac{r_1r_2}{r_1+r_2}$.  Then
      $\frac{dR}{dr_1} = \frac{r_2^2}{(r_1+r_2)^2} > 0$, thus $R$ is
      an increasing function of $r_1$.

    }
    \vfill
  \item Assume that $a$ and $b$ are positive numbers.  Where on the interval
    $[a,b]$ does $R$ take its maximum value?
    \vfill
    {\color{blue}

      Since $R$ is increasing, the max occurs at the end of the
      interval, at $r_1 = b$.

    }
    \vfill
  \end{enumerate}

  \newpage

\item Let $f(x)=ax^2+bx+c$, where $a$, $b$, and $c$ are constants.

  \begin{enumerate}
  \item Find the $x$-value of the critical point of $f(x)$.
    \vfill
    {\color{blue}

      $f'(x) = 2ax+b = 0$ when $x=-\frac{b}{2a}$, so $f(x)$ has a
      critical point at $x=-\frac{b}{2a}$.

    }
    \vfill

  \item Under what conditions on $a$, $b$, and $c$ is the critical
    value a minimum? a maximum?
    \vfill
    {\color{blue}

      $f''(x) = 2a$, so by the second derivative test the critical
      point is a local maximum when $a < 0$ and is a local minimum
      when $a > 0$.

    }
    \vfill
  \end{enumerate}


\end{enumerate}
\end{document}
