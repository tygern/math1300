\documentclass[11pt]{article} 
\usepackage{calc}
\usepackage[margin={1in,1in}]{geometry} 
\usepackage[hwkhandout]{hwk}
\usepackage[pdftitle={Calc 1
  Notes},colorlinks=true,urlcolor=blue]{hyperref}

\renewcommand{\theclass}{\textsc{math}1300: calculus I}
\renewcommand{\theauthor}{Tyson Gern}
\renewcommand{\theassignment}{Families of Functions}
\renewcommand{\dateinfo}{section 4.3}

\newcommand{\ds}{\displaystyle}

\begin{document}
\drawtitle
\begin{enumerate}
\item A bell curve is a function of the form
  \[
  f(x) = e^{-\frac{(x-a)^2}{b}}
  \]
  where $b>0$.

  \textit{The solution to this problem will help you solve problem 2
    of tonight's WeBWorK.}
  
  \begin{enumerate}
  \item Find the $x$-coordinate of the local maximum of a bell curve.
    \vfill
  \item Find the $x$-coordinates of both inflection points of a bell curve.
    \vfill
  \end{enumerate}

  \newpage


\item Let $f(x)$ be a function of the form
  \[
  f(x) = A\sin(Bx)+C
  \]
  with a local maximum at $(4, 12)$, a local minimum at $(12, 6)$, and
  no critical points between these two points.

  \begin{enumerate}
  \item Sketch a graph of $f(x)$.
    \vfill
  \item Using your graph, find the period, amplitude, and vertical shift of $f(x)$.
    \vfill
  \item Find the formula for $f(x)$.
    \vfill
  \end{enumerate}

  \newpage
  
\item Let $a > 0$ and let
  \[
  f(x) = x^3-ax^2+b.
  \]
  \begin{enumerate}
  \item Find and classify all critical points of $f(x)$.
    \vfill
  \item What effect does increasing the value of $a$ have on the
    $x$-position of the maximum(s) you found?
    \vfill
  \end{enumerate}

  \newpage

\item The \textit{logistic model} is often used to model population
  growth when the maximum population is limited by the environment.
  Using this model the population is given by
  \[
  P(t) = \frac{L}{1+Ae^{-kt}}
  \]
  where $L$, $k$, and $A$ are positive constants.

  \begin{enumerate}
  \item Find $\displaystyle \lim_{t\to\infty} P(t)$.  This is called
    that \textit{carrying capacity}.
    \vfill
  \item Find the $t$-coordinate of the inflection point of $P(t)$.
    \vfill
  \end{enumerate}
  
  \newpage

  Determining the carrying capacity of the earth is a very complex
  problem.  Although many demographers believe it is much lower, most
  agree that the carrying capacity of the earth is below 30 billion
  people.

  \begin{enumerate}
  \item[(c)] Assuming that the carrying capacity is 30 billion, that
    the world's population in 2000 was 6 billion ($P(0) = 6$), and
    that the population was growing at a rate of 120 million people
    per year at that time ($P'(0) = .12$), find a formula for the
    population of the world (in billions) since 2000.

    \vfill

    \newpage

  \item[(d)] Use your model from part (c) to predict the world
    population in 2050, 2100, and 2200.

    \vfill

  \item[(e)] Find the inflection point of the graph of the world's
    population.  What does this mean in practical terms.

    \vfill
  \end{enumerate}
  
%\item Find a function of the form
%  \[
%  y = \frac{1}{1+be^{-t}}
%  \]
%  with $y$-intercept 2 and an inflection point at $t=1$.

\end{enumerate}
\end{document}
