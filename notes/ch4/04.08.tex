\documentclass[11pt]{article} 
\usepackage{calc}
\usepackage[margin={1in,1in}]{geometry} 
\usepackage[hwkhandout]{hwk}
\usepackage[pdftitle={Calc 1
  Notes},colorlinks=true,urlcolor=blue]{hyperref}

\renewcommand{\theclass}{\textsc{math}1300: calculus I}
\renewcommand{\theauthor}{Tyson Gern}
\renewcommand{\theassignment}{Parametric Equations}
\renewcommand{\dateinfo}{section 4.8}

\newcommand{\ds}{\displaystyle}

\begin{document}
\drawtitle

\section*{Introduction}
\begin{description}
\item[Definition] Motion of a particle on the $x$-$y$ plane.
  \begin{align*}
    x(t)\mbox{ horizontal} &&& y(t)\mbox{ vertical}
  \end{align*}
\item[Example] $x(t)=\cos(t)$, $y(t)=\sin(t)$.  First look at
  multiples of $\dfrac{\pi}{2}$.  Look at interval!
\item[Difference] Two different parametric equations can trace out the
  same curve.  Consider
  \begin{align*}
    x(t)=\cos\left((x-1)^2\right)n &&&  y(t)=\sin\left((x-1)^2\right).
  \end{align*}
\item[Lines] Slope $\dfrac{b}{a}$ and point $(x_0,y_0)$
  \begin{align*}
    x(t)=x_0+at &&& y(t)=y_0+bt
  \end{align*}
\item[Parameterizing Function] Example. $y=x^3-4x+8x-13$.  Set
  $x(t)=t$ and parameterize.
\end{description}

\section*{Calculus}

\begin{description}
\item[Speed] Draw picture with vectors (don't call them that)
  \[
  \mbox{speed} = \sqrt{\left(\frac{dx}{dt}\right)^2 +
    \left(\frac{dy}{dt}\right)^2}
  \]
\item[Tangent line] Line with same point, same $\dfrac{dx}{dt}$ and
  $\dfrac{dy}{dt}$.
\item[Slope] Use chain rule:
  \begin{align*}
    \frac{dy}{dt} &=\frac{dy}{dx}\frac{dx}{dt}\\
    \intertext{so}
    \frac{dy}{dx} &=\frac{\frac{dy}{dt}}{\frac{dx}{dt}}
  \end{align*}
\item[Second Derivative]
  \[ \frac{d^2y}{dx^2} = \frac{d}{dt} \left(\frac{dy}{dx}\right)
  \left/\frac{dx}{dt}\right..
  \]
\end{description}

\section*{Group Work}
\begin{description}
\item[Section 4.8] 7, 9, 17, 20, 14, 24, 44
\end{description}
\end{document}
