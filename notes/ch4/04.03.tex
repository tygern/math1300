\documentclass[11pt]{article} 
\usepackage{calc}
\usepackage[margin={1in,1in}]{geometry} 
\usepackage[hwkhandout]{hwk}
\usepackage[pdftitle={Calc 1
  Notes},colorlinks=true,urlcolor=blue]{hyperref}

\renewcommand{\theclass}{\textsc{math}1300: calculus I}
\renewcommand{\theauthor}{Tyson Gern}
\renewcommand{\theassignment}{Families of Functions}
\renewcommand{\dateinfo}{section 4.3}

\newcommand{\ds}{\displaystyle}

\begin{document}
\drawtitle

\section*{Motion Under Gravity}
You already know many families of functions.  Here are some more that
are useful for modeling.
\begin{description}
\item[Example] $y= -4.9t^2+v_0t+y_0$ (in m/s)
\end{description}

\section*{Bell Curve}
\begin{description}
\item[Basic] Talk about probability distribution, histogram
  \[
  y=e^{-(x-a)^2/b}
  \]
\item[Sage] Change variables
\end{description}

\section*{Exponential With Limit}
\begin{description}
\item[Examples] Temperature, terminal velocity
\item[Sage] Change variables
  \[
  y=a\left(1-e^{-bx}\right)
  \]
\end{description}

\section*{Logistic Model}
\begin{description}
\item[Examples] Population on an island
\item[Sage] Change variables
  \[
  y=\dfrac{L}{1+Ae^{-kt}}
  \]
\end{description}

\section*{Group Work}
\begin{description}
\item[Other] Find gravitational constant, find inflection points for
  bell curve and logistic model.  What do these mean?
\end{description}
\end{document}
