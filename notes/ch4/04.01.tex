\documentclass[11pt]{article} 
\usepackage{calc}
\usepackage[margin={1in,1in}]{geometry} 
\usepackage[hwkhandout]{hwk}
\usepackage[pdftitle={Calc 1
  Notes},colorlinks=true,urlcolor=blue]{hyperref}

\renewcommand{\theclass}{\textsc{math}1300: calculus I}
\renewcommand{\theauthor}{Tyson Gern}
\renewcommand{\theassignment}{Local Minima/Maxima}
\renewcommand{\dateinfo}{section 4.1}

\newcommand{\ds}{\displaystyle}

\begin{document}
\drawtitle

\section*{Local Extrama}
\begin{description}
\item[Picture] $f(x)=x^3+6x^2-36x+8$, $(-6,224)$, $(2, -32)$
\item[Definition] maximum of $f(a)$ at $x=a$\dots ``values near $a$'' 

\end{description}

\section*{Finding Extrema}
\begin{description}
\item[Picture] With horizontal tangent lines and corners
\item[Definition] critical point at $x=a$ if $f(a)$ is defined and
  $f'(a)=0$ or $f'(a)$ DNE.
\item[Caution] Not all CP are extrama. $f(x)= x^3$, $g(x)=
  \sqrt[3]{x}$, $h(x)= \dfrac{1}{x^2}$ ($f$ must be defined at extrema)
\end{description}

\section*{First Derivative Test}
\begin{description}
%\item[Picture] What distinguishes extrema from other CP?
\item[Test] First derivative test
\item[Example] above example, $f(x)=|x|$, $g(x)=x^3$,
  $h(x)=\dfrac{1}{x(x-2)}$.
\end{description}

\section*{Second Derivative Test}

\begin{description}
\item[Concavity] make connection with picture
\item[Test] Second derivative test. $f'(a)=0$, $>$, $<$, $=$ 0
\item[Example] $f(x)=x^3-3x^2-72x-3$, $g(x) = x^3$, $h(x) = x^4$.
\end{description}

\section*{Inflection Points}

\begin{description}
\item[Definition] Change in concavity
\item[Finding] $f''(a)=0$ or DNE, changes from positive to negative
\item[Connection] Local min/max of $f'$, above example
%\item[Example] See above.
\end{description}

\section*{Group Work}
\begin{description}
\item[Section 4.1] 7, 17, 19, 27
\end{description}
\end{document}
