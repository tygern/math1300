\documentclass[11pt]{article} 
\usepackage{calc}
\usepackage[margin={1in,1in}]{geometry} 
\usepackage[hwkhandout]{hwk}
\usepackage[pdftitle={Calc 1
  Notes},colorlinks=true,urlcolor=blue]{hyperref}

\renewcommand{\theclass}{\textsc{math}1300: calculus I}
\renewcommand{\theauthor}{Tyson Gern}
\renewcommand{\theassignment}{Marginal Analysis}
\renewcommand{\dateinfo}{section 4.5}

\newcommand{\ds}{\displaystyle}

\begin{document}
\drawtitle

\section*{Functions}
\begin{description}
\item[Cost] Draw function on the board, talk about fixed cost,
  concavity (usually concave up).
\item[Revenue] Draw function.  Talk about concavity (usually concave
  down) and $y$-intercept (at 0). $R=\text{price}\cdot\text{quantity}
  = p\cdot q$
\item[Profit] $\pi(q) = R(q) - C(q)$
\end{description}

\section*{Marginal Analysis}
\begin{description}
\item[Define] $MC(q)=C'(q)$, same for revenue.
  \[
  MC(q) = C'(q) = \lim_{h\to 0} \frac{C(q+h)-C(q)}{h} \approx C(q+1) - C(q)
  \]
\item[Cases] $MR>MC$, $MR<MC$, $MR=MC$
\end{description}

\section*{Maximize Profit}
\begin{description}
\item[Draw] Revenue and cost functions.  Where is profit maximized?
\item[Write] $\pi(q)=R-C$, so $\pi'(q)=MR-MC$.  Candidates for
  maximization when $MR=MC$.
\item[Example] Selling TVs.
  \begin{itemize}
  \item Starting price is $\$1000$.
  \item For every TV you sell the price for all goes down by $\$0.50$.
  \item Fixed cost of $\$100,000$.
  \item Cost is $\$10$ for each additional TV manufactured.
  \end{itemize}

  $\text{price} = p=1000-\dfrac{q}{2}$

  $R(q)=1000q-\dfrac{q^2}{2}$
  
  $C(q)=100000+200q$
  
  Maximum profit of $\$220000$ at $q=800$.
\end{description}

\section*{Group Work}
\begin{description}
\item[Worksheet]
\end{description}
\end{document}
