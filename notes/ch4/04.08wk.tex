\documentclass[11pt]{article} 
\usepackage{calc, tikz}
\usepackage[margin={1in,1in}]{geometry} 
\usepackage[hwkhandout]{hwk}
\usepackage[pdftitle={Calc 1
  Notes},colorlinks=true,urlcolor=blue]{hyperref}

\renewcommand{\theclass}{\textsc{math}1300: calculus I}
\renewcommand{\theauthor}{Tyson Gern}
\renewcommand{\theassignment}{Parametric Equations}
\renewcommand{\dateinfo}{section 4.8}

\newcommand{\ds}{\displaystyle}

\begin{document}
\drawtitle

\begin{enumerate}

\item Suppose that a particle in the $xy$-plane moves according to the
  parametric equation:
  \begin{align*}
    x(t) &= \cos(t)\\
    y(t) &= \sin(t),
  \end{align*}
  where $0\leq t\leq 2\pi$.
  \begin{enumerate}
  \item Give the position of the particle at the following times.
    \begin{enumerate}
    \item $t = 0$
      \vfill
    \item $t = \frac{\pi}{4}$
      \vfill
    \item $t = \pi$
      \vfill
    \item $t = \frac{3\pi}{2}$
      \vfill
    \end{enumerate}

  \item Sketch a graph of the path of the particle and indicate the
    direction of motion.
    \vfill\vfill\vfill\vfill
    
    \newpage

  \item Suppose that we instead restrict the time interval to
    $-\frac{\pi}{4}\leq t \leq \frac{5\pi}{4}$. Sketch a graph of the
    path of the particle and indicate the direction of motion.
    \vfill

  \item Consider the parametric equation:
    \begin{align*}
      x(t) &= \cos(2t)\\
      y(t) &= \sin(2t).
    \end{align*}
    where $0\leq t\leq 2\pi$. Graph this parametric equation and
    describe its motion. Compare this to your answer in part (b).

    \vfill\vfill


  \end{enumerate}


  \newpage

\item Write the parameterizations of the following lines.
  \begin{enumerate}
  \item The line through the points $(2,-1)$ and $(1,3)$
    \vfill
  \item A horizontal line through the point $(2, 8)$
    \vfill
  \item A vertical line through the point $(-4, 7)$
    \vfill
  \item The line $y=7x-14$
    \vfill
  \end{enumerate}
  \newpage

\item Describe the motion of a particle moving according to the given
  parametric equations, and find the equation of the curve along which
  the particle moves.
  
  \begin{enumerate}
  \item 
    \begin{align*}
      x(t) &= t - 9\\
      y(t) &= 9 - 3t.
    \end{align*}
    \vfill
  \item
    \begin{align*}
      x(t) &= 2\cos(-t)\\
      y(t) &= 3\sin(-t).
    \end{align*}
    \vfill
  \end{enumerate}
  \newpage
  
\item Two particles move in the $xy$-plane.  At time $t$, the position
  of a particle $A$ is given by
  \begin{align*}
    x(t) &= 4t-4 & y(t) &= 2t-k,\\
    \intertext{and the position of particle $B$ is given by}
    x(t) &= 3t & y(t) &= t^2-2t-1.
  \end{align*}

  \begin{enumerate}
  \item If $k=5$ do the particles ever collide?  Explain.
    \vfill
  \item Find $k$ so that the two particles do collide.
    \vfill
  \end{enumerate}

\end{enumerate}

\end{document}
