\documentclass[11pt]{article} 
\usepackage{calc}
\usepackage[margin={1in,1in}]{geometry} 
\usepackage[hwkhandout]{hwk}
\usepackage[pdftitle={Calc 1
  Notes},colorlinks=true,urlcolor=blue]{hyperref}

\renewcommand{\theclass}{\textsc{math}1300: calculus I}
\renewcommand{\theauthor}{Tyson Gern}
\renewcommand{\theassignment}{Local Minima/Maxima}
\renewcommand{\dateinfo}{section 4.1}

\newcommand{\ds}{\displaystyle}

\begin{document}
\drawtitle

\section*{Inflection Points}

\begin{description}
\item[Definition] A function $f(x)$ has an inflection point at $x=a$
  if there is a change in concavity in the graph of $f(x)$ at $x=a$.
\item[Finding] Find whree $f''(a)=0$ or DNE (critical points of
  $f'(x)$!). Check that $f''(x)$ changes from positive to negative at
  these points.
\item[Example 1] $f(x)=x^3-3x^2-72x-3$.  Find all inflection
  points. (There is an inflection point at $(1, -77)$.)
\item[Caution] $f''(x)=0\not \Rightarrow x$ is an inflection point!
\item[Example 2] $f(x)=x^4$.
\item[Connection] An inflection point of $f(x)$ is a local min/max of
  $f'(x)$, above example
\item[Example 3] See \textbf{Example 1} above.
\end{description}

\section*{Group Work}
\begin{description}
\item[Section 4.1] See worksheet.
\end{description}
\end{document}
