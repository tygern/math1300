\documentclass[11pt]{article} 
\usepackage{calc}
\usepackage[margin={1in,1in}]{geometry} 
\usepackage[hwkhandout]{hwk}
\usepackage[pdftitle={Calc 1
  Notes},colorlinks=true,urlcolor=blue]{hyperref}
%\usepackage{mathpazo}
\usepackage{multicol}

\renewcommand{\theclass}{\textsc{math}1300: calculus I}
\renewcommand{\theauthor}{Tyson Gern}
\renewcommand{\theassignment}{Second FTOC}
\renewcommand{\dateinfo}{section 6.4}

\newcommand{\ds}{\displaystyle}

\begin{document}
\drawtitle

\section*{Setup}
\begin{description}
\item[Recall] First FTOC
  \[
  F(b)-F(a)=\int_a^b f(x)\;dx
  \]
\item[Antiderivative] From before, if $f(x)=\sin(x^2)$ then hard to
  find antiderivative.  How to find one?
  \[
  F(b)-F(a)=\int_a^b \sin(t^2)\;dt.
  \]
  Set $b=x$ and $a=0$.  Then
  \[
  F(x)-F(0)=\int_0^x \sin(t^2)\;dt,
  \]
  where $F'(x)=\sin(x^2)$.  Say want $F(0)=0$.  Then
  \[
  F(x)=\int_0^x \sin(t^2)\;dt
  \]
  is an antiderivative.

\item[Example] Find antiderivative, $F$, of
  $f(x)=\sqrt{\ln(x)\sin^4(x)}$ such that $F(7)=0$

\end{description}

\section*{Second FTOC}
\begin{description}
\item[Theorem] Let $f(x)$ be continuous on an interval.  Then if $a$
  is any number on that interval, the function
  \[
  F(x)=\int_a^x f(t)\;dt,
  \]
  is an antiderivative for $f(x)$.  Equivalently,
  \[
  \frac{d}{dx}\left(\int_a^x f(t)\;dt\right)=f(x)
  \]
\item[Examples] A few without chain rule.  Some with chain rule.
  Double.


\end{description}

\section*{Group Work}
\begin{description}
\item[Section 6.4] 8, 11, 16, 29, 36
\end{description}
\end{document}
