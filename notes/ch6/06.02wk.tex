\documentclass[11pt]{article} 
\usepackage{calc}
\usepackage[margin={1in,1in}]{geometry} 
\usepackage[hwkhandout]{hwk}
\usepackage{tikz}
\usepackage{graphicx}
\usepackage[pdftitle={Calc 1
  Notes},colorlinks=true,urlcolor=blue]{hyperref}

\renewcommand{\theclass}{\textsc{math}1300: calculus I}
\renewcommand{\theauthor}{Tyson Gern}
\renewcommand{\theassignment}{Constructing Antiderivatives}
\renewcommand{\dateinfo}{section 6.2}

\newcommand{\ds}{\displaystyle}

\begin{document}
\drawtitle

\begin{enumerate}
  
\item Compute the following indefinite integrals.
  \begin{enumerate}
  \item $\displaystyle\int \left( x^3 - \frac{1}{x^2} +
      \sqrt[4]{x}\right)\;dx$
    \vfill
  \item $\displaystyle\int \frac{t^4-7}{t^2}\;dt$
    \vfill
  \item $\displaystyle\int x^3\left( 1 - x\right)\;dx$
    \vfill

    \newpage

  \item $\displaystyle\int \cos(3x)\;dx$
    \vfill

  \item $\displaystyle\int 2^t\;dt$
    \vfill
  \end{enumerate}
  
  \newpage

\item Use the fundamental theorem of calculus to compute the following
  indefinite integrals.
  \begin{enumerate}
  \item $\displaystyle\int_{-2}^3 x^2\;dx$
    \vfill
  \item $\displaystyle\int_1^{e^2} \frac{1}{x}\;dx$
    \vfill
  \item $\displaystyle\int_0^{2\pi} \sin(t)\;dt$
    \vfill

    \newpage
    
  \item $\displaystyle\int_0^3 (t-1)^4\;dt$
    \vfill

  \item $\displaystyle\int_0^1 e^{2x}\;dx$
    \vfill
  \end{enumerate}
  
  \newpage
  
\item Suppose that a large pail of water is being filled at a rate of
  $f(t) = 2+\sqrt{t}$ gallons per minute.  How much water goes into
  the pail between $t = 1$ and $t = 9$?
  
  \vfill

\item Suppose that the population of rabbits in a forest changes at a
  rate of $r(t) = \frac{\pi}{6}\sin\left(\frac{\pi}{6}t\right) +
  \frac{1}{3}$ thousand rabbits per month.  If $t$ represents months
  since January 1, 2012, how much will the population grow between
  January 1, 2012 and July 1, 2013?

  \vfill

  \newpage
  
\item Use the fundamental theorem of calculus to find the area of the
  region above $f(x) = x^2 - 15$ but below $g(x) = 2x$.
  
  \vfill

\item Find all values of $c$ such that $\displaystyle\int_{1}^c 2cx\; dx = 0$.

  \vfill

\end{enumerate}

\end{document}
