\documentclass[11pt]{article} 
\usepackage{calc}
\usepackage[margin={1in,1in}]{geometry} 
\usepackage[hwkhandout]{hwk}
\usepackage[pdftitle={Calc 1
  Notes},colorlinks=true,urlcolor=blue]{hyperref}
\usepackage{tikz}

\renewcommand{\theclass}{\textsc{math}1300: calculus I}
\renewcommand{\theauthor}{Tyson Gern}
\renewcommand{\theassignment}{Second FTOC}
\renewcommand{\dateinfo}{section 6.4}

\newcommand{\ds}{\displaystyle}

\begin{document}
\drawtitle

\begin{enumerate}
  
\item Write an expression for a function $f(x)$ with the given
  properties.
  \begin{enumerate}
  \item $f'(x) = e^{x^2}$ \vfill
  \item $f'(x) = \sqrt{\cos(x)}$, $f(9) = 0$ \vfill
  \item $f'(x) = \dfrac{1}{\ln(x)}$, $f(-4) = -3$ \vfill
  \item $f'(x) = \sin(2x)$, $f(\pi) = 1$ \vfill
  \end{enumerate}
  
  \newpage
  
\item Let $F(x) = \int_2^x f(t)\;dt$.  Use the graph of $f(x)$ to
  answer the following questions. 
  \begin{center}
    \begin{tikzpicture}[xscale = 1.5, yscale = 1.5]
      \def\xmin{0}
      \def\xmax{8.5}
      \def\ymin{-2.5}
      \def\ymax{2.5}
      \draw[->] (\xmin,0) -- (\xmax,0);
      \draw[<->] (0,\ymin) -- (0,\ymax);
      \foreach \i in {1,...,8} {
        \draw (\i, .1) -- (\i,-.1) node[below] {\i};
      }
      \foreach \i in {-2,...,2} {
        \draw (.1, \i) -- (-.1,\i) node[left] {\i};
      }
      \draw[thick, domain=0:2] plot[samples=100] function{x};
      \draw[thick, domain=2:6] plot[samples=100] function{-x+4};
      \draw[thick, domain=6:8] plot[samples=100] function{2*x-14}
      node[right] {$f(x)$};
    \end{tikzpicture}
  \end{center}
  
  \begin{enumerate}
  \item On the axes below, carefully graph $F(x)$, $F'(x)$, and $F''(x)$.
    \vfill
    \begin{center}
      \begin{tikzpicture}[xscale = 1.5, yscale = 1.5]
        \def\xmin{0}
        \def\xmax{8.5}
        \def\ymin{-3.5}
        \def\ymax{3.5}
        \draw[->] (\xmin,0) -- (\xmax,0);
        \draw[<->] (0,\ymin) -- (0,\ymax);
        \foreach \i in {1,...,8} {
          \draw (\i, .1) -- (\i,-.1) node[below] {\i};
        }
        \foreach \i in {-3,...,3} {
          \draw (.1, \i) -- (-.1,\i) node[left] {\i};
        }
      \end{tikzpicture}
    \end{center}
    \vfill
    
    \newpage
    
  \item Find the coordinates and classify all local extrema of $F(x)$.
    \vfill
  \item Find the coordinates of all inflection points of $F(x)$.
    \vfill
  \item Find all global extrema of $F(x)$ on $[0,8]$.
    \vfill
  \end{enumerate}
  
  \newpage
  
\item Compute the following derivatives.
  \begin{enumerate}
  \item $\ds\frac{d}{dx}\left(\int_3^x \frac{\ln(t)}{\sin(t)}\;dt\right)$ \vfill
  \item $\ds\frac{d}{dx}\left(\int_x^6 \frac{\sqrt{t}}{\cos(t)}\;dt\right)$ \vfill
  \item $\ds\frac{d}{dx}\left(\int_7^{\sin(x)} e^{t^8}\;dt\right)$ \vfill
    \newpage
  \item $\ds\frac{d}{dx}\left(\int_{x^{3}}^{\sin(x)} \tan(4t^2)\;dt\right)$ \vfill
  \item $\ds\frac{d}{dx}\left(\left(x^2-x\right) \cdot
      \left(\int_{8}^{x^2} t\cos(t)\;dt\right)\right)$ \vfill
  \end{enumerate}
  
  \newpage
  
\item Suppose that you are making a cylindrical can that will hold 21
  cubic inches of liquid.  If you want to use the minimal amount of
  material to make the can, what should the dimensions of the can be?
  \vfill 
  
\item Find the derivative of $f(x)=x^2-3$ using the limit definition
  of the derivative.
  \vfill
  
  \newpage
  
\item A cone with a height of 16 meters and a radius of 8 meters at
  the top is leaking water at a rate of $4$ meters$^3$/hour.  When
  the water in the cone is 8 meters deep, how fast is the height of
  the water dropping in meters/hour? (The volume of a cone is given by
  $V=\frac{1}{3}\pi r^2 h$)
  
  \newpage
  
\item Let $f(x)$ and $g(x)$ be functions with values defined below.
  \[
  \begin{array}{c||c|c|c|c|c}
    x & -2 & -1 & 0 & 1 & 2\\
    \hline\hline
    f(x) & -2 & -1 & 1 & 2 & 4\\ \hline
    f'(x) & 1 & 2 & 2 & 1 & 2\\ \hline
    g(x) & -1 & 0 & 1 & -1 & -2\\ \hline
    g'(x) & 2 & 1 & 0 & -2 & -1\\ 
  \end{array}
  \]
  Calculate the following.
  \begin{enumerate}
  \item Let $h(x)=f(x)g(x)$.  Find $h'(1)$.\vfill
  \item Let $h(x)=\dfrac{f(x)}{g(x)}$.  Find $h'(-2)$.\vfill
  \item Let $h(x)=f(g(x))$.  Find $h'(-2)$.\vfill
  \item Let $h(x)=f^{-1}(x)$.  Find $h'(1)$.\vfill
  \end{enumerate}
  
  
\end{enumerate}
\end{document}