\documentclass[11pt]{article} 
\usepackage{calc}
\usepackage[margin={1in,1in}]{geometry} 
\usepackage[hwkhandout]{hwk}
\usepackage[pdftitle={Calc 1
  Notes},colorlinks=true,urlcolor=blue]{hyperref}
%\usepackage{mathpazo}
\usepackage{multicol}

\renewcommand{\theclass}{\textsc{math}1300: calculus I}
\renewcommand{\theauthor}{Tyson Gern}
\renewcommand{\theassignment}{Equations of Motion}
\renewcommand{\dateinfo}{section 6.5}

\newcommand{\ds}{\displaystyle}

\begin{document}
\drawtitle

\begin{enumerate}
\item A ball that is dropped from a window hits the ground in five
  seconds. How high is the window?

  \vfill
  
\item On the moon the acceleration due to gravity is 5 ft/sec$^2$.  An
  astronaut jumps into the air with an initial upward velocity of 10
  ft/sec.  How high does he go? How long is the astronaut off the
  ground?

  \vfill\newpage

\item While attempting to understand the motion of bodies under
  gravity, Galileo stated that:

  \begin{quote}
    The time in which any space is traversed by a body starting at
    rest and uniformly accelerated is equal to the time in which the
    same space would be traversed by the same body moving at a uniform
    speed whose value is the mean of the highest velocity and the
    velocity just before acceleration began.
  \end{quote}

  \begin{enumerate}
  \item Write Galileo's statement in symbols, defining all the symbols
    you use.

    \vfill
    
  \item Check Galileo's statement for a body dropped off a 100-foot
    building accelerating from rest under gravity until it hits the
    ground.

    \vfill
    
  \item Show why Galileo's statement is true in general.

    \vfill
    
  \end{enumerate}

  \newpage

\item Newton's law of gravity says that the gravitational force
  between two bodies is attractive and given by
  \[
  F=\frac{GMm}{r^2},
  \]
  where $G$ is the gravitational constant, $m$ and $M$ are the masses
  of the two bodies, and $r$ is the distance between them.  According
  to Newton's second law,
  \[
  \text{Force}=\text{Mass}\times\text{Acceleration}.
  \]
  Let $s$ be the position of an object of mass $m$ that is being acted
  upon by gravity.  Then
  \begin{align*}
    m\cdot a &= -\frac{GMm}{r^2} \\
    m\frac{d^2s}{dt^2} &= -\frac{GMm}{r^2} \\
    \frac{d^2s}{dt^2} &= -\frac{GM}{r^2},
  \end{align*}
  where $r$ is the distance between the object and the center of the
  earth.  Since in everyday life $r$ does not change significantly
  over the course of motion we usually assume $r$ is constant.
  However let's see what happens as $r$ varies greatly.

  The acceleration due to gravity 2 meters from the ground is 9.8
  m/sec$^2$.  What is acceleration due to gravity 100 meters from the
  ground? At 100,000 meters? (The radius of the earth is $6.4\cdot
  10^6$ meters)

  \newpage

\item On a different sheet of paper, compute the following:
  \begin{enumerate}
  \item $\ds\frac{d}{dx}\int_3^x \frac{\ln(t)}{\sin(t)}\;dt$ \vfill
  \item $\ds\frac{d}{dx}\int_x^6 \frac{\sqrt{t}}{\cos(t)}\;dt$ \vfill
  \item $\ds\frac{d}{dx}\int_7^{\sin(x)} e^{t^8}\;dt$ \vfill
  \item $\ds\frac{d}{dx}\int_{t^{-2}}^{\sin(x)} \tan(4t^2)\;dt$ \vfill
  \item $\ds\int e^{x+1}\; dx$ \vfill
  \item $\ds\int \sqrt{3x+4}\; dx$ \vfill
  \item $\ds\int \frac{1}{2x-81}\; dx$ \vfill
  \end{enumerate}

\end{enumerate}
\end{document}
