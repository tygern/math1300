\documentclass[11pt]{article} 
\usepackage{calc}
\usepackage[margin={1in,1in}]{geometry} 
\usepackage[hwkhandout]{hwk}
\usepackage[pdftitle={Calc 1
  Notes},colorlinks=true,urlcolor=blue]{hyperref}
\usepackage{graphicx}
\usepackage{multicol}
\usepackage{enumerate}
\renewcommand{\theclass}{\textsc{math}1300: calculus I}
\renewcommand{\theauthor}{Tyson Gern}
\renewcommand{\theassignment}{Review Worksheet}
\renewcommand{\dateinfo}{}

\newcommand{\ds}{\displaystyle}

\begin{document}
\drawtitle

\noindent Please answer the following questions completely, showing all work. Clearly
state each answer.

\begin{enumerate}
\item Compute the following derivatives and indefinite integrals.  You
  may \textbf{not} use your calculator on this portion of the exam.
  \begin{enumerate}
  \item $\ds\frac{d}{dx}\left(\sqrt{\frac{\sin(2x)}{\cos(2x)}}\right)$
    \vfill
  \item $\ds\int\left(3e^x+\frac{4}{x}\right)\;dx$
    \vfill
    \newpage
  \item $\ds\frac{d}{dx}\left(\tan^2\left(\ln\left(
          x\right)\right)\right)$ \vfill
  \item $\ds\int\left(x+7\right)^3\;dx$
    \vfill
    \newpage
  \item $\ds\frac{d}{dx}\left(\sqrt{2^x}\cdot \frac{1}{x^2}\right)$
    \vfill
  \item $\ds\int\left(\frac{x+2}{x}\right)\;dx$
    \vfill
  \end{enumerate}

  \newpage
  
\item Let $f(x)$ be a continuous function such that $f(0)=0$, and let
  the graph below represent $\mathbf{f'(x)}$.  Draw the graphs of
  $f(x)$ and $f''(x)$ on the axes given below.
  \begin{itemize}
  \item $f'(x)$
    \begin{center}
      \includegraphics[width=3.5in]{fprime.pdf}
    \end{center}
    \vfill
  \item[(a)] $f(x)$
    \begin{center}
      \includegraphics[width=3.5in]{axes.pdf}
    \end{center}
    \vfill
  \item[(b)] $f''(x)$
    \begin{center}
      \includegraphics[width=3.5in]{axes.pdf}
    \end{center}
    \vfill
  \end{itemize}

  \newpage

\item Suppose that you are making a cylindrical can that will hold 21
  cubic inches of liquid.  If you want to use the minimal amount of
  material to make the can, what should the dimensions of the can be?
  \vfill \newpage

\item Suppose that a car is traveling in a straight line with velocity
  given by $v(t)=t^2$ meters per second after $t$ seconds.
  \begin{enumerate}
  \item How far does the car travel in the first 10 seconds?
    \vfill
  \item When is the car's acceleration equal to $6$ m/s$^2$?
    \vfill
  \end{enumerate}

  \newpage

\item Find the derivative of $f(x)=x^2-3$ using the limit definition
  of the derivative.
  \vfill

\item Use the linear approximation of $f(x)=\sqrt{x}$ near $x=9$ to
  estimate $\sqrt{9.2}$.
  \vfill

  \newpage

\item Suppose that the marginal cost of producing $q$ VCRs is given by
  \[
  MC = 6q.
  \]
  What is the difference between the cost of producing 100 VCRs and
  the cost of producing 150 VCRs?
  \vfill

\item On the axes below, draw a graph of a function with the following
  properties:
  \begin{multicols}{2}
    \begin{enumerate}[(i)]
    \item $f(x)$ is continuous everywhere except $x=1$
    \item $f(x)$ is differentiable everywhere except $x=1$ and $x=-1$.
    \item $\ds\lim_{x\to\infty} f(x) = 2$.
    \item $f(x)$ is increasing and concave down on $[-1,1]$.
    \item $f(x)$ has a critical point at $x=-2$.
    \item $\ds\lim_{x\to 1} f(x)$ exists.
    \end{enumerate}
  \end{multicols}
  \begin{center}
    \includegraphics{axes.pdf}
  \end{center}
\end{enumerate}

\end{document}
