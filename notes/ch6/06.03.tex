\documentclass[11pt]{article} 
\usepackage{calc}
\usepackage[margin={1in,1in}]{geometry} 
\usepackage[hwkhandout]{hwk}
\usepackage[pdftitle={Calc 1
  Notes},colorlinks=true,urlcolor=blue]{hyperref}
%\usepackage{mathpazo}
\usepackage{multicol}

\renewcommand{\theclass}{\textsc{math}1300: calculus I}
\renewcommand{\theauthor}{Tyson Gern}
\renewcommand{\theassignment}{Differential Equations}
\renewcommand{\dateinfo}{section 6.3}

\newcommand{\ds}{\displaystyle}

\begin{document}
\drawtitle

\section*{Setup}
\begin{description}
\item[Velocity] Constant velocity, what is position?
\item[Example] $v=\frac{ds}{dt}=50$, what is $s$?
  \textbf{Antiderivative}.
  \[
  s=50t+C
  \]
  What is the meaning of $C$?
\item[Acceleration] $g=-32$ ft/sec$^2$.  Ball thrown up from 50 ft
  building at a velocity of $10$ ft/s.  What is position?  Walk through.
\end{description}

\section*{General}
\begin{description}
\item[Differential Equation] An equation involving derivatives.  Give
  example.
\item[Special Case] $\frac{dy}{dx}=f(x)$, then $y=\int
  f(x)\;dx=F(x)+C$ where $F$ is antiderivative of $f$.
\item[Example] $\frac{dy}{dx}=x^5+sin(x)$.
\item[Initial Value Problem] Find $C$. $\frac{dy}{dx}=f(x)$ and $y(x_0)=y_0$.
\item[Example] $\dfrac{dy}{dx}=\sqrt{x}+x^2$ and $y(9)=250$.
\end{description}

\section*{Group Work}
\begin{description}
\item[Section 6.3] 6, 11, 14, 24
\end{description}
\end{document}
