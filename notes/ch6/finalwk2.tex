\documentclass[11pt]{article} 
\usepackage{calc}
\usepackage[margin={1in,1in}]{geometry} 
\usepackage[hwkhandout]{hwk}
\usepackage[pdftitle={Calc 1
  Notes},colorlinks=true,urlcolor=blue]{hyperref}
\usepackage{graphicx}
\usepackage{multicol}
\usepackage{enumerate}
\renewcommand{\theclass}{\textsc{math}1300: calculus I}
\renewcommand{\theauthor}{Tyson Gern}
\renewcommand{\theassignment}{Practice Final Exam}
\renewcommand{\dateinfo}{spring 2012}

\newcommand{\ds}{\displaystyle}

\begin{document}
\drawtitle

\noindent Please answer the following questions completely,
\textbf{showing all work}.  If you use your calculator please indicate
how it was used.  Clearly state each answer.

\begin{enumerate}
\item Compute the following derivatives and indefinite integrals.  You
  may \textbf{not} use your calculator on this portion of the exam.
  \begin{enumerate}
  \item $\ds\frac{d}{dx}\left(\sqrt[6]{\frac{\ln(x)}{x^{-1}}}\right)$
    \vfill
  \item $\ds\int\left(\frac{1}{3x+1}\right)\;dx$
    \vfill
    \newpage
  \item $\ds\frac{d}{dx}\left(\int_{x^2}^{\tan(x)}\tan^2(t)\;dt\right)$ \vfill
  \item $\ds\int\sin\left(-2x\right)\;dx$
    \vfill
    \newpage
  \item $\ds\frac{d}{dx}\left(e^{\pi}-\pi^e+\ln(4)^x\right)$
    \vfill
  \item $\ds\int\left(\frac{1}{\cos^2(x)}\right)\;dx$
    \vfill
  \end{enumerate}

  \newpage

\item Calculate the following limits.
  \begin{enumerate}
  \item $\ds\lim_{x\to \infty}\frac{4x^{-3}+7x^4-x}{8x^e-9x^4}$ \vfill
  \item $\ds\lim_{x\to 2}\frac{x^2-5x+6}{x^2-3x+2}$ \vfill
  \item $\ds\lim_{x\to \infty}\frac{5+6x}{e^x}$ \vfill\newpage
  \item $\ds\lim_{x\to 0}\frac{x^3-3+7x}{2^x}$ \vfill
  \item $\ds\lim_{x\to 0}\left(\frac{1}{x}-\frac{1}{e^x-1}\right)$ \vfill
  \end{enumerate}
  
  \newpage

\item A cone with a height of 16 meters and a radius of 8 meters at
  the top is leaking water at a rate of $-4$ meters$^3$/hour.  When
  the water in the cone is 8 meters deep, how fast is the height of
  the water dropping in meters/hour? (The volume of a cone is given by
  $V=\frac{1}{3}\pi r^2 h$)

  \newpage

\item Let $f(x)$ and $g(x)$ be functions with values defined below.
  \[
  \begin{array}{c||c|c|c|c|c}
    x & -2 & -1 & 0 & 1 & 2\\
    \hline\hline
    f(x) & -2 & -1 & 1 & 2 & 4\\ \hline
    f'(x) & 1 & 2 & 2 & 1 & 2\\ \hline
    g(x) & -1 & 0 & 1 & -1 & -2\\ \hline
    g'(x) & 2 & 1 & 0 & -2 & -1\\ 
  \end{array}
  \]
  Calculate the following.
  \begin{enumerate}
  \item Let $h(x)=f(x)g(x)$.  Find $h'(1)$.\vfill
  \item Let $h(x)=\dfrac{f(x)}{g(x)}$.  Find $h'(-2)$.\vfill
  \item Let $h(x)=f(g(x))$.  Find $h'(-2)$.\vfill
  \item Let $h(x)=f^{-1}(x)$.  Find $h'(1)$.\vfill
  \end{enumerate}

  \newpage

\item Let $\ds F(x)=\int_{\pi/2}^x \dfrac{\sin(t)}{t}\;dt$.  Find the
  global min and global max of $F(x)$ on the interval
  $\left[\frac{\pi}{2},\frac{3\pi}{2}\right]$.

  \newpage

\item A line goes through the origin and a point on the curve $y=x^2
  e^{-3x}$, for $x\geq 0$.  Find the maximum slope of such a line.  At
  what $x$-values does it occur?

  \newpage

\item If $V(r)$ represents the volume of a sphere with radius $r$ and
  $S(r)$ represents its surface area, it can be shown that
  $V'(r)=S(r)$.  Use the fact that $S(r)=4\pi r^2$ to obtain the
  formula for $V(r)$.

  \vfill

\item Find $\int_2^5 f(x)\;dx$ given the following.
  \begin{enumerate}
  \item $f(x)$ is even, $\int_{-2}^2 f(x)\;dx=7$, and $\int_{-5}^5 f(x)\;dx=15$

    \vfill

  \item $\int_2^5 (3f(x)+4)\;dx = 18$

    \vfill
  \end{enumerate}
  
\end{enumerate}

\end{document}
