\documentclass[11pt]{article} 
\usepackage{calc}
\usepackage[margin={1in,1in}]{geometry} 
\usepackage[hwkhandout]{hwk}
\usepackage{tikz}
\usepackage{graphicx}
\usepackage[pdftitle={Calc 1
  Notes},colorlinks=true,urlcolor=blue]{hyperref}

\renewcommand{\theclass}{\textsc{math}1300: calculus I}
\renewcommand{\theauthor}{Tyson Gern}
\renewcommand{\theassignment}{The Antiderivative}
\renewcommand{\dateinfo}{section 6.1}

\newcommand{\ds}{\displaystyle}

\begin{document}
\drawtitle

\begin{enumerate}
  
\item Give the values of the derivative $f'(x)$ in the table below and that
  $f(0)=50$, estimate $f(x)$ for $x=2,4,6$.
  \[
  \begin{array}{c|c|c|c|c}
    \hline
    x & 0 & 2 & 4 & 6\\
    \hline
    f'(x) & 10 & 18 & 23 & 25 \\
    \hline
  \end{array}
  \]
  
  \vfill
  
\item Given the graph of $f(x)$, below, sketch the graph of
  \emph{two} different antiderivatives for $f$.
  \begin{center}
    \begin{tikzpicture}
      \def\xmin{-4}
      \def\xmax{4}
      \def\ymin{-3}
      \def\ymax{3}
      \draw[<->] (\xmin,0) -- (\xmax,0);
      \draw[<->] (0,\ymin) -- (0,\ymax);
      \draw[thick, domain=\xmin:\xmax] plot[samples=100]
      function{1.5*sin(x)};
    \end{tikzpicture}
  \end{center}
  
  \newpage
  
\item Let $g(x)$ be defined as below.
  \begin{center}
    \begin{tikzpicture}[xscale = 2, yscale = 1.3]
      \def\xmin{0}
      \def\xmax{5.5}
      \def\ymin{-2.25}
      \def\ymax{1.5}
      \draw[->] (\xmin,0) -- (\xmax,0);
      \draw[<->] (0,\ymin) -- (0,\ymax);
      \foreach \i in {1,...,5} {
        \draw (\i, .1) -- (\i,-.1) node[below] {\i};
      }
      \draw[thick, domain=\xmin:5] plot[samples=100]
      function{.15*x*(x-2)*(x-4)*(x-5)} node[above] {$g(x)$};
      \draw[<-] (1,-1) -- (2,-1) node[right] {$\text{Area} = 16$};
      \draw[<-] (3,.5) -- (2,.5) node[left] {$\text{Area} = 8$};
      \draw[<-] (4.5,-.25) -- (4.5,-1) node[below] {$\text{Area} = 2$};
    \end{tikzpicture}
  \end{center}
  Use this information to graph an antiderivative $G(t)$ of $g(t)$
  satisfying $G(0)=5$.  Label each critical point of $G(t)$ with its
  coordinates.
  
  \newpage
  
\item Use the graph of $f(x)=2\sin(x^2)$, given below, to determine
  where an antiderivative, $F(x)$, of $f(x)$ function reaches its maximum
  on $0\leq x\leq 3$. If $F(1)=5$, use a calculator to find the
  maximum $y$-value attained by $F$.
  \begin{center}
    \begin{tikzpicture}[xscale = 3, yscale = 1.5]
      \def\xmin{0}
      \def\xmax{3.25}
      \def\ymin{-2}
      \def\ymax{2}
      \draw[->] (\xmin,0) -- (\xmax,0);
      \draw[<->] (0,\ymin) -- (0,\ymax);
      \foreach \i in {1,...,3} {
        \draw (\i, .1) -- (\i,-.1) node[below] {\i};
      }
      \draw[thick, domain=\xmin:\xmax] plot[samples=100]
      function{2*sin(x*x)} node[right] {$f(x)=2\sin(x^2)$};
    \end{tikzpicture}
  \end{center}
  
  \vfill
  
  \newpage
  
\item Let $f''(x)$ be given below.
  \begin{center}
    \begin{tikzpicture}[xscale = .4, yscale = 2.5]
      \def\xmin{-15}
      \def\xmax{15}
      \def\ymin{-1}
      \def\ymax{1}
      \draw[<->] (\xmin,0) -- (\xmax,0);
      \draw[<->] (0,\ymin) -- (0,\ymax);
      \draw (-10, .08) -- (-10, -.08) node[below] {$x_1$};
      \draw (-5, .08) -- (-5, -.08) node[below] {$x_2$};
      \node at (.7,.08) {$x_3$};
      \draw (2.3, .08) -- (2.3, -.08) node[below] {$x_4$};
      \draw (10, .08) -- (10, -.08) node[below] {$x_5$};
      \draw[thick, domain=\xmin:\xmax] plot[samples=100]
      function{-x/(abs(x)+1)} node[above] {$f''(x)$};
    \end{tikzpicture}
  \end{center}
  Draw graphs of $f(x)$ and $f'(x)$, assuming both go through the
  origin, and use them to decide at which of the labeled $x$-values:
  \begin{enumerate}
  \item $f''(x)$ is the greatest.
    \vfill
  \item $f''(x)$ is the least.
    \vfill
  \item $f'(x)$ is the greatest.
    \vfill
  \item $f'(x)$ is the least.
    \vfill
  \item $f(x)$ is the greatest.
    \vfill
  \item $f(x)$ is the least.
    \vfill
  \end{enumerate}
  
  \newpage
  
\item The vertical velocity of a cork bobbing up and down on the waves
  in the sea is given below.
  \begin{center}
    \begin{tikzpicture}[yscale = 1, xscale = 1]
      \def\xmin{0}
      \def\xmax{11}
      \def\ymin{-2.1}
      \def\ymax{2.1}
      \draw[->] (\xmin,0) -- (\xmax,0) node[right] {time};
      \draw[<->] (0,\ymin) -- (0,\ymax) node[above] {velocity};
      \draw[thick, domain=\xmin:\xmax] plot[samples=100]
      function{2*sin(x)};
      \draw[fill = black] (1.57,2) circle (.12);
      \node at (1.57, 2.35) {A};
      \draw[fill = black] (3.14,0) circle (.12);
      \node at (3, -.3) {B};
      \draw[fill = black] (4.71,-2) circle (.12);
      \node at (4.71, -2.35) {C};
      \draw[fill = black] (6.28,0) circle (.12);
      \node at (6.5, -.3) {D};
    \end{tikzpicture}
  \end{center}
  Upward is considered positive.
  \begin{enumerate}
  \item Describe the motion of the cork at each of the labeled points.
    \vfill
    \newpage
  \item At which point(s), if any, is the acceleration zero?
    \vfill
  \item Sketch a graph of the height of the cork above the sea floor
    as a function of time.
    \vfill
  \end{enumerate}
  
\end{enumerate}

\end{document}
