\documentclass[11pt]{article} 
\usepackage{calc}
\usepackage[margin={1in,1in}]{geometry} 
\usepackage[hwkhandout]{hwk}
\usepackage{tikz}
\usepackage{graphicx}
\usepackage[pdftitle={Calc 1
  Notes},colorlinks=true,urlcolor=blue]{hyperref}

\renewcommand{\theclass}{\textsc{math}1300: calculus I}
\renewcommand{\theauthor}{Tyson Gern}
\renewcommand{\theassignment}{The Antiderivative}
\renewcommand{\dateinfo}{section 6.1}

\newcommand{\ds}{\displaystyle}

\begin{document}
\drawtitle

\begin{enumerate}

\item Give the values of the derivative $f'(x)$ in the table and that
  $f(0)=50$, estimate $f(x)$ for $x=2,4,6$.
  \[
    \begin{array}{c|c|c|c|c}
      \hline
      x & 0 & 2 & 4 & 6\\
      \hline
      f'(x) & 10 & 18 & 23 & 25 \\
      \hline
   \end{array}
    \]

   \vfill

  \item Given the graph of $f(x)$, below, sketch the graph of
    \emph{two} different antiderivatives for $f$.
    \begin{center}
      \begin{tikzpicture}
        \def\xmin{-4}
        \def\xmax{4}
        \def\ymin{-4}
        \def\ymax{4}
        \draw[->] (\xmin,0) -- (\xmax,0);
        \draw[->] (0,\ymin) -- (0,\ymax);
        \draw[thick, domain=\xmin:\xmax] plot[samples=100]
      function{sin(x)};
    \end{tikzpicture}
  \end{center}

\newpage

\item Let $g(t)$ be defined as below.
  \begin{center}
    \includegraphics{graphofg.pdf}
  \end{center}
  Use this information to graph an antiderivative $G(t)$ of $g(t)$
  satisfying $G(0)=5$.  Label each critical point of $G(t)$ with its
  coordinates.

  \vfill

\item Use a graph of $f(x)=2\sin(x^2)$ to determine where an
  antiderivative, $F$, of this function reaches its maximum on $0\leq
  x\leq 3$. If $F(1)=5$, find the maximum $y$-value attained by $F$.

  \vfill

  \newpage

\item Let $f''(x)$ be given below.
  \begin{center}
    \includegraphics{graphoff.pdf}
  \end{center}
  Draw graphs of $f(x)$ and $f'(x)$, assuming both go through the
  origin, and use them to decide at which of the labeled $x$-values:
  \begin{enumerate}
  \item $f(x)$ is the greatest.
    \vfill
  \item $f(x)$ is the least.
    \vfill
  \item $f'(x)$ is the greatest.
    \vfill
  \item $f'(x)$ is the least.
    \vfill
  \item $f''(x)$ is the greatest.
    \vfill
  \item $f''(x)$ is the least.
    \vfill
  \end{enumerate}

  \newpage

\item The vertical velocity of a cork bobbing up and down on the waves
  in the sea is given below.
  \begin{center}
    \includegraphics{corkgraph.pdf}
  \end{center}
  Upward is considered positive.
  \begin{enumerate}
  \item Describe the motion of the cork at each of the labeled points.
    \vfill
  \item At which point(s), if any, is the acceleration zero?
    \vfill
  \item Sketch a graph of the height of the cork above the sea floor
    as a function of time.
    \vfill
  \end{enumerate}

\end{enumerate}

\end{document}
