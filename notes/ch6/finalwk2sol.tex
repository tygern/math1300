\documentclass[11pt]{article} 
\usepackage{calc}
\usepackage[margin={1in,1in}]{geometry} 
\usepackage[hwkhandout]{hwk}
\usepackage[pdftitle={Calc 1
  Notes},colorlinks=true,urlcolor=blue]{hyperref}
\usepackage{graphicx}
\usepackage{multicol}
\usepackage{enumerate}
\renewcommand{\theclass}{\textsc{math}1300: calculus I}
\renewcommand{\theauthor}{Tyson Gern}
\renewcommand{\theassignment}{Practice Final Exam}
\renewcommand{\dateinfo}{spring 2012}

\newcommand{\ds}{\displaystyle}

\begin{document}
\drawtitle

\noindent Please answer the following questions completely,
\textbf{showing all work}.  If you use your calculator please indicate
how it was used.  Clearly state each answer.

\begin{enumerate}
\item Compute the following derivatives and indefinite integrals.  You
  may \textbf{not} use your calculator on this portion of the exam.
  \begin{enumerate}
  \item $\ds\frac{d}{dx}\left(\sqrt[6]{\frac{\ln(x)}{x^{-1}}}\right)$
    \vfill
    \[
    (1+\ln(x))\cdot\frac{1}{6}\cdot (x\ln(x))^{-5/6}
    \]
    \vfill
  \item $\ds\int\left(\frac{1}{3x+1}\right)\;dx$
    \vfill
    \[
    \frac{1}{3}\ln|3x+1|+C
    \]
    \vfill
    \newpage
  \item $\ds\frac{d}{dx}\left(\int_{x^2}^{\tan(x)}\tan^2(t)\;dt\right)$
    \vfill
    \[
    \frac{1}{\cos^2(x)}\cdot\tan^2(\tan(x))-2x\cdot\tan^2(x^2)
    \]
    \vfill
  \item $\ds\int\sin\left(-2x\right)\;dx$
    \vfill
    \[
    \frac{1}{2}\cos(-2x)+C
    \]
    \vfill
    \newpage
  \item $\ds\frac{d}{dx}\left(e^{\pi}-\pi^e+\ln(4)^x\right)$
    \vfill
    \[
    \ln(\ln(4))\ln(4)^x
    \]
    \vfill
  \item $\ds\int\left(\frac{1}{\cos^2(x)}\right)\;dx$
    \vfill
    \[
    \tan(x)+C
    \]
    \vfill
  \end{enumerate}

  \newpage

\item Calculate the following limits.
  \begin{enumerate}
  \item $\ds\lim_{x\to \infty}\frac{4x^{-3}+7x^4-x}{8x^e-9x^4}$
    \vfill
    \[
    -\frac{7}{9}
    \]
    \vfill
  \item $\ds\lim_{x\to 2}\frac{x^2-5x+6}{x^2-3x+2}$
    \vfill
    \[
    =\lim_{x\to 2}\frac{x-3}{x-1}=-1
    \]
    \vfill
  \item $\ds\lim_{x\to \infty}\frac{5+6x}{e^x}$
    \vfill
    \[
    =\lim_{x\to\infty} \frac{6}{e^x}=0
    \]
    \vfill\newpage
  \item $\ds\lim_{x\to 0}\frac{x^3-3+7x}{2^x}$
    \vfill
    \[
    -3
    \]
    \vfill
  \item $\ds\lim_{x\to 0}\left(\frac{1}{x}-\frac{1}{e^x-1}\right)$
    \vfill
    \[
    =\lim_{x\to 0}\frac{e^x-1-x}{xe^x-x} =\lim_{x\to
      0}\frac{e^x-1}{xe^x+e^x-1} =\lim_{x\to
      0}\frac{e^x}{xe^x+2e^x}=\frac{1}{2}
    \]
    \vfill
  \end{enumerate}
  
  \newpage

\item A cone with a height of 16 meters and a radius of 8 meters at
  the top is leaking water at a rate of $-4$ meters$^3$/hour.  When
  the water in the cone is 8 meters deep, how fast is the height of
  the water dropping in meters/hour? (The volume of a cone is given by
  $V=\frac{1}{3}\pi r^2 h$)

  \vfill

  By similar triangles we know $r=\frac{1}{2}h$.  Then
  \begin{align*}
    V &=\frac{\pi}{12}h^3 \\
    \intertext{so}
    \frac{dV}{dt} &=\frac{\pi}{4}h^2\frac{dh}{dt}\\
    \intertext{then when $h=8$}
    -4 &=\frac{\pi}{4}\cdot 64 \cdot \frac{dh}{dt}\\
    \intertext{so}
    \frac{dh}{dt} &= -0.079599 \text{ m/h}.
  \end{align*}

  \vfill

  \newpage

\item Let $f(x)$ and $g(x)$ be functions with values defined below.
  \[
  \begin{array}{c||c|c|c|c|c}
    x & -2 & -1 & 0 & 1 & 2\\
    \hline\hline
    f(x) & -2 & -1 & 1 & 2 & 4\\ \hline
    f'(x) & 1 & 2 & 2 & 1 & 2\\ \hline
    g(x) & -1 & 0 & 1 & -1 & -2\\ \hline
    g'(x) & 2 & 1 & 0 & -2 & -1\\ 
  \end{array}
  \]
  Calculate the following.
  \begin{enumerate}
  \item Let $h(x)=f(x)g(x)$.  Find $h'(1)$.
    \vfill
    \[
    h'(1)=f'(1)g(1)+g'(1)f(1)= -5
    \]
    \vfill
  \item Let $h(x)=\dfrac{f(x)}{g(x)}$.  Find $h'(-2)$.
    \vfill
    \[
    h'(-2)=\frac{g(-2)f'(-2)-f(-2)g'(-2)}{g(-2)^2}= 3
    \]
    \vfill
  \item Let $h(x)=f(g(x))$.  Find $h'(-2)$.
    \vfill
    \[
    h'(-2)=g'(-2)\cdot f'(g(-2)) = 4
    \]
    \vfill
  \item Let $h(x)=f^{-1}(x)$.  Find $h'(1)$.
    \vfill
    \[
    h'(1)=\frac{1}{f'(f^{-1}(1))}=\frac{1}{2}
    \]
    \vfill
  \end{enumerate}

  \newpage

\item Let $\ds F(x)=\int_{\pi/2}^x \dfrac{\sin(t)}{t}\;dt$.  Find the
  global min and global max of $F(x)$ on the interval
  $\left[\frac{\pi}{2},\frac{3\pi}{2}\right]$.

  \vfill

  By the second FTOC we know that $F'(x)=\frac{\sin(x)}{x}$, so the
  only critical point in the given interval is at $x=\pi$.  Then
  \begin{align*}
    F\left(\frac{\pi}{2}\right) &= \int_{\pi/2}^{\frac{\pi}{2}}
    \dfrac{\sin(t)}{t}\;dt = 0\\
    F\left(\pi\right) &= \int_{\pi/2}^{\pi}
    \dfrac{\sin(t)}{t}\;dt \approx 0.481175\\
    F\left(\frac{\pi}{2}\right) &= \int_{\pi/2}^{\frac{3\pi}{2}}
    \dfrac{\sin(t)}{t}\;dt \approx 0.237611\\
  \end{align*}
  so $F$ has a max of $0.481175$ at $x=\pi$ and a min of $0$ at
  $x=\frac{\pi}{2}$.

  \vfill

  \newpage

\item A line goes through the origin and a point on the curve $y=x^2
  e^{-3x}$, for $x\geq 0$.  Find the maximum slope of such a line.  At
  what $x$-values does it occur?

  \vfill

  The slope of the line through the points $(0,0)$ and $(x,y)$ is
  given by
  \[
  s(x)= \frac{\Delta y}{\Delta
    x}=\frac{y}{x}=\frac{x^2e^{-3x}}{x}=xe^{-3x}.
  \]
  Then we have a critical point when
  \[
  s'(x)=e^{-3x}-3xe^{-3x}=(1-3x)e^{-3x}=0,
  \]
  so at $x=\frac{1}{3}$.  We see that $s(x)\to 0$ as $x\to 0$ or
  $x\to\infty$, so the critical point is a max.  Then the line has a
  maximum slope of $s(1/3)=\frac{1}{3e}\approx 0.122626$ at $x=\frac{1}{3}$.
  
  \vfill
  \newpage

\item If $V(r)$ represents the volume of a sphere with radius $r$ and
  $S(r)$ represents its surface area, it can be shown that
  $V'(r)=S(r)$.  Use the fact that $S(r)=4\pi r^2$ to obtain the
  formula for $V(r)$.

  \vfill

  \[
  V(r)=\int S(r)\;dr=\frac{4}{3}\pi r^3 +C.
  \]
  We know that $V(0)=0$, so $C=0$ and thus $V(r)=\frac{4}{3}\pi r^3$.

  \vfill

\item Find $\int_2^5 f(x)\;dx$ given the following.
  \begin{enumerate}
  \item $f(x)$ is even, $\int_{-2}^2 f(x)\;dx=7$, and $\int_{-5}^5 f(x)\;dx=15$

    \vfill

    Since $f(x)$ is even we know that
    \begin{align*}
      \int_0^2 f(x)\;dx &= \frac{1}{2}\int_{-2}^2 f(x)\;dx = \frac{7}{2}\\
      \intertext{and}
      \int_0^5 f(x)\;dx = \frac{1}{2}\int_{-5}^5 f(x)\;dx = \frac{15}{2}\\
    \end{align*}
    Then
    \begin{align*}
      \int_2^5 f(x)\;dx &= \int_2^0 f(x)\; dx +\int_0^5 f(x)\; dx\\
      &= -\int_0^2 f(x)\; dx +\int_0^5 f(x)\; dx\\
      &= -\frac{7}{2}+\frac{15}{2}\\
      &= 4
    \end{align*}
    
    \vfill

  \item $\int_2^5 (3f(x)+4)\;dx = 18$

    \vfill

    \[
    18=\int_2^5 (3f(x)+4)\;dx=3\int_2^5 f(x)\;dx+\int_2^5
    4\;dx=3\int_2^5 f(x)\;dx+12,
    \]
    so
    \[
    \int_2^5 f(x)\;dx = 2.
    \]
    
    \vfill
  \end{enumerate}
  
\end{enumerate}

\end{document}
