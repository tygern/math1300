\documentclass[11pt]{article} 
\usepackage{calc}
\usepackage[margin={1in,1in}]{geometry} 
\usepackage[hwkhandout]{hwk}
\usepackage[pdftitle={Calc 1
  Notes},colorlinks=true,urlcolor=blue]{hyperref}
%\usepackage{mathpazo}
\usepackage{multicol}

\renewcommand{\theclass}{\textsc{math}1300: calculus I}
\renewcommand{\theauthor}{Tyson Gern}
\renewcommand{\theassignment}{Finding Antiderivatives}
\renewcommand{\dateinfo}{section 6.2}

\newcommand{\ds}{\displaystyle}

\begin{document}
\drawtitle

\section*{Intro}
\begin{description}
\item[Goal] Given a function $f(x)$ find all antiderivatives.  This is
  \textbf{hard} in general.
\item[Notation] Indefinite integral.  Let $f(x)$ be a continuous
  function and let $F(x)$ be an antiderivative for $f(x)$.  Then
  \[
  \int f(x)\;dx = F(x)+C.
  \]
\item[Caution] Note the difference between the definite and indefinite
  integrals.  One is a number, the other is a family of functions.
\end{description}

\section*{Rules}
\begin{description}
\item[Question] ``What is $\int f(x)\;dx$?'' is the same as ``Find all
  antiderivatives for $f(x)$''.
\item[Antiderivatives] Know these
  \begin{multicols}{2}
    \begin{itemize}
    \item $f(x)=0$
    \item $f(x)=k$
    \item $f(x)=x^n$, $n\neq 1$
    \item $f(x)=\dfrac{1}{x}$
    \item $f(x)=e^x$
    \item $f(x)=\sin(x)$
    \item $f(x)=\cos(x)$
    \end{itemize}
  \end{multicols}
\item[Properties] Sum and constant multiple.
\item[Examples] Combinations of polynomials and others.
\end{description}

\section*{Application}
\begin{description}
\item[Compute] Use antiderivative to compute definite integral.
  \[
  \int_a^b f(x)\;dx=F(b)-F(a)=F(x)\Big\vert_a^b
  \]
\item[Practice] Do some
\end{description}

\section*{Group Work}
\begin{description}
\item[Section 6.2] Various definite and indefinite integrals
\end{description}
\end{document}
