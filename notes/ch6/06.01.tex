\documentclass[11pt]{article} 
\usepackage{calc}
\usepackage[margin={1in,1in}]{geometry} 
\usepackage[hwkhandout]{hwk}
\usepackage[pdftitle={Calc 1
  Notes},colorlinks=true,urlcolor=blue]{hyperref}
%\usepackage{mathpazo}

\renewcommand{\theclass}{\textsc{math}1300: calculus I}
\renewcommand{\theauthor}{Tyson Gern}
\renewcommand{\theassignment}{The Antiderivative}
\renewcommand{\dateinfo}{section 6.1}

\newcommand{\ds}{\displaystyle}

\begin{document}
\drawtitle

\section*{Recall}
\begin{description}
\item[FTOC] State theorem.  Goal: find $F(x)$.
\item[Definition] Antiderivative
\end{description}

\section*{Equations}
\begin{description}
\item[Example] Let $f(x)=2x$.  Then $F(x)=x^2$ is an antiderivative.
  Are there any others?
\item[Graph] Look at the above situation.  How do the shapes compare?
  Differ by a constant.
\item[Idea] Let $f(x)$ be a continuous function with antiderivative
  $F(x)$.  Then any antiderivative of $f(x)$ is of the form
  \[
  F(x)+C
  \]
  where $C$ is constant.  Family of functions.
\end{description}

\section*{Graphical}
\begin{description}
\item[Graphs] Draw graphs and find antiderivatives.  Later, say $f(0)=4$.
\end{description}

\section*{Numerical}
\begin{description}
\item[Idea] Use FTOC to find $F(x)$ from $f(x)$.
  \[
  F(b)-F(a)-\int_a^b f(x)\;dx
  \]
\item[Example] Draw piecewise constant function called $f'(x)$.  Let
  $f(0)=-2$ and find $f(6)$.  If time do with table and estimations.
\end{description}

\section*{Group Work}
\begin{description}
\item[Section 6.1] 6, 15, 18, 21
\end{description}
\end{document}
