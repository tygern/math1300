\documentclass[11pt]{article} 
\usepackage{calc}
\usepackage[margin={1in,1in}]{geometry} 
\usepackage[hwkhandout]{hwk}
\usepackage[pdftitle={Calc 1
  Information},colorlinks=true,urlcolor=blue]{hyperref}


\renewcommand{\theclass}{\textsc{math}1300: calculus I}
\renewcommand{\theauthor}{Tyson Gern}
\renewcommand{\theassignment}{Polynomial and Rational Functions}
\renewcommand{\dateinfo}{section 1.6}

\def\ww{WeBWorK }

\begin{document}
\drawtitle

\section*{Basics}
\begin{description}
\item[Power] $kx^n$ for $n$ odd or even
\item[Polynomials] Sum or difference of power functions.
  \begin{itemize}
  \item shape
  \item zeros: number, factor, graphs (multiplicity)
  \item 2 zeros at -3, 1 at 2, f(0)=10
  \end{itemize}
\end{description}

\section*{Rational Functions}
\begin{description}
\item[Definition] $\displaystyle f(x)=\frac{p(x)}{q(x)}$ where $p(x)$
  and $q(x)$ are polynomials.
\item[Asymptotes] visual, how to find
  \begin{itemize}
  \item vertical: $q(x)=0$, $p(x)\neq 0$
  \item horizontal: what happens when $x$ gets very large or very
    small?
  \end{itemize}
\end{description}

\section*{Group Work}
\begin{description}
\item[Section 1.6] 12, 13, 15, 22, 26
\end{description}
\end{document}
