\documentclass[11pt]{article} 
\usepackage{calc}
\usepackage[margin={1in,1in}]{geometry} 
\usepackage[hwkhandout]{hwk}
\usepackage[pdftitle={Calc 1
  Information},colorlinks=true,urlcolor=blue]{hyperref}


\renewcommand{\theclass}{\textsc{math}1300: calculus I}
\renewcommand{\theauthor}{Tyson Gern}
\renewcommand{\theassignment}{Trigonometric Functions}
\renewcommand{\dateinfo}{section 1.5}

\def\ww{WeBWorK }

\begin{document}
\drawtitle

\section*{Basics}
\begin{description}
\item[Radians] unit circle
\item[Definitions] $\sin(x)$ and $\cos(x)$, with circle.  Also mention
  triangles. $\tan(x)=\displaystyle\frac{\sin(x)}{\cos(x)}$
\item[Identity] $a^2+b^2=1\Rightarrow \cos^2(x)+\sin^2(x)=1$
\end{description}

\section*{Properties}
\begin{description}
\item[Graphs] Draw all
\item[Periodic] definition
  \begin{itemize}
  \item amplitude
  \item period
  \end{itemize}
\item[Functions] ``sinusoidal'' - $f(t)=A\sin(Bt)+C$, $g(t)=A\cos(Bt)+C$
\item[Mention] inverse trig functions.  (lookup)
\end{description}

\section*{Example}
\begin{description}
\item[Modeling] temperature: min temp of 50$^{\circ}$ at 6\textsc{am},
  max temp of 80$^{\circ}$ at 6\textsc{pm}.  Draw graph.
\end{description}

\section*{Group Work}
\begin{description}
\item[Section 1.5] 27, 38-41, 45
\end{description}
\end{document}
