\documentclass[11pt]{article} 
\usepackage{calc}
\usepackage[margin={1in,1in}]{geometry} 
\usepackage[hwkhandout]{hwk}
\usepackage[pdftitle={Calc 1
  Information},colorlinks=true,urlcolor=blue]{hyperref}


\renewcommand{\theclass}{\textsc{math}1300: calculus I}
\renewcommand{\theauthor}{Tyson Gern}
\renewcommand{\theassignment}{Logarithms}
\renewcommand{\dateinfo}{section 1.4}

\def\ww{WeBWorK }

\begin{document}
\drawtitle

\section*{Doubling Time Example}

\begin{description}
\item[Problem]\textit{Suppose a town has a population that grows at a
    rate of 10\% per year.  Find the doubling time for the population
    of the town.}

\vspace{.3in}

We know that the equation for the population of the town is given by
\[
P(t)=P_0 (1.1)^t,
\]
where $P_0$ is the initial population of the town.  Then the doubling
time will be the value for $t$ where the following equations holds:
\[
2P_0=P_0(1.1)^t.
\]
We can divide both sides of the above equation by $t$ to get
\[
2=(1.1)^t.
\]
Note that the above equation does not involve $P_0$.  This means that
the initial population is not needed for determining doubling time!
Now we just have to solve for $t$.  We can do so using logarithms.  We
have

\begin{align*}
  2 &= (1.1)^t\\
  \ln(2) &= \ln\left((1.1)^t\right)\\
  \ln(2) &= t\cdot\ln\left(1.1\right)\\
  t &= \frac{\ln(2)}{\ln(1.1)}\approx 7.273,
\end{align*}

so the doubling time for the population of the town is about $7.273$
years.


\end{description}

\end{document}
