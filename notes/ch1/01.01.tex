\documentclass[11pt]{article} 
\usepackage{calc}
\usepackage[margin={1in,1in}]{geometry} 
\usepackage[hwkhandout]{hwk}
\usepackage[pdftitle={Calc 1
  Information},colorlinks=true,urlcolor=blue]{hyperref}


\renewcommand{\theclass}{\textsc{math}1300: calculus I}
\renewcommand{\theauthor}{Tyson Gern}
\renewcommand{\theassignment}{Functions and Rate of Change}
\renewcommand{\dateinfo}{section 1.1}

\def\ww{WeBWorK }

\begin{document}
\drawtitle

\section*{General}
\begin{description}
\item[Definitions] function, range, domain
\item independent and dependent variables
\item Draw simple pictures and explain interval notation.
\item[Examples] graph, equation, table, word
  \begin{itemize}
  \item Temperature over a day
  \item Gas tank
  \item Water boil temperature
  \end{itemize}
\item[Check] vertical line test
\end{description}

\section*{Linear Functions}
\begin{description}
\item[Definition] words, graph, etc.
\item[Slope] \textit{rate of change}, rise/run, $\frac{\Delta
    y}{\Delta x}$, difference quotient, $y=mx+b$.
\item[Examples]
  \begin{itemize}
  \item Gas tank
  \item Cost of manufacturing
  \end{itemize}
\item[Computation] Given two points, find a line.

Gas Bill: 1000 cu ft cost \$35, 1500 cu ft cost \$50 ($0.03x+5$)

\end{description}

\section*{Properties}
\begin{description}
\item[Terms] increasing, decreasing, constant, proportional, inversely
  proportional
\item[Examples] Ask for suggestions
  \begin{itemize}
  \item $C=\pi\cdot D$
  \item 2 foot tall box with area of 10
  \end{itemize}
\end{description}

\section*{Group Work}
\begin{description}
\item[Section 1.1] 36, 41, 43
\end{description}
\end{document}
