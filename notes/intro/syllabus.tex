\documentclass[11pt]{article} 
\usepackage{calc, color}
\usepackage[margin={.9in,.9in}]{geometry} 
\usepackage[hwkhandout]{hwk}
\definecolor{darkgray}{rgb}{0.4,0.4,0.4}
\usepackage[pdftitle={Calc 1
  Syllabus},colorlinks=true,urlcolor=darkgray]{hyperref}

\renewcommand{\theclass}{\textsc{math} 1300-013: Calculus I}
\renewcommand{\theauthor}{Tyson Gern}
\renewcommand{\theassignment}{Syllabus}
\renewcommand{\dateinfo}{Fall 2012}

\def\ww{WeBWorK }

\begin{document}
\drawtitle

\section*{General Information}
\begin{description}
\item[Instructor:] Tyson Gern
  (\href{mailto:tyson.gern@colorado.edu}{tyson.gern@colorado.edu})
\item[Teaching Assistant:] Andrew Healy
\item[Class] Class will be held 12:00--12:50\textsc{pm} every weekday
  in MUEN D144.  You will have lecture with Tyson every day except
  Thursday, when you will have recitation with Andrew.  The course will
  move rather quickly, so it is vital that you come to every class.
\item[Office Hours:] I will hold office hours from
  1:00--1:50\textsc{pm} on Mondays and 11:00--11:50\textsc{am} on
  Wednesdays in my office, Math 364.  I urge all of you to come to
  office hours often in order to get help on homework and difficult
  concepts.
\item[Course Webpages:] I will maintain a webpage with information
  specific to our section at
  \begin{center}
    \url{http://www.tysongern.com/1300f12}.
  \end{center}
  The main course webpage contains general course information at
  \begin{center}
      \url{http://math.colorado.edu/math1300}.
  \end{center}
  Please check these websites daily for information relevant to the
  course such as homework assignments, WebWork, the course schedule,
  lists of instructors and TAs, and a copy of this syllabus.
\end{description}

\section*{Course Information}
\begin{description}

\item[Prerequisites:] \textbf{Minimum ALEKS placement score: 65.}  Two
  years of high school algebra, one year of geometry, and one
  half-year of trigonometry; \textbf{or} MATH 1150: Precalculus.

\item[Textbook:] Calculus, 5th Edition, by Hughes-Hallett, Gleason,
  McCallum, et al., Wiley, 2009.  There are two versions sold in the
  bookstore: a hardcover and a cheaper loose-leaf version (however, the
  bookstore will not buy back the loose-leaf version). The
  ``WileyPlus'' version is \textbf{not required} in spite of what it
  says on the package, and Wiley says the code will not work for CU
  anyway.

  You are expected to \textbf{read the corresponding section of the
    textbook before each class}. The lectures are intended to
  supplement and reinforce the textbook, not to replace it. Occasional
  short quizzes will be given randomly in class to see if you have
  done the required reading.

\item[Calculators:] You will be required to have an electronic device
  for in-class activities which is capable of graphing functions and
  doing numerical integration. Acceptable devices are a calculator
  such as a TI-83 or better, a graphing calculator application for a
  smartphone, software packages such as Maple or Mathematica, and web
  sites such as Wolfram Alpha.  \textbf{Absolutely no such devices
    will be allowed on exams or quizzes.}

\item[Assignments:] We believe that the only effective way to learn
  Calculus is to do problems often. There are four types of
  assignments in this course.
  \begin{description}
  \item[In-lecture problem-solving:] These problems will be
    occasionally assigned during lectures to practice techniques you
    have just learned. They will not be graded, but you may be asked
    to present solutions on the board. There is no penalty for getting
    an answer wrong or not completing a problem: everyone learns even
    from an unsuccessful attempt. Class participation is extremely
    important for effective learning, and is the only way to really
    take advantage of your instructor's knowledge.
  \item[Recitation worksheets:] The recitation is held every Thursday
    and is supervised by a graduate teaching assistant and an
    undergraduate learning assistant. In recitations, you will solve
    problems based on lecture material in small groups (3--4
    students), with the assistants answering any questions you may
    have. These worksheets will be collected and graded at the end of
    each recitation.
  \item[Written homework:] You will be assigned several conceptual
    problems out of the textbook each week. You are expected to write
    up complete, legible, and logical solutions to these problems,
    which will be graded by your teaching assistant. Homework will be
    collected and returned in Thursday recitations. It should be
    written using complete sentences to explain your steps. Late
    homework will absolutely not be accepted under any circumstances,
    but your lowest homework score will be dropped.
  \item[Online WebWork:] These problems are generally straightforward
    and computational, and you can repeat them up to five times until
    you get the correct answer. The philosophy behind this is that
    instantaneous feedback is more effective than waiting days for a
    grade, and that doing a problem over if it's wrong is better than
    simply seeing the right answer. Because problems are graded by a
    computer, there are occasional technical issues, but we believe
    the trade-off is worthwhile. See below for details.
  \end{description}

\item[Midterms:] This course has three midterm exams and a final
  exam. They have already been scheduled. \textbf{You must be able to
    attend all of them.} Calculators and cell phones will not be
  allowed during any portion of any exam. \textbf{Use of any
    electronic device at any time during the exam will be considered
    cheating.}

  Plan your schedule now. There will be \textbf{no makeup exams} given
  under any circumstances. However, \textbf{the lowest of your three
    midterm scores will be dropped automatically}, which will apply in
  particular if you cannot attend an exam due to emergency, illness,
  religious observance, or other reason.
  \begin{description}
  \item[Midterm 1:] Wednesday, September 19. 5:15 pm to 6:45 pm.
  \item[Midterm 2:] Wednesday, October 17. 5:15 pm to 6:45 pm.
  \item[Midterm 3:] Wednesday, November 14. 5:15 pm to 6:45 pm.
  \end{description}

\item[Final Exam:] The final exam for the course is
  \textbf{cumulative}.  It is scheduled for:

  \begin{description}
  \item[Wednesday, December 19. 10:30 am to 1:00 pm.]
  \end{description}

  You may not reschedule this exam even if you have three exams on the
  same day (university policy only allows for the third exam to be
  rescheduled). A makeup will only be given in case of a
  \textbf{documented emergency with documentation from a medical
    professional or the student's dean.}

\item[WebWork:] To sign up for the WebWork online homework system, go
  to
  \begin{center}
    \url{https://webwork.colorado.edu/signup}
  \end{center}
  The system will ask for your name, student ID number, Identikey
  login, three-digit section number, and email address. Your initial
  password will be your student ID; you should change it once you log
  in successfully.  Each lecture will have its own WebWork
  assignment. You are responsible for checking the web site and doing
  the assignment on time.

  Occasionally there are technical difficulties with the system. You
  may email your instructor to ask about a problem, but \textbf{only}
  under the following conditions:
  \begin{enumerate}
  \item Email will only be responded to during normal hours (9 am to 5
    pm) and not after any assignment's deadline.
  \item Only one message per problem set will be answered.
  \item You must clearly explain which problem and assignment you are
    having a problem with and what your problem is.
  \item You must clearly explain what you tried already to fix it.
  \end{enumerate}
  Any WebWork-related email which does not satisfy all four conditions
  will be ignored (although we are happy to respond to email which
  does satisfy them, and email about any other topic).  Your lowest
  three Webwork assignment scores will be dropped.

\item[Quizzes:] There are two types of quizzes. Random pop quizzes
  will be given based on your reading of the assigned textbook section
  \textbf{before} the class. They will be short and easy if you have
  read the section. This is to encourage you to read the textbook and
  to attend class. Your two lowest quiz scores of this type will be
  dropped.

  The other type of quiz is regularly scheduled for Tuesdays (except
  in midterm weeks). These will be more substantial and computational
  and based on the types of problems you have seen in lecture and
  WebWork during the previous week. The purpose of these quizzes is to
  encourage you to study throughout the semester rather than only
  before an exam; in this way you will find exams easier and less
  stressful. Your two lowest quiz scores of this type will also be
  dropped. \textbf{Calculators will not be allowed on quizzes.}

\item[Recitation Worksheets:] These projects will be distributed in
  the tutorials, and you will work on them in small groups with
  several of your classmates. (The group to which you are assigned
  will change frequently.)  A graduate teaching assistant (TA) and
  undergraduate learning assistant (LA) will be present during
  tutorials to facilitate your work on the projects, but the goal is
  for you (and your group-mates) to \textbf{work through, and
    complete, these projects on your own} as much as possible.

  Your LA and TA will be making sure that you participate in your
  group's explorations and discoveries.  You will be graded on your
  participation, so {\it participate}.  Each tutorial will be graded
  on a 10-point scale---you will receive 5 points for participating
  and working during the entire tutorial, and 5 points based on how
  much of the worksheet you complete successfully.

  Missed worksheets cannot be made up; if you miss a Thursday
  tutorial, you will receive a zero for that worksheet. However your
  lowest recitation grade will be dropped.

\item[Grades:] The grade distribution will be calculated based on the
  following rubric:
  \begin{description}
  \item[Midterms] (20\% each, two highest out of three counted)
  \item[Final Exam] (25\%)
  \item[WebWork] (10\%)
  \item[Written Homework] (10\%)
  \item[Recitation Worksheets] (5\%)
  \item[Quizzes] (10\%)
  \end{description}

\item[About the Course:] The fundamental idea of calculus is the
  principle that complicated things can be understood by first
  breaking them up into small simple pieces, doing easy calculations,
  and then putting the pieces back together. Although this type of
  technique was occasionally applied (with great difficulty) two
  thousand years ago, the true invention of calculus dates to Newton
  and Leibniz and their successors in the 1600s, who realized that
  this was a very general technique and could be applied
  systematically using a few basic formulas.

  Calculus has led to profound human achievements: initially created
  to solve basic geometric problems, it soon led to a nearly complete
  understanding of the motion of the planets. Nowadays it finds
  application constantly in chemistry, economics, biology, psychology,
  physics, and every type of engineering.  However, it should not be
  viewed as simply a set of algorithms. It is in many ways an art,
  arising purely from human imagination and capable of creating great
  beauty for its own sake.

  In this course you will learn about the basic building blocks of
  calculus, particularly functions and their derivatives and
  integrals. You will also learn a number of useful formulas, although
  that is not the primary purpose of calculus any more than correct
  spelling is the primary purpose of literature. Finally you will
  learn how to understand calculus conceptually and build your own
  approaches to solving practical problems with the tools you have.

\item[Classroom Behavior:] Students and faculty each have
  responsibility for maintaining an appropriate learning
  environment. Those who fail to adhere to such behavioral standards
  are subject to discipline.  Disruptive behavior, as applied to the
  academic setting, means behavior that a reasonable faculty member
  would view as interfering with normal academic functions. Examples
  include, but are not limited to: persistently speaking without being
  recognized or interrupting other speakers; behavior that distracts
  the class from the subject matter or discussion; or in extreme
  cases, physical threats, harassing behavior or personal insults, or
  refusal to comply with faculty direction. Instructors may ask
  disruptive students to leave the classroom.  See policies at
  \begin{center}
    \url{http://www.colorado.edu/policies/classbehavior.html}    
  \end{center}
  and
  \begin{center}
    \url{http://www.colorado.edu/studentaffairs/judicialaffairs/code.html}
  \end{center}

\item[Undergraduate Mathematics Resource Center:] You may seek
  assistance with your math questions in the Undergraduate Mathematics
  Resource Center (Help Lab) in Math 175.  The Center will open
  Monday, August 27 at 9 am, and thereafter will be open (on school
  days only) Monday--Thursday 9 am--5 pm, and Friday 9 am--2 pm.  (The
  Center closes for the semester at 2 PM on Friday, December 14.)

\end{description}

\section*{General Information}
\begin{description}
\item[Disability Accommodations:] If you qualify for accommodations
  because of a disability, please submit to your professor a letter
  from Disability Services in a timely manner (for exam accommodations
  provide your letter at least one week prior to the exam) so that
  your needs can be addressed. Disability Services determines
  accommodations based on documented disabilities. Contact Disability
  Services at 303-492-8671 or by e-mail at
  \href{mailto:dsinfo@colorado.edu}{dsinfo@colorado.edu}.

  If you have a temporary medical condition or injury, see Temporary
  Medical Conditions: Injuries, Surgeries, and Illnesses guidelines
  under Quick Links at Disability Services website and discuss your
  needs with your professor.
  \begin{center}
    \url{http://www.colorado.edu/disabilityservices/}
  \end{center}

\item[Religious Observances:] Campus policy regarding religious
  observances requires that faculty make every effort to deal
  reasonably and fairly with all students who, because of religious
  obligations, have conflicts with scheduled exams, assignments or
  required attendance.  In this class, makeups will not be provided
  but one missed exam or assignment will be dropped.  See full details
  at
  \begin{center}
    \url{http://www.colorado.edu/policies/fac_relig.html}
  \end{center}

\item[Respect for Diversity:] Professional courtesy and sensitivity
  are especially important with respect to individuals and topics
  dealing with differences of race, color, culture, religion, creed,
  politics, veteran's status, sexual orientation, gender, gender
  identity and gender expression, age, disability, and nationalities.
  Class rosters are provided to the instructor with the student's
  legal name. I will gladly honor your request to address you by an
  alternate name or gender pronoun. Please advise me of this
  preference early in the semester so that I may make appropriate
  changes to my records.

\item[Discrimination and Harassment:] The University of Colorado
  Boulder (CU-Boulder) is committed to maintaining a positive
  learning, working, and living environment. The University of
  Colorado does not discriminate on the basis of race, color, national
  origin, sex, age, disability, creed, religion, sexual orientation,
  or veteran status in admission and access to, and treatment and
  employment in, its educational programs and activities. (Regent Law,
  Article 10, amended 11/8/2001).  CU-Boulder will not tolerate acts
  of discrimination or harassment based upon Protected Classes or
  related retaliation against or by any employee or student. For
  purposes of this CU-Boulder policy, ``Protected Classes'' refers to
  race, color, national origin, sex, pregnancy, age, disability,
  creed, religion, sexual orientation, gender identity, gender
  expression, or veteran status.  Individuals who believe they have
  been discriminated against should contact the Office of
  Discrimination and Harassment (ODH) at 303-492-2127 or the Office of
  Student Conduct (OSC) at 303-492-5550.  Information about the ODH,
  the above referenced policies, and the campus resources available to
  assist individuals regarding discrimination or harassment can be
  obtained at
  \begin{center}
    \url{http://www.colorado.edu/odh}
  \end{center}

\item[Honor Code:] All students of the University of Colorado at
  Boulder are responsible for knowing and adhering to the academic
  integrity policy of this institution.  Violations of this policy may
  include: cheating, plagiarism, aid of academic dishonesty,
  fabrication, lying, bribery, and threatening behavior.  All
  incidents of academic misconduct shall be reported to the Honor Code
  Council (honor@colorado.edu; 303-735-2273). Students who are found
  to be in violation of the academic integrity policy will be subject
  to both academic sanctions from the faculty member and non-academic
  sanctions (including but not limited to university probation,
  suspension, or expulsion). Other information on the Honor Code can
  be found at 
  \begin{center}
    \url{http://www.colorado.edu/policies/honor.html}
  \end{center}
  and
  \begin{center}
    \url{http://www.colorado.edu/academics/honorcode/}
  \end{center}
\end{description}

\end{document}
