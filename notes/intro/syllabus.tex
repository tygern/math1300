\documentclass[11pt]{article} 
\usepackage{calc, color}
\usepackage[margin={.9in,.9in}]{geometry} 
\usepackage[hwkhandout]{hwk}
\definecolor{darkgray}{rgb}{0.2,0.2,0.7}
\usepackage[pdftitle={Calc 1
  Syllabus},colorlinks=true,urlcolor=darkgray]{hyperref}

\renewcommand{\theclass}{\textsc{math} 1300-003: Calculus I}
\renewcommand{\theauthor}{Tyson Gern}
\renewcommand{\theassignment}{Syllabus}
\renewcommand{\dateinfo}{Spring 2013}

\def\ww{WeBWorK }

\begin{document}
\drawtitle

\section*{Contact and Meeting Information}

\begin{description}
\item[Instructor:] Tyson Gern
  (\href{mailto:tyson.gern@colorado.edu}{tyson.gern@colorado.edu})
\item[Teaching Assistant:] Joshua Frinak
\item[Learning Assistant:] Emma Carr
\item[Class:] Class will be held 9:00--9:50\textsc{am} every weekday in
  MUEN E118.  You will have lecture with Tyson every day except
  Thursday, when you will have recitation with Joshua and Emma.  The
  course will move rather quickly, so it is vital that you come to
  every class.
\item[Office Hours:] I will hold office hours from
  10:00--10:50\textsc{am} on Mondays and 8:00--8:50\textsc{am} on
  Wednesdays in my office, Math 364.  I urge all of you to come to
  office hours often in order to get help on homework and difficult
  concepts.
\end{description}

\section*{Course Information}
\begin{description}
\item[Prerequisites:] \textbf{Minimum ALEKS placement score: 65.}  Two
  years of high school algebra, one year of geometry, and one
  half-year of trigonometry; \textbf{or} MATH 1150: Precalculus.

\item[Textbook:] Calculus, 5th Edition, by Hughes-Hallett, Gleason,
  McCallum, et al., Wiley, 2009.  There are two versions sold in the
  bookstore: a hardcover and a cheaper looseleaf version (however, the
  bookstore will not buy back the looseleaf version). The
  ``WileyPlus'' version is \textbf{not required} in spite of what it
  says on the package, and Wiley says the code will not work for CU
  anyway.

\item[Course ethos:] Education researchers have shown the efficacy of
  learning mathematics via what is sometimes called {\it direct
    learning} or {\it inquiry-based learning}. In a nutshell, they
  have found that most students learn little in traditional lecture
  courses. Rather they need to be actively engaged, solving problems
  and making discoveries and linkages as they go. As a friend of mine
  used to say, ``I've been watching people play guitar my whole life,
  but I still can't play!''

  This course and the book we are using are designed for a classroom
  which does not follow a traditional lecture format. Do not be
  surprised if your instructor spends only half a class period at the
  board lecturing or solving problems: the rest of the time, you
  should expect to be working at your desk, either singly or in
  groups, or at the board, presenting your work.

  In this vein, you will be expected to read a section in the book
  {\bf before} it is discussed in class.  Lectures are intended to
  highlight aspects of the text, not to replace it.

\item[About Calculus:] Roughly speaking, calculus is the mathematics of
  {\it change}.  In particular, calculus is a powerful tool for
  understanding change in physical quantities and phenomena that {\it
    depend on}, or are {\it related to}, each other.

  The dependence of a given quantity upon another (or others) is often
  described mathematically by a {\it function}.  Thus, the heart of
  calculus {\it is} the study of functions, and how they
  change. Differential calculus studies the instantaneous change of a
  function as quantities vary, and integral calculus measures the
  cumulative effect of the change of a function.

  Calculus is applied constantly in mathematics, chemistry, economics,
  biology, psychology, physics, and every type of engineering.
  However, it need not be viewed only as a tool: it arose from human
  imagination and is capable of creating great beauty on its own.

  In this course you will learn a number of useful formulas, though
  their mastery is not the primary purpose of calculus any more than
  correct spelling is the primary purpose of literature. Our goal is
  to have you learn how to understand calculus conceptually so you can
  build your own approaches to solving practical problems.

\end{description}

\section*{Course Logistics}

\begin{description}
  
\item[Website:] Please see
  \begin{center}
    \url{http://math.colorado.edu/math1300}
  \end{center}
  for homework assignments, a link to WeBWorK, the course schedule,
  and lists of instructors and TAs and LAs.

  I will maintain a webpage with information specific to our section
  at
  \begin{center}
    \url{http://math.colorado.edu/~gern/1300S13}
  \end{center}
  Please see this page for information regarding quizzes, in-class
  worksheets, and announcements, and for a copy of this syllabus.

\item[Calculators:] You will be required to have an electronic device
  for in-class activities which is capable of graphing functions and
  doing numerical integration. Acceptable devices are a calculator
  such as a TI-83 or better, a graphing calculator application for a
  smartphone, software packages such as Maple or Mathematica, and web
  sites such as Wolfram Alpha and Desmos.  \textbf{Absolutely no such
    devices will be allowed on exams or quizzes. Nor will they be
    needed on exams or quizzes.}

\item[Assignments:] As we said, the only effective way to learn
  Calculus is to do lots and lots of problems.  Besides working on
  problems in class every day, there are three types of assignments in
  this course to enhance your skills and understanding.

  \begin{description}
  \item[Recitation Worksheets:] The recitation is every Thursday and is
    supervised by a graduate Teaching Assistant and an undergraduate
    Learning Assistant.  See details below.
  \item[Written Homework:] You will be assigned several conceptual
    problems out of the textbook each week. You are expected to write
    up complete, legible, and logical solutions to these problems,
    which will be graded by your Teaching Assistant. Homework will be
    collected and returned in Thursday recitations. It should be
    written using complete sentences to explain your steps. Late
    homework will absolutely not be accepted under any circumstances,
    but your lowest homework score will be dropped.  Your homework
    must be stapled to be counted for credit.
  \item[Online WeBWorK:] These problems are generally straightforward
    and computational, and you can repeat them any number of times
    until you get the correct answer. The philosophy behind this is
    that instantaneous feedback is more effective than waiting days
    for a grade, and that doing a problem over if it's wrong is better
    than simply seeing the right answer. Because problems are graded
    by a computer, there are occasional technical issues, but we
    believe the trade-off is worthwhile. See details below.
  \end{description}

\item[Midterms:] This course has three midterm exams and a final
  exam. They have already been scheduled. \textbf{You must be able to
    attend all of them.} Calculators and cell phones will not be
  allowed during any portion of any exam. \textbf{Use of any
    electronic device at any time during the exam will be considered
    cheating.}

  Plan your schedule now. There will be \textbf{no makeup exams} given
  under any circumstances. However, \textbf{if you must miss a midterm
    exam, your Final Exam score will be used as your grade for that
    midterm}, which will apply in particular if you cannot attend an
  exam due to emergency, illness, religious observance, or other
  reason.  If you do not miss any midterm exams, we will replace your
  lowest midterm grade by your final exam grade if it is higher.

  \begin{description}
  \item[Midterm 1:] Wednesday, February 6. 5:15\textsc{pm} to
    6:45\textsc{pm}.
  \item[Midterm 2:] Wednesday, March 6. 5:15\textsc{pm} to
    6:45\textsc{pm}.
  \item[Midterm 3:] Wednesday, April 10. 5:15\textsc{pm} to
    6:45\textsc{pm}.
  \end{description}
  Note that midterms {\bf are at night and not in your regular
    classroom.}

\item[Final Exam:] The final exam for the course is
  \textbf{cumulative}.  It is scheduled for:

  \begin{description}
  \item[Saturday, May 4, from 7:30am to 10:00am.]
    Yes, that is 7:30--10:00 in the morning!
  \end{description}

  You may not reschedule this exam even if you have three exams on the
  same day (university policy only allows for the third exam to be
  rescheduled). A makeup will only be given in case of a
  \textbf{documented emergency with documentation from a medical
    professional or the student's dean.}

\item[WeBWorK:] WeBWorK can be accessed through the link on the main
  course webpage, or directly at

  \begin{center}
    \url{https://webwork.colorado.edu/webwork2/Math1300}
  \end{center}

  If you registered for the course \emph{before} Thursday, January 10,
  then you already have a WeBWorK login. In this case, your username
  is the same as your Identikey username, and your password is your
  $9$-digit student ID number (without dashes). If you registered for
  the course \emph{on or after} Thursday, January 10, then you will
  need to email your Identikey username and your $9$-digit student ID
  number to your instructor, and your instructor will add you to the
  WeBWorK system.

  Occasionally there are technical difficulties with the system. You
  may email your instructor to ask about a problem, but but
  \textbf{only} under the following conditions:
  \begin{enumerate}
  \item Email will only be responded to during normal hours (8 am to 6
    pm) and not after any assignment's deadline.
  \item Only one message per problem set will be answered.
  \item You must clearly explain your problem and the steps you have
    taken.
  \end{enumerate}
  Your lowest three WeBWorK assignment scores will be dropped.

\item[Quizzes:] There will be a 10 minute quiz every Tuesday (except in
  midterm weeks and the first week).  Your two lowest quiz scores will
  be dropped. \textbf{Calculators will not be allowed on quizzes.}

\item[Recitation Worksheets:] These projects will be distributed every
  Thursday, and you will work on them in small groups with several of
  your classmates. (The group to which you are assigned will change
  frequently.)  A graduate Teaching Assistant (TA) and undergraduate
  Learning Assistant (LA) will be present during recitations to
  facilitate your work on the projects, but the goal is for you (and
  your group-mates) to \textbf{work through, and complete these
    projects on your own} as much as possible.

  Your LA and TA will be making sure that you participate in your
  group's explorations and discoveries.  You will be graded on your
  participation, so {\it participate}.

  Each worksheet will be graded on a 5 point scale --- you will
  receive 2 points for attending the entire recitation, another 2
  points for working on the project during the entire recitation, and
  1 additional point for working with the other students in your
  assigned group.

  Missed worksheets cannot be made up: if you miss a Thursday
  recitation, you will receive a zero for that worksheet. However,
  your lowest recitation grade will be dropped.

\item[Grades:] The grade distribution will be calculated based on the
  following rubric:
  \begin{description}
  \item[Midterms] (15\% each)
  \item[Final Exam] (20\%)
  \item[WeBWorK] (10\%)
  \item[Written homework] (10\%)
  \item[Recitation worksheets] (5\%)  
  \item[Quizzes] (10\%)
  \end{description}

\item[Undergraduate Mathematics Resource Center:] You may seek
  assistance with your math questions in the Undergraduate Mathematics
  Resource Center (Help Lab) in Math 175.  The Center will open
  Monday, January 14 at 9 am, and thereafter will be open (on school
  days only) Monday--Thursday 9 am--5 pm, and Friday 9 am--2 pm.  (The
  Center closes for the semester at 2 PM on Friday, May 3.) You may
  request help from any lab tutor.
\end{description}

\section*{University Policies and Standards}

\begin{description}
\item[Classroom behavior:] Students and faculty each have
  responsibility for maintaining an appropriate learning
  environment. Those who fail to adhere to such behavioral standards
  are subject to discipline.  Disruptive behavior, as applied to the
  academic setting, means behavior that a reasonable faculty member
  would view as interfering with normal academic functions. Examples
  include, but are not limited to: persistently speaking without being
  recognized or interrupting other speakers; behavior that distracts
  the class from the subject matter or discussion; or in extreme
  cases, physical threats, harassing behavior or personal insults, or
  refusal to comply with faculty direction. Instructors may ask
  disruptive students to leave the classroom.  See policies at
  \begin{center}
    \url{http://www.colorado.edu/policies/classbehavior.html}
  \end{center}
  and at
  \begin{center} \url{http://www.colorado.edu/studentaffairs/judicialaffairs/code.html#student_code}
  \end{center}

\item[Disability accommodations:] If you qualify for accommodations
  because of a disability, please submit to your professor a letter
  from Disability Services by the end of the second week of the course
  so that your needs can be addressed. Disability Services determines
  accommodations based on documented disabilities. Contact Disability
  Services at 303-492-8671 or by e-mail at
  \href{mailto:dsinfo@colorado.edu}{dsinfo@colorado.edu}.

  If you have a temporary medical condition or injury, see Temporary
  Medical Conditions: Injuries, Surgeries, and Illnesses guidelines
  under Quick Links at Disability Services website and discuss your
  needs with your professor. See
  \begin{center}
    \url{http://www.colorado.edu/disabilityservices/}
  \end{center}

\item[Religious observances:] Campus policy regarding religious
  observances requires that faculty make every effort to deal
  reasonably and fairly with all students who, because of religious
  obligations, have conflicts with scheduled exams, assignments or
  required attendance.  Please notify us within the first two weeks of
  the course if you must miss a class, exam, or assignment because of
  a religious observance.  See full details at
  \begin{center}
    \url{http://www.colorado.edu/policies/fac_relig.html}
  \end{center}

\item[Respect for diversity:] Professional courtesy and sensitivity
  are especially important with respect to individuals and topics
  dealing with differences of race, color, culture, religion, creed,
  politics, veteran's status, sexual orientation, gender, gender
  identity and gender expression, age, disability, and nationalities.
  Class rosters are provided to the instructor with the student's
  legal name. I will gladly honor your request to address you by an
  alternate name or gender pronoun. Please advise me of this
  preference early in the semester so that I may make appropriate
  changes to my records.

\item[Discrimination and harassment:] The University of Colorado
  Boulder (CU Boulder) is committed to maintaining a positive
  learning, working, and living environment. The University of
  Colorado does not discriminate on the basis of race, color, national
  origin, sex, age, disability, creed, religion, sexual orientation,
  or veteran status in admission and access to, and treatment and
  employment in, its educational programs and activities. (Regent Law,
  Article 10, amended 11/8/2001).  CU Boulder will not tolerate acts
  of discrimination or harassment based upon Protected Classes or
  related retaliation against or by any employee or student. For
  purposes of this CU-Boulder policy, ``Protected Classes'' refers to
  race, color, national origin, sex, pregnancy, age, disability,
  creed, religion, sexual orientation, gender identity, gender
  expression, or veteran status.  Individuals who believe they have
  been discriminated against should contact the Office of
  Discrimination and Harassment (ODH) at 303-492-2127 or the Office of
  Student Conduct (OSC) at 303-492-5550.  Information about the ODH,
  the above referenced policies, and the campus resources available to
  assist individuals regarding discrimination or harassment can be
  obtained at
  \begin{center}
    \url{http://www.colorado.edu/odh}
  \end{center}

\item[Honor code:] All students of the University of Colorado Boulder
  are responsible for knowing and adhering to the academic integrity
  policy of this institution.  Violations of this policy may include:
  cheating, plagiarism, aid of academic dishonesty, fabrication,
  lying, bribery, and threatening behavior.  All incidents of academic
  misconduct shall be reported to the Honor Code Council
  (\href{mailto:honor@colorado.edu}{honor@colorado.edu};
  303-735-2273). Students who are found to be in violation of the
  academic integrity policy will be subject to both academic sanctions
  from the faculty member and non-academic sanctions (including but
  not limited to university probation, suspension, or
  expulsion). Other information on the Honor Code can be found at
  \begin{center}
    \url{http://honorcode.colorado.edu}
  \end{center}
\end{description}
\end{document}
