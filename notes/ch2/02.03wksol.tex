\documentclass[11pt]{article} 
\usepackage{calc}
\usepackage[margin={1in,1in}]{geometry} 
\usepackage[hwkhandout]{hwk}
\usepackage[pdftitle={2.3 Worksheet},colorlinks=true,urlcolor=blue]{hyperref}
\usepackage{graphicx, tikz}

\renewcommand{\theclass}{\textsc{math} 1300-013: Calculus I}
\renewcommand{\theauthor}{Tyson Gern}
\renewcommand{\theassignment}{The Derivative Function}
\renewcommand{\dateinfo}{spring 2013 2012}

\def\ww{WeBWorK }

\begin{document}
\drawtitle
\begin{enumerate}

\item Find the derivative of $f(x)=2x^2-3x+1$ using the limit
  definition of derivative.

  \vfill
  {\color{blue}
    \begin{align*}
      f'(x) &= \lim_{h\to 0} \frac{2(x+h)^2-3(x+h)+1-(2x^2-3x+1)}{h}\\
      &= \lim_{h\to 0} \frac{2x^2+4xh+2h^2-3x-3h+1-2x^2+3x-1}{h}\\
      &= \lim_{h\to 0} \frac{4xh+2h^2-3h}{h}\\
      &= \lim_{h\to 0} \frac{h(4x+2h-3)}{h}\\
      &= \lim_{h\to 0} 4x+2h-3\\
      &= 4x-3
    \end{align*}
  }
  \vfill

\item Let $f(x)$ be a continuous function.  Draw a possible graph of
  $f(x)$ that satisfies \textbf{all} of the following conditions.
  \begin{itemize}
  \item $f(0)=0$
  \item $f'(x)\leq 0$ when $x<3$, and $f'(x)\geq 0$ for $x>3$.
  \item $\displaystyle\lim_{x\rightarrow\infty} f(x)=2$.
  \end{itemize}
  \begin{center}
    {\color{blue} Answers will vary.}
    \begin{tikzpicture}[scale = 1.2]
      \draw[<->] (-1,0) -- (7.5,0);
      \draw[<->] (0,-2.5) -- (0,3);
      \draw (3, .12) -- (3, -.12) node[below] {3};
      \draw (.12, 2) -- (-.12, 2) node[left] {2};
      \draw[thick, blue] plot [smooth,tension=.4] coordinates{(-1,1)
        (0,0) (3,-1) (4, 0) (5,1.5) (6,1.8) (7,1.9) (8,1.9)};
    \end{tikzpicture}
  \end{center}


\newpage

\item Suppose that the function $h(t)$ describes the temperature of a
  cold glass of water $t$ seconds after placing the glass in a $75^\circ$ room.
  
  \begin{enumerate}
  \item What is the sign of $h'(t)$?
    \vfill
    {\color{blue} $h'(t)$ is positive.}
    \vfill
  \item What are the units of $h'(t)$?
    \vfill
    {\color{blue} The units for $h'(t)$ are degrees per second.}
    \vfill
  \item What is $\displaystyle\lim_{t\to\infty} h(t)$?
    \vfill
    {\color{blue} $\displaystyle\lim_{t\to\infty}h(t) = 75$ since the
      temperature will eventually approach 75 degrees.}
    \vfill
  \item On the axes below, draw graphs of $h(t)$ and $h'(t)$.
  \begin{center}
    \begin{tikzpicture}
      \draw[->] (0,0) -- (10,0);
      \draw[<->] (0,-1.5) -- (0,5.5);
      \draw[blue] (.1,1) -- (-.1,1) node[left] {35};
      \draw[blue] (.1,5) -- (-.1,5) node[left] {75};
      \draw[thick, domain=0:10, ->, blue] plot[samples=200]
      function{4*(1-2**(-x))+1} node[right] {$h(t)$};
      \draw[thick, domain=0:10, ->, blue] plot[samples=200]
      function{2.772*(2**(-x))} node[right] {$h'(t)$};
    \end{tikzpicture}
  \end{center}
  \end{enumerate}

\newpage

\item On the axes given below draw the derivatives of the following
  functions.

  \begin{center}
    \begin{tabular}{cc}
      \begin{tikzpicture}[xscale = 7/8, yscale = 5/55]
        \draw[<->] (-4,0) -- (4,0);
        \draw[<->] (0,-35) -- (0,20);
        
        \draw[thick, domain=-4:4, <->] plot[samples=200]
        function{(x+3)*(x-1)*(x-3)};
        \draw[thick, domain=-2.5:3.5, <->, blue] plot[samples=200]
        function{3*x**2-2*x-9};
      \end{tikzpicture}
      &
      \begin{tikzpicture}[xscale = 7/13, yscale = 5/3]
        \draw[<->] (-6.5,0) -- (6.5,0);
        \draw[<->] (0,-1.5) -- (0,1.5);
        
        \draw[thick, domain=-6.5:6.5, <->] plot[samples=200]
        function{sin(x)};
        \draw[thick, domain=-6.5:6.5, <->, blue] plot[samples=200]
        function{cos(x)};
      \end{tikzpicture}
      \\
      \\
      \begin{tikzpicture}[xscale = 7/8, yscale = 5/17]
        \draw[<->] (-4,0) -- (4,0);
        \draw[<->] (0,-1) -- (0,16);
        
        \draw[thick, domain=-4:4, <->] plot[samples=200]
        function{2**x};
        \draw[thick, domain=-4:4, <->, blue] plot[samples=200]
        function{.693*2**x};
      \end{tikzpicture}
      &
      \begin{tikzpicture}[xscale = 7/6, yscale = 5/3]
        \draw[<->] (-3,0) -- (3,0);
        \draw[<->] (0,-1.5) -- (0,1.5);
        
        \draw[thick, domain=-3:3, <->] plot[samples=200]
        function{1/(x**2+1)};
        \draw[thick, domain=-3:3, <->, blue] plot[samples=200]
        function{(-2*x)/(x**4+2*x**2+1)};
      \end{tikzpicture}
      \\
    \end{tabular}
  \end{center}
  
\item The temperature on a given day in Boulder is modeled by a
  sinusoidal curve.  There is a minimum temperature of $40^\circ$ at
  6\textsc{am} and a maximum temperature of $70^\circ$ at
  6\textsc{pm}.
  \begin{enumerate}
  \item Find a function $f(t)$ that gives the temperature, where $t$
    is measured in hours after midnight, and $f(t)$ is degrees.
    \vfill
    {\color{blue}
      \[
      f(t) = -15\sin\left(\frac{\pi}{12} t\right)+55.
      \]
    }
    \vfill
    \newpage

  \item Graph the function $f(t)$ along with its derivative $f'(t)$.
    \begin{center}
      \begin{tikzpicture}[yscale = 7/80, xscale = 12/24]
        \draw[->] (0,0) -- (24,0);
        \draw[<->] (0,-10) -- (0,72);
        
        \draw[thick, domain=0:24, <->, blue] plot[samples=200]
        function{-15*sin(.2618*x)+55} node[right] {$f(t)$};
        \draw[thick, domain=0:24, <->, blue] plot[samples=200]
        function{(.2618)*(-15)*cos(.2618*x)} node[right] {$f'(t)$};
      \end{tikzpicture}
    \end{center}
  \item When is the temperature growing the fastest?  When is the
    temperature growing the slowest?
    \vfill
    {\color{blue} 
      The temperature is changing that fastest at noon, and is
      changing the slowest at 6\textsc{am} and 6\textsc{pm}.
    }
    \vfill
  \end{enumerate}

\item Using shortcuts, find the derivatives of the following
  functions.
  \begin{enumerate}
  \item $f(x)=x^\pi$
    \vfill
    {\color{blue}
      \[
      f'(x) = \pi x^{\pi-1}
      \]
    }
    \vfill
    \newpage
  \item $g(x)=x^{-e}$
    \vfill
    {\color{blue}
      \[
      g'(x) = (-e)x^{-e-1}
      \]
    }
    \vfill
  \item $h(x)=\ln(4)^3+7x$
    \vfill
    {\color{blue}
      \[
      h'(x) = 7
      \]
    }
    \vfill
  \item $\displaystyle P(x)=30,000x(4)^7-4000x+25^{\frac{1}{3}}$
    \vfill
    {\color{blue}
      \[
      P'(x) = 30000\cdot4^7 - 4000
      \]
    }
    \vfill
    \newpage
  \item $\displaystyle
    f(x)=23,879\left(\frac{\pi}{\sqrt{23}}\right)-7^{\ln(2300)}$
    \vfill
    {\color{blue}
      \[
      f'(x) = 0 \text{ since $f(x)$ is constant!}
      \]
    }
    \vfill
  \item $g(x) = \sqrt{x}$.
    \vfill
    {\color{blue}
      \begin{align*}
        g(x) &= x^{\frac{1}{2}}\\
        \intertext{so}
        g'(x) &= \frac{1}{2}x^{-\frac{1}{2}}
      \end{align*}
    }
    \vfill
  \item $h(x) = \displaystyle\sqrt[7]{\frac{x^2\cdot x^{-3}}{x^5}}$
    \vfill
    {\color{blue}
      \begin{align*}
        h(x) &= x^{-\frac{6}{7}}\\
        \intertext{so}
        h'(x) &= -\frac{6}{7}x^{-\frac{13}{7}}
      \end{align*}
    }
    \vfill
  \end{enumerate}






\end{enumerate}

\end{document}
