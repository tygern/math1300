\documentclass[10pt]{article} 
\usepackage{calc}
\usepackage[margin={1in,1in}]{geometry} 
\usepackage[hwkhandout]{hwk}
\usepackage[pdftitle={2.4 Worksheet},colorlinks=true,urlcolor=blue]{hyperref}
\usepackage{graphicx, tikz}

\renewcommand{\theclass}{\textsc{math} 1300-013: Calculus I}
\renewcommand{\theauthor}{Tyson Gern}
\renewcommand{\theassignment}{Interpreting the Derivative}
\renewcommand{\dateinfo}{Fall 2012}

\def\ww{WeBWorK }

\begin{document}
\drawtitle
\begin{enumerate}
\item Let $g(v)$ be the fuel efficiency, in miles per gallon, of a car
  traveling at a velocity of $v$ miles per hour. Answer the following
  questions using complete sentences.
  \begin{enumerate}
  \item What are the units of $g(60)$?
    
    \vfill

    {\color{blue} The units are miles per gallon.}
    
    \vfill
    
  \item What is the practical meaning of the statement $g(60) = 30$?
    
    \vfill

    {\color{blue} When you are traveling at 60 miles per hour your
      fuel efficiency is 30 miles per gallon.}

    \vfill
    
  \item What are the units of $g'(60)$?
    
    \vfill

    {\color{blue} The units are $\dfrac{\text{miles per
          gallon}}{\text{miles per hour}} =
      \dfrac{\text{hours}}{\text{gallon}}$.}

    \vfill
    
  \item What is the practical meaning of the statement $g'(60) = -1.2$?
    
    \vfill

    {\color{blue} If you increase your speed from 60 to 61 miles per
      hour your fuel efficiency will drop by $1.2$ miles per gallon.}

    \vfill
    
  \end{enumerate}

  \newpage

\item A company's revenue from car sales, $C$ (in thousands of
  dollars), is a function of advertising expenditure, $a$ (in
  thousands of dollars), so $C = f(a)$. Answer the following questions
  using complete sentences.
  \begin{enumerate}
  \item What does the company hope is true about the sign of $f'(a)$?
    
    \vfill

    {\color{blue} The company hopes that $f'(a)$ is positive because
      it wants revenue to rise as advertising expenditure rises.}

    \vfill
    
  \item What does the statement $f'(100) = 2$ mean in practical terms?
    How about $f'(100) = 0.5$?
    
    \vfill

    {\color{blue} The statement $f'(100) = 2$ means that if you
      increase your advertising expenditure from \$100,000 to
      \$101,000 your revenue will rise by \$2,000. The statement
      $f'(100) = 0.5$ means that if you increase your advertising
      expenditure from \$100,000 to \$101,000 your revenue will rise
      by \$500.}

    \vfill
    
  \item Suppose the company plans to spend about \$100,000 on
    advertising. If $f'(100)=2$, should the company spend more or less
    than \$100,000 on advertising? What if $f'(100) = 0.5$?
    
    \vfill

    {\color{blue} If $f'(100) = 2$ the company should increase
      advertising expenditure, since a \$1,000 increase in expenditure
      would correspond to a \$2,000 rise in revenue. If $f'(100) =
      0.5$ the company should decrease advertising expenditure, since
      a \$1,000 increase in expenditure would only correspond to a
      \$500 rise in revenue, while a \$1,000 decrease in expenditure
      would only correspond to a \$500 fall in revenue.}

    \vfill
    
  \end{enumerate}
  


\end{enumerate}

\end{document}
