\documentclass[11pt]{article} 
\usepackage{calc}
\usepackage[margin={1in,1in}]{geometry} 
\usepackage[hwkhandout]{hwk}
\usepackage[pdftitle={Calc 1
  Information},colorlinks=true,urlcolor=blue]{hyperref}
\usepackage{tikz, array}

\renewcommand{\theclass}{\textsc{math}1300: calculus I}
\renewcommand{\theauthor}{Tyson Gern}
\renewcommand{\theassignment}{The Derivative at a Point}
\renewcommand{\dateinfo}{section 2.2}

\begin{document}
\drawtitle

\begin{enumerate}
\item Let $f(x) = 3x^2-8x+3$.  Find $f'(2)$.
  \vfill
  {\color{blue}
    Using the definition of the derivative we obtain
    \begin{align*}
      f'(2) &= \lim_{h\to 0} \frac{f(2+h)-f(2)}{h}\\
      &= \lim_{h\to 0} \frac{3(2+h)^2-8(2+h)+3 - (-1)}{h}\\
      &= \lim_{h\to 0} \frac{3h^2+4h}{h}\\
      &= \lim_{h\to 0} \frac{h(3h+4)}{h}\\
      &= \lim_{h\to 0} 3h+4\\
      &= 4
    \end{align*}
  }
  \vfill
\item Let $g(x)$ be given by the graph below.  Match the derivatives
  in the table with the points $a$, $b$, $c$, $d$, and $e$.

  \begin{center}
    \begin{tabular}{m{3.5in}m{1in}}
      \begin{tikzpicture}[scale = 1.5]
        \draw[<->] (-1,0) -- (3.75,0);
        \draw[<->] (0,-2) -- (0,3);
        \draw[-] (.27,.1) -- (.27,-.1) node[below] {$a$};
        \draw[-] (.45,.1) -- (.45,-.1) node[below] {$b$};
        \draw[-] (2,.1) -- (2,-.1) node[below] {$d$};
        \draw[-] (2.5,.1) -- (2.5,-.1) node[below] {$e$};
        \draw[-] (1,.1) -- (1,-.1) node[below] {$c$};
        
        \draw[thick, domain=-.5:3.25, <->] plot[samples=200]
        function{-x*(x-1)*(x-3)} node[right]{$g(x)$};
      \end{tikzpicture}
      &
      \begin{tabular}{c|c}
        $x$ & $g'(x)$\\
        \hline
        {\color{blue} $b$}&$0$\\
        {\color{blue} $d$}&$1$\\
        {\color{blue} $c$}&$2$\\
        {\color{blue} $a$}&$-0.5$\\
        {\color{blue} $e$}&$-2$
      \end{tabular}
    \end{tabular}
  \end{center}

  \newpage

\item Let $h(x) = \sqrt{x}$.
  \begin{enumerate}
  \item Find $h'(4)$.
  \vfill
  {\color{blue}
    Using the definition of the derivative we obtain
    \begin{align*}
      h'(4) &= \lim_{h\to 0} \frac{h(4+h)-h(4)}{h}\\
      &= \lim_{h\to 0} \frac{\sqrt{4+h}-2}{h}\\
      &= \lim_{h\to 0} \frac{\sqrt{4+h}-2}{h}\cdot\frac{\sqrt{4+h}+2}{\sqrt{4+h}+2}\\
      &= \lim_{h\to 0} \frac{4+h-4}{h\left(\sqrt{4+h}+2\right)}\\
      &= \lim_{h\to 0} \frac{h}{h\left(\sqrt{4+h}+2\right)}\\
      &= \lim_{h\to 0} \frac{1}{\sqrt{4+h}+2}\\
      &= \frac{1}{4}
    \end{align*}
  }
    \vfill
  \item Find the equation of the tangent line to $h(x)$ at $x=4$ and
    use this to estimate $\sqrt{4.1}$.
    \vfill

    {\color{blue}
      We know the slope of the tangent line to $h(x)$ at $x=4$ is
      $h'(4)$, so the tangent line if of the form $y= \frac{1}{4}x+b$.
      Then the tangent line must go through the point
      $(4,h(4))=(4,2)$, so
      \begin{align*}
        2 &= \frac{1}{4}\cdot 4 + b\\
        2 &= 1 + b\\
        b &= 1.
      \end{align*}
      Then the tangent line is given by $y=\frac{1}{4}x+1$, so
      \[
      \sqrt{4.1}=f(4.1)\approx \frac{1}{4}\cdot 4.1 + 1 = 2.025.
      \]
    }

    \vfill
  \end{enumerate}
  
  \newpage

\item Estimate the derivative of $f(x) = x^x$ at $x=2$.
  \vfill
  {\color{blue}
    We know that
    \[
    f'(2) = \lim_{h\to 0} \frac{f(2+h)-f(2)}{h} = \lim_{h\to 0}
    \frac{(2+h)^{2+h} - 4}{h}.
    \]
    We do not have the tools to solve this limit algebraically, so we
    can estimate its value by plugging in very small numbers for $h$.
    A reasonable is
    \[
    f'(2) \approx 6.77259.
    \]
  }
  \vfill

\item Suppose that $f(x)$ is a function with $f(-5) = 7$ and $f'(-5) =
  6$.  Estimate $f(-5.5)$.
  \vfill
  {\color{blue}

    We will estimate $f(-5.5)$ by finding the tangent line to $f(x)$
    at $x=5$, then plugging in $x=-5.5$. We know that the tangent line
    has a slope of $6$ and goes through the point $(-5,7)$, so
    \begin{align*}
      7 &= 6\cdot (-5) + b\\
      b &= 37.
    \end{align*}
    Then the equation for the tangent line is given by $y = 6x+37$, so
    \[
    f(-5.5)\approx 6\cdot (-5.5) + 37 = 4.
    \]

  }
  \vfill

  \newpage

\item Find the derivative of $f(x) = \frac{1}{x^2}$ at $x=-1$.
  \vfill
  {\color{blue}

    Using the definition of the derivative we obtain
    \begin{align*}
      f'(-1) &= \lim_{h\to 0} \frac{f(-1+h)-f(-1)}{h}\\
      &= \lim_{h\to 0} \frac{\frac{1}{(-1+h)^2}-1}{h}\\
      &= \lim_{h\to 0} \frac{\frac{1}{(-1+h)^2}-\frac{(-1+h)^2}{(-1+h)^2}}{h}\\
      &= \lim_{h\to 0} \frac{\frac{2h-h^2}{(-1+h)^2}}{h}\\
      &= \lim_{h\to 0} \frac{\frac{h(2-h)}{(-1+h)^2}}{h}\\
      &= \lim_{h\to 0} \frac{(2-h)}{(-1+h)^2}\\
      &= 2
    \end{align*}

  }
  \vfill

\item On the axes below, draw a graph of a population function $P(t)$
  such that the average population growth between $t=2$ and $t=7$ is
  equal to the instantaneous population growth at $t=6$.  Justify your
  answer.
  \begin{center}
    \begin{tikzpicture}[scale = 1]
      \draw[->] (0,0) -- (8,0) node[below] {$t$};
      \draw[->] (0,0) -- (0,6.5) node[left] {$P(t)$};
    \end{tikzpicture}
  \end{center}
  {\color{blue}
    Answers will vary.  The key is that the slope of the tangent line
    to $P(t)$ at $x=6$ must be equal to the slope of the secant line
    between $t=2$ and $t=7$.
  }
\end{enumerate}

\end{document}
