\documentclass[11pt]{article} 
\usepackage{calc}
\usepackage[margin={1in,1in}]{geometry} 
\usepackage[hwkhandout]{hwk}
\usepackage[pdftitle={2.5 Worksheet},colorlinks=true,urlcolor=blue]{hyperref}
\usepackage{tikz,array}

\renewcommand{\theclass}{\textsc{math} 1300-003: Calculus I}
\renewcommand{\theauthor}{Tyson Gern}
\renewcommand{\theassignment}{The Second Derivative}
\renewcommand{\dateinfo}{Spring 2013}

\def\ww{WeBWorK }

\begin{document}
\drawtitle
\begin{enumerate}

\item Find the second derivative of $f(x)=2x^2-x$. You may
  \textbf{only} use shortcuts for power, linear, and constant
  functions.
  \vfill
  {\color{blue}
    
    Using the limit definition of the derivative we see that
    \begin{align*}
      f'(x) &= \lim_{h\to 0} \frac{2(x + h)^2 - (x + h) - (2x^2 - x)}{h}\\
      &= \lim_{h\to 0} \frac{2x^2+4xh+2h^2 - x - h - 2x^2 + x}{h}\\
      &= \lim_{h\to 0} \frac{4xh+2h^2 - h)}{h}\\
      &= \lim_{h\to 0} 4x+2h - 1\\
      &= 4x - 1.\\
    \end{align*}
    
    Then since $f'(x) = 4x - 1$ is a line we have $f''(x) = 4$.

  }
  \vfill

  \newpage
  
\item On the axes given below draw the second derivatives of the
  following functions.
  \begin{center}
    \begin{tabular}{cc}
      \begin{tikzpicture}[xscale = 7/8, yscale = 5/55]
        \draw[<->] (-4,0) -- (4,0);
        \draw[<->] (0,-35) -- (0,20);
        
        \draw[thick, domain=-4:4, <->] plot[samples=200]
        function{(x+3)*(x-1)*(x-3)};
        \draw[color=blue, thick, domain=-3:3.5, <->] plot[samples=200]
        function{3*x**2-2*x-9};
      \end{tikzpicture}
      &
      \begin{tikzpicture}[xscale = 7/13, yscale = 5/3]
        \draw[<->] (-6.5,0) -- (6.5,0);
        \draw[<->] (0,-1.5) -- (0,1.5);
        
        \draw[thick, domain=-6.5:6.5, <->] plot[samples=200]
        function{sin(x)};
        \draw[color=blue, thick, domain=-6.5:6.5, <->] plot[samples=200]
        function{-sin(x)};
      \end{tikzpicture}
      \\
      \begin{tikzpicture}[xscale = 7/8, yscale = 5/17]
        \draw[<->] (-4,0) -- (4,0);
        \draw[<->] (0,-1) -- (0,16);
        
        \draw[thick, domain=-4:4, <->] plot[samples=200]
        function{2**x};
        \draw[color=blue, domain=-4:4, <->] plot[samples=200]
        function{(.69**2)*(2**x)};
      \end{tikzpicture}
      &
      \begin{tikzpicture}[xscale = 7/6, yscale = 5/3]
        \draw[<->] (-3,0) -- (3,0);
        \draw[<->] (0,-1.5) -- (0,1.5);
        
        \draw[thick, domain=-3:3, <->] plot[samples=200]
        function{1/(x**2+1)};
        \draw[color=blue, thick, domain=-3:3, <->] plot[samples=200]
        function{(8*x**2)/(x**2+1)**3-(2)/(x**2+1)**2};
      \end{tikzpicture}
      \\
    \end{tabular}
  \end{center}

\item Let $f(x)$ be a continuous function.  Draw a possible graph of
  $f(x)$ that satisfies each of the following conditions.
 	\begin{itemize}
 	\item $f(0)=2$
 	\item $f'(x)> 0$ when $x<-1$, and $f'(x)< 0$ for $x>-1$.
 	\item $f''(x)<0$ when $-3 < x < 3$.
 	\item $f''(x)>0$ otherwise.
 	\end{itemize}
  \begin{center}
    \begin{tikzpicture}[scale = 1.2]
      \def\xmin{-5.5}\def\xmax{5.5} \def\ymin{-2}\def\ymax{3}
      \draw (-5,.1) -- (-5,-.1) node[below] {$-5$};
      \draw ( 5,.1) -- ( 5,-.1) node[below]  {$5$};
      \draw (.1, 2.75) -- (-.1, 2.75) node[left]   {$3$};
      \draw (.1,-1.75) -- (-.1,-1.75) node[left]  {$-2$};
      \draw[<->, thick] (\xmin,0) -- (\xmax,0); \draw[<->, thick] (0,\ymin) --
      (0,\ymax);
      \draw[color=blue, thick, domain=-5:5, <->] plot[samples=200]
      function{3/((.3*(x+1))**2+1)-1};
    \end{tikzpicture}
  \end{center}

  \newpage

\item Let $P(t)$ represent the price of a share of stock of a
  corporation at time $t$. What do each of the following statements
  tell us about the signs of the first and second derivatives of
  $P(t)$? Draw a graph to explain each situation.
  \begin{enumerate}
  \item ``The price of the stock is rising faster and faster.''
    \vfill
    {\color{blue}
      \begin{center}
        \begin{tikzpicture}[xscale = 14/8, yscale = 5/17]
          \draw[->] (0,0) -- (4,0);
          \draw[->] (0,0) -- (0,17);
          
          \draw[thick, domain=0:4, ->] plot[samples=200]
          function{2**x};
        \end{tikzpicture}
      \end{center}
    }
    \vfill
  \item ``The price of the stock is close to bottoming out.''
    \vfill
    {\color{blue}
      \begin{center}
        \begin{tikzpicture}[xscale = 14/8, yscale = 5/17]
          \draw[->] (0,0) -- (4,0);
          \draw[->] (0,0) -- (0,17);
          
          \draw[thick, domain=0:4, ->] plot[samples=200]
          function{2**(4-x)};
        \end{tikzpicture}
      \end{center}
    }
    \vfill
  \end{enumerate}

\newpage

\item ``Winning the war on poverty'' has been described cynically as
  slowing the rate at which people are slipping below the poverty
  line. Assuming that this is happening:
  \begin{enumerate}
  \item Graph the total number of people in poverty against time.
    \vfill
    {\color{blue}
      \begin{center}
        \begin{tikzpicture}[xscale = 14/8, yscale = 5/17]
          \draw[->] (0,0) -- (4,0);
          \draw[->] (0,0) -- (0,20);
          
          \draw[thick, domain=0:4, ->] plot[samples=200]
          function{-(2**(4-x)) + 18};
        \end{tikzpicture}
      \end{center}
    }
    \vfill
  \item If $N$ is the number of people below the poverty line at time
    $t$, what are the signs of $\dfrac{dN}{dt}$ and
    $\dfrac{d^2N}{dt^2}$? Explain.
    \vfill
    {\color{blue}

      Since $N(t)$ is an increasing, concave down function we have
      $\dfrac{dN}{dt} > 0$ and $\dfrac{d^2N}{dt^2} < 0$.

    }
    \vfill
  \end{enumerate}

\newpage

\item Suppose that $f(x)$ is a function with the following properties:
  \begin{itemize}
  \item $f$ is decreasing
  \item $f$ is concave down
  \item $f(2)=4$
  \item $f'(2)=-\frac{1}{2}$
  \end{itemize}

  \begin{enumerate}
  \item Find the tangent line to $f$ at $x=2$.
    \vfill
    {\color{blue}
      \[
      y = -\frac{1}{2} x + 5
      \]
    }
    \vfill
  \item Draw a possible graph of $f$ and the tangent line from part
    (a).
    \vfill
    {\color{blue}
      \begin{center}
        \begin{tikzpicture}[xscale = 1, yscale = 1]
          \draw[<->] (-1,0) -- (12,0);
          \draw[<->] (0,-1) -- (0,8);
          \draw (2,.15) -- (2,-.15) node[below] {2};
          \draw (.15,4) -- (-.15,4) node[left] {4};
          
          \draw[thick, domain=-1:11, <->] plot[samples=200]
          function{-.5*x + 5};
          \draw[thick, domain=.25:4.5, <->] plot[samples=200]
          function{-.125*x**2 + 4.5};
        \end{tikzpicture}
      \end{center}
    }
    \vfill
  \item How many zeros does $f(x)$ have on the following intervals?
   \[
     \begin{array}{|c|c|}
       \hline
       \mbox{interval} & \mbox{zeros}\\
       \hline
       (-\infty,0)&{\color{blue} 0}\\
       \hline
       (0,2)&{\color{blue} 0}\\
       \hline
       (2,10)&{\color{blue} 1}\\
       \hline
       (10,\infty)&{\color{blue} 0}\\
       \hline
     \end{array}
   \]
   \end{enumerate}
  \newpage
\item Suppose that $g(x)$ is a function with the following
   properties:
  \begin{itemize}
  \item $g(3)=5$
  \item $g'(3)=-2$
  \item $g''(x)>0$ for all $x$
  \end{itemize}

  \begin{enumerate}
  \item Find the tangent line to $g$ at $x=3$.
    \vfill
    {\color{blue}
      \[
      y = -2x + 11
      \]
    }
    \vfill
  \item Draw a possible graph of $g$ and the tangent line from part
    (a).
    \vfill
    {\color{blue}
      \begin{center}
        \begin{tikzpicture}[xscale = 1, yscale = 1/2]
          \draw[<->] (-1,0) -- (6,0);
          \draw[<->] (0,-1) -- (0,12);
          \draw (3,.3) -- (3,-.3) node[below] {3};
          \draw (.15,5) -- (-.15,5) node[left] {5};
          
          \draw[thick, domain=-.25:5, <->] plot[samples=200]
          function{-2*x + 11};
          \draw[thick, domain=1:5, <->] plot[samples=200]
          function{.5*(x-5)**2 + 3};
        \end{tikzpicture}
      \end{center}
    }
    \vfill
  \item Which of the following $y$-values are possible for $g(0)$ and $g(4)$?
    \begin{center}
      \begin{tabular}{|c|c|c|}
        \hline
        $y$-value & possible for $g(0)$ & possible for $g(4)$\\
        \hline
        $0$&{\color{blue} no}&{\color{blue} no}\\
        \hline
        $3$&{\color{blue} no}&{\color{blue} no}\\
        \hline
        $7$&{\color{blue} no}&{\color{blue} yes}\\
        \hline
        $11$&{\color{blue} no}&{\color{blue} yes}\\
        \hline
        $1000$&{\color{blue} yes}&{\color{blue} yes}\\
        \hline
      \end{tabular}
    \end{center}
  \end{enumerate}

\end{enumerate}
\end{document}
