\documentclass[11pt]{article} 
\usepackage{calc}
\usepackage[margin={1in,1in}]{geometry} 
\usepackage[hwkhandout]{hwk}
\usepackage[pdftitle={Calc 1
  Information},colorlinks=true,urlcolor=blue]{hyperref}

\renewcommand{\theclass}{\textsc{math}1300: calculus I}
\renewcommand{\theauthor}{Tyson Gern}
\renewcommand{\theassignment}{The Derivative Function}
\renewcommand{\dateinfo}{section 2.3}

\begin{document}
\drawtitle

\section*{Basics}
\begin{description}
\item[Review] Derivative at a point.
\item[Function] Now do for arbitrary $x$.
\item[Define] $f'(x)$
\end{description}

\section*{Graphs and Tables}
\begin{description}
\item[Example] $f(x)=x^2+1$, look a graph
\item[Increasing] Sign of derivative and increasing/decreasing
\item[Draw] Random graph and graph of derivative.
\item[Table] Find $f'(x)$:
  \begin{center}
    \begin{tabular}{|c|c|c|c|c|c|}
\hline
$x$    & -2 & -1 & 0 & 1 &  2\\
\hline
$f(x)$ &  3 &  1 & 2 & 1 & -1\\
\hline
    \end{tabular}
  \end{center}
\end{description}

\section*{Shortcuts}
\begin{description}
\item[Constant] Graph
\item[Linear] Slope
\item[Power] Do for $x^2$ and $x^3$ first.  If time expand $(x+h)^n$.
  Works if $n\neq 0$
\end{description}

\section*{Group Work}
\begin{description}
\item[Section 2.3] 17-22
\end{description}
\end{document}
