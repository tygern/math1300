\documentclass[11pt]{article} 
\usepackage{calc}
\usepackage[margin={1in,1in}]{geometry} 
\usepackage[hwkhandout]{hwk}
\usepackage[pdftitle={2.3 Worksheet},colorlinks=true,urlcolor=blue]{hyperref}
\usepackage{graphicx, tikz}

\renewcommand{\theclass}{\textsc{math} 1300-013: Calculus I}
\renewcommand{\theauthor}{Tyson Gern}
\renewcommand{\theassignment}{The Derivative Function}
\renewcommand{\dateinfo}{Fall 2012}

\def\ww{WeBWorK }

\begin{document}
\drawtitle
\begin{enumerate}

\item Find the derivative of $f(x)=2x^2-3x+1$ using the limit
  definition of derivative.

  \vfill

\item Let $f(x)$ be a continuous function.  Draw a possible graph of
  $f(x)$ that satisfies \textbf{all} of the following conditions.
  \begin{itemize}
  \item $f(0)=0$
  \item $f'(x)\leq 0$ when $x<3$, and $f'(x)\geq 0$ for $x>3$.
  \item $\displaystyle\lim_{x\rightarrow\infty} f(x)=2$.
  \end{itemize}
  \begin{center}
    \begin{tikzpicture}[scale = 1.2]
      \draw[<->] (-1,0) -- (7.5,0);
      \draw[<->] (0,-2.5) -- (0,3);
      \draw (3, .12) -- (3, -.12) node[below] {3};
      \draw (.12, 2) -- (-.12, 2) node[left] {2};
    \end{tikzpicture}
  \end{center}


\newpage

\item Suppose that the function $h(t)$ describes the temperature of a
  cold glass of water $t$ seconds after placing the glass in a $75^\circ$ room.
  
  \begin{enumerate}
  \item What is the sign of $h'(t)$?
    \vfill
  \item What are the units of $h'(t)$?
    \vfill
  \item What is $\displaystyle\lim_{t\to\infty} h(t)$?
    \vfill
  \item On the axes below, draw graphs of $h(t)$ and $h'(t)$.
  \begin{center}
    \begin{tikzpicture}
      \draw[->] (0,0) -- (10,0) node[below] {$t$};
      \draw[<->] (0,-1.5) -- (0,5.5);
    \end{tikzpicture}
  \end{center}
  \end{enumerate}

\newpage

\item On the axes given below draw the derivatives of the following
  functions.

  \begin{center}
    \begin{tabular}{cc}
      \begin{tikzpicture}[xscale = 7/8, yscale = 5/55]
        \draw[<->] (-4,0) -- (4,0);
        \draw[<->] (0,-35) -- (0,20);
        
        \draw[thick, domain=-4:4, <->] plot[samples=200]
        function{(x+3)*(x-1)*(x-3)};
      \end{tikzpicture}
      &
      \begin{tikzpicture}[xscale = 7/13, yscale = 5/3]
        \draw[<->] (-6.5,0) -- (6.5,0);
        \draw[<->] (0,-1.5) -- (0,1.5);
        
        \draw[thick, domain=-6.5:6.5, <->] plot[samples=200]
        function{sin(x)};
      \end{tikzpicture}
      \\
      \\
      \begin{tikzpicture}[xscale = 7/8, yscale = 5/17]
        \draw[<->] (-4,0) -- (4,0);
        \draw[<->] (0,-1) -- (0,16);
        
        \draw[thick, domain=-4:4, <->] plot[samples=200]
        function{2**x};
      \end{tikzpicture}
      &
      \begin{tikzpicture}[xscale = 7/6, yscale = 5/3]
        \draw[<->] (-3,0) -- (3,0);
        \draw[<->] (0,-1.5) -- (0,1.5);
        
        \draw[thick, domain=-3:3, <->] plot[samples=200]
        function{1/(x**2+1)};
      \end{tikzpicture}
      \\
    \end{tabular}
  \end{center}
  
\item The temperature on a given day in Boulder is modeled by a
  sinusoidal curve.  There is a minimum temperature of 40$^\circ$ at
  6\textsc{am} and a maximum temperature of 70$^\circ$ at
  6\textsc{pm}.
  \begin{enumerate}
  \item Find a function $f(t)$ that gives the temperature, where $t$
    is measured in hours after midnight, and $f(t)$ is degrees.
    \vfill

    \newpage

  \item Graph the function $f(t)$ along with its derivative $f'(t)$.
    \begin{center}
      \begin{tikzpicture}[yscale = 7/80]
        \draw[->] (0,0) -- (12,0) node[below] {$t$};
        \draw[<->] (0,-30) -- (0,80);
        
      \end{tikzpicture}
    \end{center}
  \item When is the temperature growing the fastest?  When is the
    temperature growing the slowest?
    \vfill
  \end{enumerate}

\item Using shortcuts, find the derivatives of the following
  functions.
  \begin{enumerate}
  \item $f(x)=x^\pi$
    \vfill
    \newpage
  \item $g(x)=x^{-e}$
    \vfill
  \item $h(x)=\ln(4)^3+7x$
    \vfill
  \item $\displaystyle P(x)=30,000x(4)^7-4000x+25^{\frac{1}{3}}$
    \vfill
    \newpage
  \item $\displaystyle
    f(x)=23,879\left(\frac{\pi}{\sqrt{23}}\right)-7^{\ln(2300)}$
    \vfill
  \item $g(x) = \sqrt{x}$.
    \vfill
  \item $h(x) = \displaystyle\sqrt[7]{\frac{x^2\cdot x^{-3}}{x^5}}$
    \vfill
  \end{enumerate}






\end{enumerate}

\end{document}
