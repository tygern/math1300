\documentclass[11pt]{article} 
\usepackage{calc}
\usepackage[margin={1in,1in}]{geometry} 
\usepackage[hwkhandout]{hwk}
\usepackage[pdftitle={Calc 1
  Information},colorlinks=true,urlcolor=blue]{hyperref}
\usepackage{tikz, array, etex}

\renewcommand{\theclass}{\textsc{math}1300: calculus I}
\renewcommand{\theauthor}{Tyson Gern}
\renewcommand{\theassignment}{The Derivative at a Point}
\renewcommand{\dateinfo}{section 2.2}

\begin{document}
\drawtitle

\begin{enumerate}
\item Let $f(x) = 3x^2-8x+3$.  Find $f'(2)$.
  \vfill
\item Let $g(x)$ be given by the graph below.  Match the derivative
  in the table with the points $a$, $b$, $c$, $d$, and $e$.

  \begin{center}
    \begin{tabular}{m{3.5in}m{1in}}
      \begin{tikzpicture}[scale = 1.5]
        \draw[<->] (-1,0) -- (3.75,0);
        \draw[<->] (0,-2) -- (0,3);
        \draw[-] (.27,.1) -- (.27,-.1) node[below] {$a$};
        \draw[-] (.45,.1) -- (.45,-.1) node[below] {$b$};
        \draw[-] (2,.1) -- (2,-.1) node[below] {$d$};
        \draw[-] (2.5,.1) -- (2.5,-.1) node[below] {$e$};
        \draw[-] (1,.1) -- (1,-.1) node[below] {$c$};
        
        \draw[thick, domain=-.5:3.25, <->] plot[samples=200]
        function{-x*(x-1)*(x-3)} node[right]{$g(x)$};
      \end{tikzpicture}
      &
      \begin{tabular}{c|c}
        $x$ & $g'(x)$\\
        \hline
        &$0$\\
        &$1$\\
        &$2$\\
        &$-0.5$\\
        &$-2$
      \end{tabular}
    \end{tabular}
  \end{center}

  \newpage

\item Let $h(x) = \sqrt{x}$.
  \begin{enumerate}
  \item Find $h'(4)$.
    \vfill
  \item Find the equation of the tangent line to $h(x)$ at $x=4$ and
    use this to estimate $\sqrt{4.1}$.
    \vfill
  \end{enumerate}
  
  \newpage

\item Estimate the derivative of $f(x) = x^x$ at $x=2$.

  \vfill

\item Suppose that $f(x)$ is a function with $f(-5) = 7$ and $f'(-5) =
  6$.  Estimate $f(-5.5)$.
  \vfill

  \newpage

\item Find the derivative of $f(x) = \frac{1}{x^2}$ at $x=-1$.
  \vfill

\item On the axes below, draw a graph of a population function $P(t)$
  such that the average population growth between $t=2$ and $t=7$ is
  equal to the instantaneous population growth at $t=6$.  Justify your
  answer.
  \begin{center}
    \begin{tikzpicture}[scale = 1]
      \draw[->] (0,0) -- (8,0) node[below] {$t$};
      \draw[->] (0,0) -- (0,6.5) node[left] {$P(t)$};
    \end{tikzpicture}
  \end{center}
  
\end{enumerate}

\end{document}
