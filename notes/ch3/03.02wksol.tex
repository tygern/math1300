\documentclass[11pt]{article} 
\usepackage{calc}
\usepackage[margin={1in,1in}]{geometry} 
\usepackage[hwkhandout]{hwk}
\usepackage[pdftitle={Calc 1
  Notes},colorlinks=true,urlcolor=blue]{hyperref}
\usepackage{verbatim}

\renewcommand{\theclass}{\textsc{math}1300: calculus I}
\renewcommand{\theauthor}{Tyson Gern}
\renewcommand{\theassignment}{Derivatives of Exponential Functions}
\renewcommand{\dateinfo}{section 3.2}

\newcommand{\ds}{\displaystyle}

\begin{document}
\drawtitle

\begin{enumerate}
\item Find the derivatives of the following functions using the rule
  \[
  \frac{d}{dx}\left(a^x\right)=\ln(a)\cdot a^x
  \]
  and any other derivative rules that we have discussed in class.
  \begin{enumerate}
  \item $\displaystyle f(x)=e^x$ \vfill
    {\color{blue}

      \[
      f'(x) = e^x\cdot\ln(e) = e^x
      \]

    }
    \vfill

  \item $\displaystyle f(x)=10^x$ \vfill
    {\color{blue}

      \[
      f'(x) = 10^x\cdot\ln(10)
      \]

    }
    \vfill

    \newpage
  \item $\displaystyle f(x)=\frac{3^x}{5^x}+2^4$ \vfill 
    {\color{blue}

      We can simplify to obtain $f(x) = \left(\frac{3}{5}\right)^x +
      2^4$, so
      \[
      f'(x) = \left(\frac{3}{5}\right)^x\cdot \ln\left(\frac{3}{5}\right).
      \]

    }
    \vfill

  \item $\displaystyle f(x)=4\cdot 7^x+\frac{1}{x}$ \vfill
    {\color{blue}

      We can simplify to obtain $f(x) = 4\cdot 7^x + x^{-1}$, so
      \[
      f'(x) = 4\cdot7^x\cdot\ln(7) - x^{-2}.
      \]

    }
    \vfill

    \newpage
  \item $\displaystyle f(x)=4^{x+2}+x^{2e}$ \vfill
    {\color{blue}

      We can simplify to obtain $f(x) = 4^x\cdot 4^2 + x^{2e}$, so
      \[
      f'(x) = 4^x\cdot\ln(4)\cdot 4^2 + 2ex^{2e-1}.
      \]

    }
    \vfill

  \item $\displaystyle f(x)=4e^{x+7}+(\ln 4)^x$\vfill
    {\color{blue}

      We can simplify to obtain $f(x) = 4e^xe^7+(\ln 4)^x$, so
      \[
      f'(x) = 4e^xe^7 + (\ln 4)^x\cdot\ln\left(\ln 4\right).
      \]

    }
    \vfill

  \end{enumerate}
  \newpage

\item The population of the world in billions of people can be modeled
  by the function $P(t) = 6.115(1.0113)^t$, where $t$ is years since 2000.
  \begin{enumerate}
  \item Find $P(12)$, $P'(12)$, $P(30)$, and $P'(30)$.
    \vfill
    {\color{blue}

      We see that $P'(t) = 6.115(1.0113)^t\cdot \ln(1.0113)$, so
      \begin{align*}
        P(12) &= 6.115(1.0113)^{12} \approx 6.998\\
        P'(12) &= 6.115(1.0113)^{12}\cdot \ln(1.0113) \approx .0786\\
        P(30) &= 6.115(1.0113)^{30} \approx 8.566\\
        P'(30) &= 6.115(1.0113)^{30}\cdot \ln(1.0113) \approx .0963
      \end{align*}

    }
    \vfill

  \item Using units, explain what each answer tells you about the
    population of the world.
    \vfill
    {\color{blue}

      In 2012 the world's population was 6.998 billion people.\\
      Between 2012 and 2013 the world's population rose by approximately 78.6 million people.\\
      In 2030 the world's population will be 8.566 billion people.
      Between 2030 and 2031 the world's population will rise by approximately 96.3 million people.\\


    }
    \vfill

  \item What is the sign of $P''(t)$?  What does this tell you about
    the world's population, and why might this be worrying?
    \vfill
    {\color{blue}

      We see that $P''(t) = 6.115(1.0113)^t\cdot
      \ln(1.0113)\cdot\ln(1.0113) > 0$, so the graph of the world's
      population is concave up. This means that the population is
      growing at an increasing rate, so population growth will become
      more and more of a problem.

    }
    \vfill

  \end{enumerate}


\end{enumerate}

\end{document}
