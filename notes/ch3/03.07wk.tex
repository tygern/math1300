\documentclass[11pt]{article} 
\usepackage{calc}
\usepackage[margin={1in,1in}]{geometry} 
\usepackage[hwkhandout]{hwk}
\usepackage[pdftitle={Calc 1
  Notes},colorlinks=true,urlcolor=blue]{hyperref}
\usepackage{chngcntr}

\renewcommand{\theclass}{\textsc{math}1300: calculus I}
\renewcommand{\theauthor}{Tyson Gern}
\renewcommand{\theassignment}{Implicit Differentiation}
\renewcommand{\dateinfo}{section 3.7}

\newcommand{\ds}{\displaystyle}

\counterwithout{equation}{section}

\begin{document}
\drawtitle

\noindent \textit{The following problem is on the next homework assignment.  It
  is very difficult, so we will tackle it in parts on today's
  worksheet.}

\noindent \textbf{Problem 3.7.34:} Sketch the circles
\begin{align*}
  y^2+x^2 &= 1\\
  y^2 + (x-3)^2 & = 4
\end{align*}
There is a line with positive slope that is tangent to both circles.
Determine the points at which this tangent line touches each circle.

\begin{enumerate}
\item Sketch a graph of the two circles and the tangent line described
  above.  Label the point where the tangent line touches circle
  \ref{circ1} $(a,b)$ and the point where it touched circle
  \ref{circ2} $(c,d)$.
  
  \vfill 
  
  We can relate $a$ with $b$ and $c$ with $d$ as shown below.
  \begin{align}
    \label{circ1}
    b^2+a^2 &= 1\\
    \label{circ2}
    d^2 + (c-3)^2 & = 4
  \end{align}
  
  \newpage
  
\item Use equation \ref{circ1} to find $b$ in terms of $a$. Use
  equation \ref{circ2} to find $d$ in terms of $c$.
  
  \vfill
  \begin{align}
    \label{ba}
    b &= \hspace{3in}\\ 
    \label{dc}
    d &= \hspace{3in}
  \end{align}
  
\item Find $\dfrac{dy}{dx}$ for equations \ref{circ1} and \ref{circ2}.
  
  \vfill
  
  \begin{align}
    \label{slope1}
    \dfrac{dy}{dx} &= \hspace{3in}\\ 
    \label{slope2}
    \dfrac{dy}{dx} &= \hspace{3in}
  \end{align}
  
  \newpage
  
\item From our picture, we know that slope of the first circle at
  $(a,b)$ must equal the slope of the second circle at $(c,d)$.  Set
  up an equation that describes this.  Using this equation and
  equations \ref{ba} and \ref{dc}, solve for $c$ in terms of
  $a$. 
  
  \textit{Hint: Use your picture to resolve problems with absolute
    value.}
  
  
  \vfill
  
  \begin{equation}\label{ca}
    c = \hspace{3in}
  \end{equation}
  
  \newpage
  
\item Use equation \ref{ca} to substitute for $c$ in equation
  \ref{circ2} to find $d$ in terms of a.  Then use equation \ref{ba}
  to find $d$ in terms of $b$.
  \vfill
  \begin{equation}\label{db}
    d =\hspace{3in}
  \end{equation}
\item Calculate the slope between the two points, $(a,b)$ and
  $(c,d)$.  We know that this must be equal to the slope of the
  tangent line to circle \ref{circ1} at $(a,b)$.  Use equation
  \ref{slope1} to help set up this equality.
  \vfill
  \begin{equation}\label{slopes}
    =
  \end{equation}
  \newpage
  
\item Use equations \ref{ba}, \ref{db}, and \ref{ca} to get equation
  \ref{slopes} completely in terms of $a$.  Solve for $a$.
  
  \vfill
  
  \begin{equation}\label{a}
    a = \hspace{3in}
  \end{equation}
  
\item Now use equations \ref{ba}, \ref{ca}, and \ref{db} to find
  $b$, $c$, and $d$.
  
  \vfill
  
\end{enumerate}

\end{document}
