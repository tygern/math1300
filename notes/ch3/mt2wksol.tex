\documentclass[11pt]{article} 
\usepackage{calc}
\usepackage[margin={1in,1in}]{geometry} 
\usepackage[hwkhandout]{hwk}
\usepackage[pdftitle={Chapters 2 and 3 Worksheet},urlcolor=blue]{hyperref}
\usepackage{graphicx, tikz}

\renewcommand{\theclass}{\textsc{math}1300: calculus I}
\renewcommand{\theauthor}{Tyson Gern}
\renewcommand{\theassignment}{Chapters 2 and 3 Worksheet}
\renewcommand{\dateinfo}{fall 2012}

\newcommand{\ds}{\displaystyle}

\begin{document}
\drawtitle

\noindent Please answer the following questions completely, showing
all work.  Clearly state each answer.

\begin{enumerate}
\item The temperature $H$, in degrees Celsius, of a cup of coffee
  placed on the kitchen counter is given by $H=f(t)$, where $t$ is in
  minutes since the hot cup of coffee was placed on the counter.
  \begin{enumerate}
  \item Is $f'(t)$ positive or negative? Why?
    \vfill
    {\color{blue}
      $f(t)$ is decreasing, so $f'(t)$ is negative.
    }
    \vfill
  \item Is $f''(t)$ positive or negative? Why?
    \vfill
    {\color{blue}
      $f'(t)$ is increasing, so $f''(t)$ is positive.
    }
    \vfill
  \end{enumerate}

\item Identify any $x$-values at which the absolute value function,
  $f(x)=7|x+3|$, is not differentiable.
  \vfill
  {\color{blue}
  $f(x)$ has a sharp corner at $x=-3$, so $f(x)$ is not differentiable
  at $x=-3$.  It is differentiable for all other values of $x$.}
  \vfill
\newpage

\item The depth of the water, $f(t)$, in meters, in the Bay of Fundy,
  Canada, is given as a function of time, $t$, in hours after
  midnight, by the function
  \[
  f(t)=10+6\cos\left(\frac{\pi}{6}t\right).
  \]
  \begin{enumerate}
  \item What is the depth of the water at 1:00\textsc{am}?
    Include units in your answer.
    \vfill
    {\color{blue}
      The depth of the water at 1\textsc{am} is $f(1) =
      10+6\cos\left(\frac{\pi}{6}\right) = 10+3\sqrt{3}$ meters.
    }
    \vfill
  \item How quickly is the tide rising or falling at 1:00\textsc{am}?  Include
    units in your answer. \vfill
    {\color{blue}
      $f'(t) = -\pi\sin\left(\frac{\pi}{6} t\right)$, so the tide is
      changing at a rate of $f'(1) =
      -\pi\sin\left(\frac{\pi}{6}\right) = -\frac{\pi}{2}$
      meters per hour.  Thus, the depth of the water is falling at a
      rate of $\frac{\pi}{2}$ meters per hour.
    }
    \vfill
  \item Using {\bf only} your answers for parts (a) and (b), {\bf estimate} the depth
    of the water at 2:00\textsc{am}. Include units in your answer.
    \vfill
    {\color{blue}
      From part be we know that the depth of the water will fall by
      approximately $\frac{\pi}{2}$ meters between
      1\textsc{am} and 2\textsc{am}, so
      \[
      f(2) \approx 10+3\sqrt{3} - \frac{\pi}{2},
      \]
      and the depth at 2\textsc{am} is approximately $10+3\sqrt{3} -
      \frac{\pi}{2}$ meters.
    }
    \vfill
    
  \end{enumerate}

\newpage

\item Find constants $m$ and $b$ so that the function
  \[
  f(x) = \begin{cases} 3x^2-10 &\mbox{if } x\geq 2\\
    mx+b &\mbox{if } x < 2 \end{cases}
  \]
  is differentiable.
  \vfill
  {\color{blue}
  
    We know that a function must be continuous to be differentiable,
    so the two pieces of the function must have the same \(y\)-value
    at \(x=2\).  This tells us that
    \[
    3\cdot2^2 - 10 = m\cdot 2 +b.
    \]
    
    Furthermore, if \(f(x)\) is differentiable, its graph must not contain
    any sharp corners.  This means that the two pieces must have the
    same derivative where they meet up at \(x=2\).  The derivative of
    the first piece is \(6x\) and the derivative of the second piece
    is \(m\), so \[6\cdot 2 = m\] and thus \(m=12\).  Then we can
    substitute \(m=12\) into the first equation to get \[2 = 12\cdot 2 +
    b\] so \(b=-22\).  Then the function
    \[
    f(x)=\begin{cases}
      3x^2 - 10 &\mbox{if } x\geq 2\\
      12x - 22  &\mbox{if } x < 2
    \end{cases}
    \]
    is differentiable.

  }
  \vfill

\newpage

\item Find the equation of the tangent line of
  \[
  f(x)=\frac{1}{1+3x}
  \]
  at $x=0$.
\vfill
{\color{blue}
  We are looking for the tangent line $y=mx+b$.
  \[
  f'(x)=\frac{-3}{(1+3x)^2},
  \]
  so $m=f'(0)=-3$.  Then the point $(0,f(0))=(0,1)$ is on the line, so
  the tangent line is
  \[
  y=-3x+1.
  \]
}
\vfill
\newpage

\item For each of the parts (a)-(e) use the values of $g(x)$ given by
  the table below to compute each of the following derivatives.
  \[
  \begin{array}{|c||c|c|c|c|c|}
    \hline
    x&-2&-1&0&1&2\\
    \hline
    \hline
    g(x)&0&1&2&6&12\\
    \hline
    g'(x)&1&2&3&5&7\\
    \hline
  \end{array}
  \]

  \begin{enumerate}
  \item Let $f(x)=x^5+4g(x)$. Find $f'(2)$.
    \vfill
    {\color{blue}
      \[
      \begin{array}{rcl}
        f'(x)&=&5x^4+4\cdot g'(x)\\
        f'(2)&=&5\cdot 16+4\cdot 7\\
        f'(2)&=&108
      \end{array}
      \]
    }
    \vfill
  \item Let $f(x)=g(x)^2$. Find $f'(-2)$.
    \vfill
    {\color{blue}
      \[
      \begin{array}{rcl}
        f'(x)&=&2g(x)\cdot g'(x)\\
        f'(-2)&=&0
      \end{array}
      \]
    }
    \vfill
  \item Let $f(x)=3^{g(x)}$. Find $f'(2)$.
    \vfill
    {\color{blue}
      \[
      \begin{array}{rcl}
        f'(x)&=&g'(x)\cdot\ln(3)\cdot 3^{g(x)}\\
        f'(2)&=&7\ln(3)\cdot 3^{12}
      \end{array}
      \]
    }
    \vfill
    \newpage
    \[
    \begin{array}{|c||c|c|c|c|c|}
      \hline
      x&-2&-1&0&1&2\\
      \hline
      \hline
      g(x)&0&1&2&6&12\\
      \hline
      g'(x)&1&2&3&5&7\\
      \hline
    \end{array}
    \]
  \item Let $f(x)=\cos(g(x))$. Find $f'(2)$.
    \vfill
    {\color{blue}
      \[
      \begin{array}{rcl}
        f'(x)&=&g'(x)(-\sin(g(x)))\\
        f'(2)&=&-7\sin(12)
      \end{array}
      \]
    }
    \vfill
  \item Let $f(x)=g^{-1}(x)\cdot x^2$.  Find $f'(1)$.
    \vfill
    {\color{blue}
      \begin{align*}
        f'(x) & = 2x\cdot g^{-1}(x)+x^2\cdot\dfrac{1}{g'\left(g^{-1}(x)\right)}\\
        f'(1) & = 2\cdot 1\cdot g^{-1}(1)+1^2\cdot\dfrac{1}{g'\left(g^{-1}(1)\right)}\\
        f'(1) & = 2\cdot (-1)+\dfrac{1}{g'\left(-1\right)}\\
        f'(1) & = 2\cdot (-1)+\dfrac{1}{2}\\
        f'(1) & = -\dfrac{3}{2}\\
      \end{align*}
    }
    \vfill

  \end{enumerate}

\newpage

\item Calculate the derivatives of the following functions.  Show all
  work and state all rules used.
  \begin{enumerate}
  \item $f(x)=\dfrac{x}{2^x+\sin(x)}$
    \vfill
    {\color{blue}
      \[
      f'(x) = \frac{(2^x+\sin(x))-x(\ln(2)2^x+\cos(x))}{(2^x+\sin(x))^2}
      \]
    }
    \vfill
  \item $g(x)=\sqrt[7]{\arctan(x)\ln(x)}$
    \vfill
    {\color{blue}
      \[
      g'(x) =
      \left(\ln(x)\frac{1}{1+x^2}+\frac{1}{x}\arctan(x)\right)\cdot
      \frac{1}{7}\left(\arctan(x)\ln(x)\right)^{-\frac{6}{7}}
      \]
    }
    \vfill
    \newpage
  \item $h(x)=e^{\ln(3)^{\pi}} + x\cdot e^{\tan(x)}$
    \vfill
    {\color{blue}
      \[
      h'(x) = e^{\tan(x)} + x\cdot\frac{1}{\cos^2(x)}e^{\tan(x)}
      \]
    }
    \vfill
  \item $p(x)=\cos^2\left(\ln(x\arcsin(x))\right)+\sin^2(\ln(x\arcsin(x)))$
    \vfill
    {\color{blue}
      We can simplify the above equation:
      \[
      p(x)=1,
      \]
      so,
      \[
      p'(x)=0.
      \]
    }
    \vfill
    \newpage
  \item $q(x)=\ln\left(e^{4x}\right)$
    \vfill
    {\color{blue}
      We can simplify the above equation:
      \[
      q(x)=4x,
      \]
      so,
      \[
      q'(x)=4.
      \]
    }
    \vfill
  \item $r(x)=\sin(\cos(x))$
    \vfill
    {\color{blue}
      \[
      r'(x) = -\sin(x)\cdot\cos(\cos(x))
      \]
    }
    \vfill
  \end{enumerate}

\newpage

\item Given the graph of $f(x)$ below, sketch graphs of $f'(x)$ and $f''(x)$.

\begin{center}
  \begin{tikzpicture}[scale=.9]
    \def\xmin{-6}
    \def\xmax{6}
    \def\ymin{-3}
    \def\ymax{3}
    
    \draw[->] (\xmin,0) -- (\xmax,0) node[right] {$x$};
    \draw[->] (0,\ymin) -- (0,\ymax) node[above] {$f(x)$};
    
    \draw[<->, color=black, thick, domain=\xmin:\xmax] plot[samples=201]
    function{.04*(x+5)*(x+1)*(x-5)};
  \end{tikzpicture}
\end{center}
\vfill
\begin{center}
  \begin{tikzpicture}[scale=.9]
    \def\xmin{-6}
    \def\xmax{6}
    \def\ymin{-3}
    \def\ymax{3}
    
    \draw[->] (\xmin,0) -- (\xmax,0) node[right] {$x$};
    \draw[->] (0,\ymin) -- (0,\ymax) node[above] {$f'(x)$};
    
    \draw[color=blue, thick, domain=\xmin:\xmax] plot[samples=201]
    function{.12*x*x+.08*x-1};
  \end{tikzpicture}
\end{center}
\vfill
\begin{center}
  \begin{tikzpicture}[scale=.9]
    \def\xmin{-6}
    \def\xmax{6}
    \def\ymin{-3}
    \def\ymax{3}
    
    \draw[->] (\xmin,0) -- (\xmax,0) node[right] {$x$};
    \draw[->] (0,\ymin) -- (0,\ymax) node[above] {$f''(x)$};
    
    \draw[color=blue, thick, domain=\xmin:\xmax] plot[samples=201]
    function{.24*x+.08};
  \end{tikzpicture}
\end{center}
\vfill
\end{enumerate}

\end{document}
