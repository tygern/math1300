\documentclass[11pt]{article} 
\usepackage{calc}
\usepackage[margin={1in,1in}]{geometry} 
\usepackage[hwkhandout]{hwk}
\usepackage[pdftitle={Calc 1
  Notes},colorlinks=true,urlcolor=blue]{hyperref}
\usepackage{graphicx}

\renewcommand{\theclass}{\textsc{math}1300: calculus I}
\renewcommand{\theauthor}{Tyson Gern}
\renewcommand{\theassignment}{Practice for Midterm 2}
\renewcommand{\dateinfo}{spring 2012}

\newcommand{\ds}{\displaystyle}

\begin{document}
\drawtitle

The following are some practice problems for the second midterm.  As
always, the best materials to study are your online and written
homework, the chapter reviews, your class notes, and the
textbook. Once you have mastered the content contained in these
materials I encourage you practice with these problems.  Solutions to
these problems will not be posted, but you are encouraged to bring any
questions that you have to class or office hours.

\begin{enumerate}
\item Let
\[
f(x) = \begin{cases} x \sin(1/x) & \text{if } x\neq 0, \\
  0 & \text{if } x = 0. \end{cases}
\]
Then $f(x)$ is
\begin{enumerate}
\item continuous but not differentiable at $x=0$.
\item differentiable but not continuous at $x=0$.
\item neither continuous nor differentiable at $x=0$.
\item none of the above (there is insufficient information).
\end{enumerate}
%=============================================================================%


%=============================================================================%
\item Let
\[
f(x) = \arcsin(xe^x) + \arccos(xe^x)
\]
\begin{enumerate}
\item Show that $f'(x) = 0$
\item What does this tell you about $f(x)$?
\item Fully simplify $f(x)$.
\end{enumerate}
%=============================================================================%


%=============================================================================%
\item If $g(2)=3$ and $g'(2)=-4$, find $f'(2)$ for the following:
  \begin{enumerate}
  \item $f(x)=x^2-4g(x)$
  \item $f(x)=\frac{x}{g(x)}$
  \item $f(x)=x^2g(x)$
  \item $f(x)=(g(x))^2$
  \item $f(x)=x\sin(g(x))$
  \item $f(x)=x^2\ln(g(x))$
  \end{enumerate}
%=============================================================================%


%=============================================================================%
\item Suppose $f''$ and $g''$ exist and that $f$ and $g$ are concave
  up for all $x$. Are the following true or false for all such $f$ and
  $g$? Give an explanation or counterexample.
  \begin{enumerate}
  \item $f(x)+g(x)$ is concave up for all $x$.
  \item $f(x)g(x)$ is concave up for all $x$.
  \item $f(x)-g(x)$ is concave up for all $x$.
  \item $f(g(x))$ is concave up for all $x$.
  \end{enumerate}
%=============================================================================%


%=============================================================================%
\item Differentiate the following functions.
  \begin{enumerate}
  \item $f(x)=(1 + (1 + x)^{3})^{4}$
  \item $f(x)=(x+(x+x^{2})^{-3})^{-5}$
  \item $f(x)=\sin^2(\sqrt x)$
  \item $f(x)=\displaystyle{\frac{\cos(3x)}{\sin(5x)}}$
  \item $f(x)=t^{3} \sin(2t)^2$
  \item $f(x)=\sqrt {\sin(\sqrt{x})}$
  \item $f(x)=x e^{\sin(x)}$
  \end{enumerate}
%=============================================================================%


%=============================================================================%
\item Find the equation of the tangent line to the graph of $\ln(xy) =
  2x$ at the point $(1,e^{2}))$
%=============================================================================%


%=============================================================================%
\item Obtain a formula for $f'(x)$ if $f(x)$ is described as
  indicated:
  \begin{enumerate}
  \item $f(x) = \frac{1}{2+\cos (x)}$
  \item $f(x) = \frac{2-\sin (x)}{2-\cos (x)}$
  \item $f(x) = \frac{1}{x^2+1} + x^5\cos (x)$
  \item $f(x) = \frac{x\sin (x)}{1+x^2}$
  \item $f(x) = \frac{1}{x} + \frac{2}{x^2} + \frac{3}{x^3}$
  \item $f(x) = \frac{x}{(1-x)^2(1+x)^3}$
  \item $f(x) = x + \sqrt{x} + \sqrt[3]{x} + \sqrt[4]{x}$
  \item $f(x) = \frac{x}{1+\sqrt{x}} + \frac{\sqrt{x}}{1+x}$ for $x > 0$
  \item $f(x) = x^{-1} + x^{-1/2} + x^{-1/3} + x^{-1/4}$ for $x>0$
  \item $f(x) = \sqrt{x+\sqrt{x}}$ for $x>0$
  \item $f(x) = \cos(2x) - 2\sin(x)$
  \item $f(x) =(2-x^2)\cos(x^2) + 2x\sin(x^3)$
  \item $f(x) = 2x\sin^2(x^4)$
  \item $f(x) = \sin(\sin(\sin(x)))$
  \item $f(x) = \tan\left(\frac{x}{2}\right) -
    \cot\left(\frac{x}{2}\right)$ for $x \ne k\pi$, $k$ an integer
  \item $f(x) = \frac{\sin^2(x)}{\sin(x^2)}$ for $x^2 \ne k\pi$, $k$ an integer
  \item $f(x) = \sec^2(x) + \csc^2(x)$ for $x \ne \frac{k\pi}{2}$, $k$
    an integer
  \item $f(x) = x\sqrt{1+x^2}$
  \item $f(x) = \frac{x}{\sqrt{4-x^2}}$ for $|x|<2$
  \item $f(x) = (1+x)(2+x^2)^{1/2}(3+x^3)^{1/3}$ for $x^3 \ne -3$
  \item $f(x) = \frac{1}{\sqrt{1+x^2}\left(x+\sqrt{1+x^2}\right)}$
  \item $f(x) = \sqrt{x+\sqrt{x+\sqrt{x}}}$
  \end{enumerate}
%=============================================================================%


%=============================================================================%
\item Let $\displaystyle f(x) = \frac{\tan x}{x}$ if $x \ne 0$.
  Sketch a graph of $f$ over the half-open intervals
  $\left[-\frac{\pi}{4},0\right)$ and $\left(0,\frac{\pi}{4}\right]$.
  What happens to $f(x)$ as $x \to 0$?  Can you define $f(0)$ so that
  $f$ becomes continuous at $0$?  [{\sc Hint}: Recall that $\tan x =
  (\sin x)/(\cos x)$.  Moreover, what is $\lim_{x \to 0} (\sin x)/x$?]
%=============================================================================%


%=============================================================================%
\item Consider the graph of the function $f$ defined by the equation
  $f(x) = x^2 + ax + b$, where $a$ and $b$ are constants.  (The curve
  is called a \emph{parabola}, of course.)  Find values of $a$ and $b$
  such that the line $y=2x$ is tangent to this graph at the point
  $(2,4)$.
%=============================================================================%


%=============================================================================%
\item Show that the line $y=-x$ is tangent to the curve given by the
  equation $y=x^3-6x^2+8x$.  Find the point of tangency.  Does this
  tangent line intersect the curve anywhere else?
%=============================================================================%


%=============================================================================%
\item If $f(x) = (ax+b)\sin x + (cx+d)\cos x$, determine values of the
  constants $a$, $b$, $c$, $d$ such that $f'(x) = x\cos x$.
%=============================================================================%


%=============================================================================%
\item If $f(x) = \left[v(x)\right]^n$, where $n$ is a positive integer
  and $v(x)$ is differentiable everywhere, compute $f'(x)$ in terms of
  $v(x)$ and $v'(x)$.
%=============================================================================%


%=============================================================================%
\item Determine the derivative $g'(x)$ in terms of $f'(x)$ if
  \begin{enumerate}
  \item $g(x) = f\left(x^2\right)$
  \item $g(x) = f\left(\sin^2(x)\right) + f\left(\cos^2(x)\right)$
  \item $g(x) = f\left(f(x)\right)$
  \item $g(x) = f\left[f\left(f(x)\right)\right]$
  \end{enumerate}
%=============================================================================%


%=============================================================================%
\item If $y = \left(1-\sqrt{x}\right)\left(1+\sqrt{x}\right)$ for
  $x>0$, find formulas for $\frac{dy}{dx}$ and $\frac{d^2y}{dx^2}$.
%=============================================================================%


%=============================================================================%
\item Find the derivative $f'(x)$.  The function $f$ in each case is
  assumed to be defined for all real $x$ for which the given formula
  for $f(x)$ is meaningful.
  \begin{enumerate}
  \item $f(x) = \ln(1+x^2)$
  \item $f(x) = \ln\sqrt{1+x^2}$
  \item $f(x) = \ln\left(\ln(x)\right)$
  \item $f(x) = \ln\left(x^2\ln(x)\right)$
  \item $f(x) = \sqrt{x+1} - \ln\left(1+\sqrt{x+1}\right)$
  \item $f(x) = x\ln\left(x+\sqrt{1+x^2}\right) - \sqrt{1+x^2}$
  \item $f(x) = x\left[\sin\left(\ln(x)\right) -
      \cos\left(\ln(x)\right)\right]$
  \item $f(x) =  \frac{x^2 e^x \cos(x)}{(1+x^3)^7}$
  \item $f(x) = \arctan\left(e^x\right)$
  \item $f(x) = (1+x)(1+e^{x^2})$
  \item $f(x) = \frac{e^x - e^{-x}}{e^x + e^{-x}}$
  \item $f(x) = \ln\left(e^x + \sqrt{1+e^{2x}}\right)$
  \item $f(x) = \left(x+\sqrt{1+x^2}\right)^{2012}$
  \item $f(x) = \arcsin\left(\frac{x}{2}\right)$
  \item $f(x) = \arccos\left(\frac{1-x}{\sqrt{2}}\right)$
  \item $f(x) = \sqrt{x} - \arctan\left(\sqrt{x}\right)$
  \item $f(x) = \arctan(x) + \frac{1}{3}\arctan\left(x^3\right)$
  \item $f(x) = \arctan\left(x+\sqrt{1+x^2}\right)$
  \item $f(x) =
    \ln\left(\arcsin\left(\frac{1}{\sqrt{x}}\right)\right)$
  \end{enumerate}
%=============================================================================%


%=============================================================================%
\item Let $\displaystyle f(x) = \arctan(x) + \arctan(1/x)$
  \begin{enumerate}
  \item For $x>0$, find and simplify $f'(x)$.
  \item What does your result tell you about $f$?
  \end{enumerate}
%=============================================================================%


%=============================================================================%
\item Show that $\displaystyle\frac{dy}{dx} = \frac{x+y}{x-y}$ if
  $\displaystyle\arctan\left(\frac{y}{x}\right) = \ln\sqrt{x^2+y^2}$.
%=============================================================================%

\end{enumerate}

\end{document}
