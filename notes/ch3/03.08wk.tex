\documentclass[11pt]{article} 
\usepackage{calc}
\usepackage[margin={1in,1in}]{geometry} 
\usepackage[hwkhandout]{hwk}
\usepackage[pdftitle={Calc 1
  Notes},colorlinks=true,urlcolor=blue]{hyperref}
\usepackage{graphicx}

\renewcommand{\theclass}{\textsc{math}1300: calculus I}
\renewcommand{\theauthor}{Tyson Gern}
\renewcommand{\theassignment}{Hyperbolic Functions}
\renewcommand{\dateinfo}{section 3.8}

\newcommand{\ds}{\displaystyle}

\begin{document}
\drawtitle

Two functions that are used commonly in engineering are the
hyperbolic sine and hyperbolic cosine functions. These functions share
many properties with the familiar sine and cosine functions. They are
defined below:

\begin{center}
  \framebox{ $\cosh(x)=\dfrac{e^x+e^{-x}}{2},\:
    \sinh(x)=\dfrac{e^x-e^{-x}}{2}$ }
\end{center}

\begin{enumerate}
\item Using the above definitions, find the following:
  \begin{enumerate}
  \item $\sinh(0)=$
    \vfill
  \item $\cosh(0)=$
    \vfill
  \item $\sinh(-x)=$
    \vfill
  \item $\cosh(-x)=$
    \vfill
  \item What do parts (c) and (d) tell you about the symmetry (even,
    odd, etc.) of $\sinh(x)$ and $\cosh(x)$?  Note the similarities
    with sine and cosine.
    \vfill\vfill
  \end{enumerate}

  \newpage

\item Note that as $x\rightarrow\infty$, $e^{-x}$ becomes very small,
  so
  \[
  \sinh(x) = \dfrac{e^x-e^{-x}}{2} = \dfrac{e^x}{2}-\dfrac{e^{-x}}{2}
  \approx \dfrac{e^x}{2}.
  \]
  Similarly, as $x\rightarrow -\infty$, $e^x$ becomes very small, so
  \[
  \sinh(x) = \dfrac{e^x-e^{-x}}{2} = \dfrac{e^x}{2}-\dfrac{e^{-x}}{2}
  \approx -\dfrac{e^{-x}}{2}.
  \]
  Use this same method to find the behavior of $\cosh(x)$ as
  \begin{enumerate}
  \item $x\to\infty$
    \vfill
  \item $x\to -\infty$
    \vfill
  \end{enumerate}

\item Using questions 1 and 2 sketch the graphs of $\sinh(x)$ and
  $\cosh(x)$ below.
  \begin{center}
    \begin{tabular}{cc}
      \includegraphics[width=3in]{cosh.png} &
      \includegraphics[width=3in]{sinh.png}
    \end{tabular}
  \end{center}

  \newpage

\item When working with trigonometric functions, it is often useful to
  use the identity $\sin^2 (x) + \cos^2 (x) = 1$. We can find a similar
  identity for hyperbolic functions. Use the facts that
  \[
  \cosh^2(x)=\dfrac{e^{2x}+2+e^{-2x}}{4},\:
  \sinh^2(x)=\dfrac{e^{2x}-2+e^{-2x}}{4},
  \]
  to find an identity for $\cosh^2(x)-\sinh^2(x)$
  \vfill

  \newpage

\item Similar to trigonometric functions, we can define the hyperbolic
  tangent function:
  \begin{center}
    \framebox{
      $\tanh(x)=\dfrac{\sinh(x)}{\cosh(x)}=\dfrac{e^x-e^{-x}}{e^x+e^{-x}}$
    }
  \end{center}
  Use the facts that $\dfrac{d}{dx}\left(e^x\right)=e^x$ and
  $\dfrac{d}{dx}\left(e^{-x}\right)=-e^{-x}$ to find the following
  derivatives.
  \begin{enumerate}
  \item $\dfrac{d}{dx}(\sinh(x))$
    \vfill
  \item $\dfrac{d}{dx}(\cosh(x))$
    \vfill
  \item $\dfrac{d}{dx}(\tanh(x))$
    \vfill
  \end{enumerate}

\end{enumerate}

If you still have time try exercises 23 and 26 in section 3.8.

\end{document}
