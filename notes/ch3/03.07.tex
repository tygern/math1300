\documentclass[11pt]{article} 
\usepackage{calc}
\usepackage[margin={1in,1in}]{geometry} 
\usepackage[hwkhandout]{hwk}
\usepackage[pdftitle={Calc 1
  Notes},colorlinks=true,urlcolor=blue]{hyperref}

\renewcommand{\theclass}{\textsc{math}1300: calculus I}
\renewcommand{\theauthor}{Tyson Gern}
\renewcommand{\theassignment}{Implicit Functions}
\renewcommand{\dateinfo}{section 3.7}

\newcommand{\ds}{\displaystyle}

\begin{document}
\drawtitle

\section*{Preliminary}
\begin{description}
\item[Explicit] $y=f(x)$.
\item[Implicit] Example: $x^2+y^2=1$ (NOT a function)
\item[Points] How to find points on the curve.
\end{description}

\section*{Derivatives}
\begin{description}
\item[Tangent] Not a function, but still can find tangent lines.  Must
  specify point.
  \item[Derivative] Use chain rule.
\end{description}

\section*{Examples}
\begin{description}
\item[Example] $xy-x-y^2=2$, point $(6,2)$
\item[Example] $\sin(xy)=\dfrac{x}{\pi}$, point
  $\left(\dfrac{\pi}{2},\dfrac{1}{3}\right)$ (slope $\approx .02$)
\end{description}

\section*{Group Work}
\begin{description}
\item[Section 3.7] 4, 11, 23, 25
\end{description}
\end{document}
