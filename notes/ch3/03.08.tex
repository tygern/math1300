\documentclass[11pt]{article} 
\usepackage{calc}
\usepackage[margin={1in,1in}]{geometry} 
\usepackage[hwkhandout]{hwk}
\usepackage[pdftitle={Calc 1
  Notes},colorlinks=true,urlcolor=blue]{hyperref}

\renewcommand{\theclass}{\textsc{math}1300: calculus I}
\renewcommand{\theauthor}{Tyson Gern}
\renewcommand{\theassignment}{Hyperbolic Functions}
\renewcommand{\dateinfo}{section 3.8}

\newcommand{\ds}{\displaystyle}

\begin{document}
\drawtitle

\section*{Preliminary}
\begin{description}
\item[Definitions] In terms of $e^x$
  \[
  \begin{array}{cc}
    \cosh(x)=\dfrac{e^x+e^{-x}}{2} & \sinh(x)=\dfrac{e^x-e^{-x}}{2}
  \end{array}
  \]
\item[Properties] even, odd, $\cosh(0)=1$, $\sinh(0)=0$, as $x$ gets
  big/small...
\item[Identity] $\cosh^2(x)-sinh^2(x)=1$
\item[Tangent] $\tanh(x)=\dfrac{\sinh(x)}{\cosh(x)}=
  \dfrac{e^x+e^{-x}}{e^x-e^{-x}}$
\end{description}

\section*{Derivatives}
\begin{description}
\item[Familiar] Show some
  \[
  \begin{array}{ccc}
    \dfrac{d}{dx}(\cosh(x))=\sinh(x), & \dfrac{d}{dx}(\sinh(x))= \cosh(x),
    & \dfrac{d}{dx}(\tanh(x))=\dfrac{1}{\cosh^2(x)}.
  \end{array}
  \]
\end{description}

\section*{Examples}
\begin{description}
\item[Limit] $\ds\lim_{x\rightarrow\infty}\dfrac{\sinh(2x)}{\cosh(3x)}$.
\end{description}

\section*{Group Work}
\begin{description}
\item[Section 3.8] 5, 15, 26
\end{description}
\end{document}
