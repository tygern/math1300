\documentclass[11pt]{article} 
\usepackage{calc}
\usepackage[margin={1in,1in}]{geometry} 
\usepackage[hwkhandout]{hwk}
\usepackage[pdftitle={Calc 1
  Notes},colorlinks=true,urlcolor=blue]{hyperref}
\usepackage{multicol}

\renewcommand{\theclass}{\textsc{math}1300: calculus I}
\renewcommand{\theauthor}{Tyson Gern}
\renewcommand{\theassignment}{Derivatives of Polynomials}
\renewcommand{\dateinfo}{section 3.1}

\newcommand{\ds}{\displaystyle}

\begin{document}
\drawtitle
\begin{enumerate}
\item For the following problems determine if the derivative rules
  covered in this section apply.  If they do, find the derivative.  If
  they don't, state why not.
  \begin{enumerate}
  \item $f(x)=x^{\ln{3}}$
    \vfill
    {\color{blue}
      \[
      f'(x) = \ln(3)x^{\ln(3)-1}
      \]
    }
    \vfill
  \item $g(x)=4^x$
    \vfill
    {\color{blue} The derivative rules in this section do not apply
      because there is a variable in the exponent.  }
    \vfill
  \item $h(x)=x^x$
    \vfill
    {\color{blue} The derivative rules in this section do not apply
      because there is a variable in the exponent.  }
    \vfill
  \item $y=x^{\sqrt{2}}$
    \vfill
    {\color{blue}
      \[
      \frac{dy}{dx} = \sqrt{2} x^{\sqrt{2}-1}
      \]
    }
    \vfill
  \end{enumerate}
\newpage  

\item Find the derivative of the given functions.
  \begin{enumerate}
  \item $f(x)=x^{-\sin(4)}$
    \vfill
    {\color{blue}
      \[
      f'(x) = -\sin(x)x^{-\sin(4)-1}
      \]
    }
    \vfill
  \item $\ds y=\frac{3}{x^3}$
    \vfill
    {\color{blue}
      $y = 3x^{-3}$ so $\frac{dy}{dx} = 3\cdot(-3)x^{-4}$.
    }
    \vfill
  \item $y=\sqrt[7]{x}$
    \vfill
    {\color{blue}
      $y = x^{\frac{1}{7}}$, so $\frac{dy}{dx} = \frac{1}{7}x^{-\frac{6}{7}}$.
    }
    \vfill
  \item $\ds g(x)=\sqrt{\frac{3}{x^{11}}}$
    \vfill

    {\color{blue} $g(x) = \sqrt{3}\cdot x^{-\frac{11}{2}}$, so $g'(x)
      = \sqrt{3}\cdot \left(-\frac{11}{2}\right)x^{-\frac{13}{2}}$.}

    \vfill
    \newpage
  \item $\ds y=x^2-\frac{1}{x}+x^\pi-\frac{4}{\sqrt{x}}$
    \vfill
    {\color{blue}
      \[
      \frac{dy}{dx} = 2x-(-1)x^{-2} + \pi
      x^{\pi-1}-4\cdot\left(-\frac{1}{2}\right)x^{-\frac{3}{2}}
      \]
    }
    \vfill
  \item $y=x^{5}(7x+3)^2$
    \vfill
    {\color{blue}
      $y = x^5(49x^2+42x+9) = 49x^7+42x^6+9x^5$, so $\frac{dy}{dx} =
      49\cdot 7x^6 + 42\cdot 6 x^5 + 9\cdot 5 x^4$.
    }
    \vfill
  \item $\ds h(x)=x\cdot\frac{x^2-16}{x+4}$
    \vfill
    {\color{blue}
      $h(x) = x\cdot\frac{(x+4)(x-4)}{x+4} = x(x-4) = x^2-4x$, so
      $h'(x) = 2x-4$.
    }
    \vfill
  \item $\ds y=\frac{6x^{\pi+3}-2x^{\ln(4)}+x^{-\sin(47)}}{x^2}$
    \vfill
    {\color{blue}

      $y=6x^{\pi+1}-2x^{\ln(4)-2}+x^{-\sin(47)-2}$, so
      \[
      \frac{dy}{dx} = 6(\pi+1) x^{\pi} - 2(\ln(4)-2) x^{\ln(4)-3} +
      (-\sin(47)-2) x^{-\sin(47) - 3}.
      \]

    }
    \vfill
  \end{enumerate}
  \newpage
  
\item Let $f(x)=x^9-3x^7+8x^4-42x+2$.
  \begin{enumerate}
  \item Find $\ds f^{(10)}(x)$ (the tenth derivative of $f$!).
    \vfill
    {\color{blue}
      Each time we take a derivative, the degree of the polynomial
      decreases by one, so $f^{(9)}(x)$ must be of degree zero, and
      thus constant.  Then $f^{(10)}(x) = 0$.
    }
    \vfill
  \item Find $\ds f^{(9)}(x)$.
    \vfill
    {\color{blue}
      
      As above, we know that $f^{(9)}(x)$ is constant.  Then we can
      recognize a pattern:
      \begin{align*}
        f'(x) &= 10x^9 - \cdots\\
        f''(x) &= 10\cdot 9x^8 - \cdots\\
        f^{(3)}(x) &= 10\cdot 9\cdot 8x^7 - \cdots\\
        f^{(4)}(x) &= 10\cdot 9\cdot 8\cdot 7x^6 - \cdots\\
        &\vdots\\
        f^{(9)}(x) &= 10\cdot 9\cdot 8\cdot 7\cdot 6\cdot 5\cdot 4\cdot 3\cdot 2\cdot 1 = 10!\\
      \end{align*}

    }
    \vfill
  \end{enumerate}

\item A ball is thrown straight up off the roof of a building.  Its
  height in feet after $t$ seconds is given by
  $f(t)=-8t^2+32t+96$. Answer the following questions.
  \begin{enumerate}
  \item How tall is the building?
    \vfill
    {\color{blue}
      The ball is on the roof of the building at $t = 0$, so the
      height of the building is equal to $f(0)$.  The building is 96
      feet tall.
    }
    \vfill
    \newpage
  \item When does the ball hit the ground?
    \vfill
    {\color{blue}
      The ball hits the ground on the first $t$-value where $f(t)=0$.
      If we solve $-8t^2+32t+96 = 0$ we get $t=-2$ and $t=6$, so the
      ball hits the ground after 6 seconds.
    }
    \vfill
  \item How hard was the ball thrown?
    \vfill
    {\color{blue}
      Another way to ask this question is to ask for the velocity of
      the ball at $t=0$, or $v(0)$.  Using shortcuts we find that
      $v(t) = f'(t) = -16t+32$, so $v(0) = 32$.  Then the ball was
      thrown with a velocity of $32$ feet per second.
    }
    \vfill
  \item When does the ball reach the maximum height?  \vfill
    {\color{blue} 
      The ball reaches its maximum height when it stops rising and
      starts falling to the earth.  When this happens, the velocity of
      the ball is zero, so we want to know the $t$-value where $v(t) =
      0$.  If we solve $-16t+32 = 0$ we get $t=2$, so the ball reaches
      its maximum height after 2 seconds.
    }
    \vfill
  \end{enumerate}
  
  \newpage

\item Using a graph to help you, find the equations of all line
  through the origin tangent to the parabola $y = x^2 + 9$.
    \vfill
    {\color{blue}

      If we draw a graph of the parabola, we see that there are two
      possible lines that could go through the origin and be tangent
      to the graph (Make sure to draw a picture of this).  Since these
      lines go through the origin their $y$-intercept is zero, so they
      are of the form $y=mx$.

      Since the lines are tangent to the graph, they must have the
      same slope where they touch.  The derivative of the parabola is
      $\frac{dy}{dx} = 2x$, so we must have $m=2x$ where the line and
      the graph meet.

      The graphs also must have the same $y$-coordinate where they
      meet, so we must have $mx=x^2+9$.  Then we can substitute $m=2x$
      from above to get $2x^2=x^2+9$.  If we solve this for $x$ we get
      $x=\pm 3$, so the lines $y=-3x$ and $y=3x$ must go through the
      origin and be tangent to the graph of $y=x^2+9$.
      
    }
    \vfill


\end{enumerate}  
\end{document}
