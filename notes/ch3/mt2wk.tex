\documentclass[11pt]{article} 
\usepackage{calc}
\usepackage[margin={1in,1in}]{geometry} 
\usepackage[hwkhandout]{hwk}
\usepackage[pdftitle={Chapters 2 and 3},
colorlinks=true,urlcolor=blue]{hyperref}
\usepackage{graphicx, tikz}

\renewcommand{\theclass}{\textsc{math}1300: calculus I}
\renewcommand{\theauthor}{Tyson Gern}
\renewcommand{\theassignment}{Chapters 2 and 3 Worksheet}
\renewcommand{\dateinfo}{fall 2012}

\newcommand{\ds}{\displaystyle}

\begin{document}
\drawtitle

\noindent Please answer the following questions completely, showing
all work.  Clearly state each answer.

\begin{enumerate}
\item The temperature $H$, in degrees Celsius, of a cup of coffee
  placed on the kitchen counter is given by $H=f(t)$, where $t$ is in
  minutes since the hot cup of coffee was placed on the counter.
  \begin{enumerate}
  \item Is $f'(t)$ positive or negative? Why?\vfill
  \item Is $f''(t)$ positive or negative? Why?\vfill
  \end{enumerate}

\item Identify any $x$-values at which the absolute value function,
  $f(x)=7|x+3|$, is not differentiable.

\vfill
\newpage

\item The depth of the water, $f(t)$, in meters, in the Bay of Fundy,
  Canada, is given as a function of time, $t$, in hours after
  midnight, by the function
  \[
  f(t)=10+6\cos\left(\frac{\pi}{6}t\right).
  \]
  \begin{enumerate}
  \item What is the depth of the water at 1:00\textsc{am}?
    Include units in your answer. \vfill
  \item How quickly is the tide rising or falling at 1:00\textsc{am}?  Include
    units in your answer. \vfill
  \item Using {\bf only} your answers for parts (a) and (b), {\bf estimate} the depth
    of the water at 2:00\textsc{am}. Include units in your answer. \vfill
  \end{enumerate}

\newpage

\item Find constants $m$ and $b$ so that the function
  \[
  f(x) = \begin{cases} 3x^2-10 &\mbox{if } x\geq 2\\
    mx+b &\mbox{if } x < 2 \end{cases}
  \]
  is differentiable.

\vfill
\newpage

\item Find the equation of the tangent line of
  \[
  f(x)=\frac{1}{1+3x}
  \]
  at $x=0$.

\vfill
\newpage

\item For each of the parts (a)-(e) use the values of $g(x)$ given by
  the table below to compute each of the following derivatives.
  \[
  \begin{array}{|c||c|c|c|c|c|}
    \hline
    x&-2&-1&0&1&2\\
    \hline
    \hline
    g(x)&0&1&2&6&12\\
    \hline
    g'(x)&1&2&3&5&7\\
    \hline
  \end{array}
  \]

  \begin{enumerate}
  \item Let $f(x)=x^5+4g(x)$. Find $f'(2)$.
    \vfill
  \item Let $f(x)=g(x)^2$. Find $f'(-2)$.
    \vfill
  \item Let $f(x)=3^{g(x)}$. Find $f'(2)$.
    \vfill
    \newpage
    \[
    \begin{array}{|c||c|c|c|c|c|}
      \hline
      x&-2&-1&0&1&2\\
      \hline
      \hline
      g(x)&0&1&2&6&12\\
      \hline
      g'(x)&1&2&3&5&7\\
      \hline
    \end{array}
    \]
  \item Let $f(x)=\cos(g(x))$. Find $f'(2)$.
    \vfill
  \item Let $f(x)=g^{-1}(x)\cdot x^2$.  Find $f'(1)$.
    \vfill

  \end{enumerate}

\newpage

\item Calculate the derivatives of the following functions.  Show all
  work and state all rules used.
  \begin{enumerate}
  \item $f(x)=\dfrac{x}{2^x+\sin(x)}$
    \vfill
  \item $g(x)=\sqrt[7]{\arctan(x)\ln(x)}$
    \vfill
    \newpage
  \item $h(x)=e^{\ln(3)^{\pi}} + x\cdot e^{\tan(x)}$
    \vfill
  \item $p(x)=\cos^2\left(\ln(x\arcsin(x))\right)+\sin^2(\ln(x\arcsin(x)))$
    \vfill
    \newpage
  \item $q(x)=\ln\left(e^{4x}\right)$
    \vfill
  \item $r(x)=\sin(\cos(x))$
    \vfill
  \end{enumerate}

\newpage

\item Given the graph of $f(x)$ below, sketch graphs of $f'(x)$ and $f''(x)$.

\begin{center}
  \begin{tikzpicture}[scale=.9]
    \def\xmin{-6}
    \def\xmax{6}
    \def\ymin{-3}
    \def\ymax{3}
    
    \draw[->] (\xmin,0) -- (\xmax,0) node[right] {$x$};
    \draw[->] (0,\ymin) -- (0,\ymax) node[above] {$f(x)$};
    
    \draw[<->, color=black, thick, domain=\xmin:\xmax] plot[samples=201 ]
    function{.04*(x+5)*(x+1)*(x-5)};
  \end{tikzpicture}
\end{center}
\vfill
\begin{center}
  \begin{tikzpicture}[scale=.9]
    \def\xmin{-6}
    \def\xmax{6}
    \def\ymin{-3}
    \def\ymax{3}
    
    \draw[->] (\xmin,0) -- (\xmax,0) node[right] {$x$};
    \draw[->] (0,\ymin) -- (0,\ymax) node[above] {$f'(x)$};
    
%    \draw[color=blue, thick, domain=\xmin:\xmax] plot[id=deriv]
%    function{.12*x*x+.08*x-1} node[right] {$f'(x)$};
  \end{tikzpicture}
\end{center}
\vfill
\begin{center}
  \begin{tikzpicture}[scale=.9]
    \def\xmin{-6}
    \def\xmax{6}
    \def\ymin{-3}
    \def\ymax{3}
    
    \draw[->] (\xmin,0) -- (\xmax,0) node[right] {$x$};
    \draw[->] (0,\ymin) -- (0,\ymax) node[above] {$f''(x)$};
    
  \end{tikzpicture}
\end{center}
\vfill
\end{enumerate}

\end{document}
